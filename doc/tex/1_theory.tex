\chapter{Wprowadzenie do tematu}


\section{Knuth}

\emph{Drzewo} zdefiniujemy formalnie jako zbiór $T$ jednego lub więcej elementów zwanych \emph{węzłami}, takich że:

\begin{enumerate}
 \item istnieje jeden wyróżniony węzeł zwany \emph{korzeniem} drzewa, $root(T)$; oraz
 \item pozostałe węzły (z wyłączeniem korzenia) są podzielone na $m \geq 0$ rozłącznych zbiorów $T_{1},\ldots, T_{m}$,
	z których każdy jest drzewem. Drzewa $T_{1},\ldots, T_{m}$ nazywane są \emph{poddrzewami} korzenia.
\end{enumerate}

Z naszej definicji wynika, że każdy węzeł drzewa jest korzeniem pewnego poddrzewa zawartego w większym drzewie.
Liczna poddrzew węzła jest nazywana stopniem tego węzła.
Węzeł o stopniu zero nazywamy \emph{liściem} lub \emph{węzłem zewnętrznym}. 
Węzeł nie będący liściem nazywamy \emph{węzłem wewnętrznym}. 
Poziom węzła jest zdefiniowany rekurencyjnie: poziom korzenia $root(T)$ równa się zero,
a poziom każdego innego węzła jest o jeden wiekszy niż poziom korzenia w najmniejszym 
\todo{w sensie ilości węzłów} zawierającym go poddrzewie.

Jeśli względny porządek poddrzew $T_{1},\ldots, T_{m}$ w części (2) definicji jest istotny,
to mówimy, że drzewo jest \emph{uporządkowane}. \todo{...}
Jeśli nie chcemy rozróżniać drzew, które różnią się jedynie kolejnością poddrzew, mówimy o drzewach zorientowanych

\emph{Lasem} nazywamy zbiór zera lub więcej rozłącznych drzew. 
Innym sposobem wyprowadzenia części (2) definicji jest stwierdzenie: \textit{węzły drzewa z wyłączeniem korzenia tworzą las}.

Abstrakcyjnie drzewa i lasy niewiele się różnią.
Jeśli usuniemy korzeń drzewa, otrzymamy las.
W drugą stronę, jeśli wszystkie drzewa w lesie uczynimy poddrzewami jednego dodatkowego węzła,
to otrzymamy drzewo.
Z tego powodu słowa ,,drzewa'' i ,,lasy'' są przy nieformalnym omawianiu struktur danych używane niemal wymiennie.

\section{Czym jest drzewo}
\index{drzewo|textbf}
Drzewo to bardzo powszechnie używane w informatyce pojęcie. W zależności od zastosowania może być różnie zdefiniowane.

\paragraph{Drzewo jako graf}
\index{graf}
\paragraph{Drzewo jako struktura rekurencyjna}
\index{rekurencja}


Tu: matematyczny opis drzew, ich podstawowe własności, omówienie terminów:
\begin{description}
    \item
    \item[drzewo]
    \item[las]
    \item[rodzic]
    \item[przodek]
    \item[korzeń]
    \item[dziecko]
    \item[potomek]
    \item[głębokość, mini]
\end{description}

\section{Podział drzew}
\subsection{Jednorodność}
Warto by dodać o drzewach jednorodnych oraz niejednorodnych.

\subsection{Kolejność}


\section{Podstawowe algorytmy dla drzew}
\subsection{Wyszukiwanie w głąb}
\index{drzewo!wyszukiwanie!w głąb}
\subsection{Wyszukiwanie wszerz}
\index{drzewo!wyszukiwanie!wszerz}


\section{Różnice pomiędzy drzewami w algorytmice a w bazach danych}

Nie ma sensu coś takiego jak drzewo czerwono-czarne gdyż w drzewach w bazach danych chodzi o strukturę a nie o optymalizację czasu dostępu.

\section{Tematy porównania - operacje}



\subsection{Operacje}
\subsection{Pobranie korzeni}
\index{drzewo!korzeń}
\subsection{Pobranie przodków}
\index{drzewo!przodekowie}
\subsection{Pobranie rodzica}
\index{drzewo!rodzic}
\subsection{Pobranie dzieci}
\index{drzewo!dzieci}
\subsection{Pobranie potomków}
\index{drzewo!potomkowie}
\subsection{Pobranie (najmłodszego) wspólnego przodka}



% \begin{description}
%   \item[pobranie rodzica] polega na pobraniu rodzica bieżącego elementu
%   \item[pobranie przodków] polega na pobraniu rodzica, rodzica rodzica aż do korzenia elementów
% \end{description}

\section{Dostępność}
\subsection{Mapowanie relacyjno-obiektowe}
\subsection{W zależności od bazy danych}


\section{Przyjęte założenia}
\todo{więcej wypisać, zastanowić się co z tym zrobić}

\paragraph{Tabela zawiera tylko jedną kolumnę z danymi użytkownika} 
Bez problemu można dodać więcej kolumn do tabeli. Obecność wyłącznie kolumny \texttt{name} zwiększa czytelność przykładów.
\paragraph{Zapytania wyłącznie związane z hierarchiczną strukturą danych} 

\paragraph{Przedstawione fragmenty kodu} 
W opisach metod znajdują się fragmenty kodu je realizujące operacje na nich. 
Tam gdzie to jest możliwe (czyli metoda jest dostępna ) została zaprezentowana składnia PostgreSQL\index{PostgreSQL}.
Fragmenty kodu \todo{napisać o tym że gdzieniegdzie pojawiają się \texttt{:name}} wymagające parametrów...

\paragraph{Operacje z wykorzystaniem \texttt{id}} Wszystkie operacje wykorzystują sztucznie utworzony klucz główny\index{klucz!główny} \texttt{id}. 

\paragraph{Terminologia anglojęzyczna} \todo{nazwać to jakoś normalnie}
Literatura dotycząca informatyki jest dzisiaj zdominowana przez publikacje w języku angielskim. Dlatego przy każdym ważnym lub nietrywialnym w tłumaczeniu terminie została podana jego angielska wersja. 

