\section{Metoda zmaterializowanych ścieżek}
\index{metoda!drzew prefiksowych|textbf}\index{metoda!lineage|see{drzew prefiksowych}}

Metoda ta polega na przechowywaniu ciągu identyfikatorów wszystkich węzłów pomiędzy korzeniem a danym węzłem.
Są one przechowywane w pojedyńczym polu danego rekordu. 
Identyfikatorem może być dowolne unikalne pole\todo{term:pole?} tabeli.
Dobrym rozwiązaniem jest numeryczny klucz główny, lecz często bywa też stosowana unikalna nazwa węzła.


Ten opis nie wymusza konkretnej implementacji. 
Listę elementów można przechowywać na wiele sposobów. 
Przykładowo ciąg identyfikatorów całkowitoliczbowych \texttt{4, 8, 15, 16, 23, 42} można przechować jako:
\begin{itemize}
 \item tablicę\todo{term:array} np. \verb|{4, 8, 15, 16, 23, 42}|
% wady: dostpne tylko w wybranych bazach danych, niedostpne dla ORM, potencjalne problemy z indeksami
 \item napis z elememtami rozdzielonymi separatorami. 
    Przykładowo dla separatora~\verb|'.'| będzie to napis \verb|'4.8.15.16.23.42'|. 
    Ważne jest aby znak (lub znaki) będące separatorem nie mogły występować w identyfikatorze.
 \item napis z elememtami przekształconymi na napis stałej długości. 
    Przykładowo dla długości 3 będzie to \verb|'004008015016023042'|. 
    Wadą tej metody jest konieczność jednorazowego długości napisu dla identyfikatora.
    Dla długości $n$ można przechować tylko $10^n-1$ węzłów\footnote{zakładając, że wszystkie dopuszczalne wartości znajdą się w użyciu, a usuwanie elementow tworzy luki}.
    Natomiast zastosowanie dużej $n$ zwiększa zużycie zasobów SZDB, oraz spowalnia działanie.
\end{itemize}


Metoda jest też zwana Lineage.


\operacja{Reprezentacja}

\begin{verbatim}[table] trie
>n1 _
id | path    | name
1  | 1       | Bazy Danych

>n2 n1
id | path    | name
2  | 1.2     | Obiektowe

>n3 n2
id | path    | name
3  | 1.2.3   | db4o

>n4 n1
id | path    | name
4  | 1.4     | Relacyjne

>n5 n4
id | path    | name
5  | 1.4.5   | Open Source

>n7 n5
id | path    | name
6  | 1.4.5.6 | PostgreSQL

>n8 n5
id | path    | name
7  | 1.4.5.7 | MySQL

>n9 n5
id | path    | name
8  | 1.4.5.8 | SQLite

>n6 n4
id | path    | name
9  | 1.4.9   | Komercyjne

>n10 n1
id | path    | name
10 | 1.10    | XML

\end{verbatim}


\operacja{Reprezentacja w SQL}
%! method-sql pathenum.create

\operacja{Wstawianie danych}
%! method-sql pathenum.insert

\operacja{Pobranie korzeni}
%! method-sql pathenum.roots

\operacja{Pobranie rodzica}
%! method-sql pathenum.parent

\operacja{Pobranie dzieci}
%! method-sql pathenum.children

\operacja{Pobranie przodków}
%! method-sql pathenum.ancestors

\operacja{Pobieranie potomków}
%! method-sql pathenum.descendants


\temat{Wyniki}

\begin{table}[h!]
  \caption{Wyniki reprezentacji krawędziowej}
  \begin{center}
%! result-table pathenum deep3
  \end{center}
\end{table}

\begin{figure}[h!t]
  \caption{Wyniki reprezentacji krawędziowej}
  \label{fig:img_chart_simple}
  \begin{center}
%! result-chart pathenum deep3
  \end{center}
\end{figure}