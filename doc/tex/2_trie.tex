\section{Metoda zmaterializowanych ścieżek}
\index{metoda!drzew prefiksowych|(textbf}\index{metoda!lineage|see{drzew prefiksowych}}

% Rodowód, pochodzenie
%Metoda jest też zwana Lineage.

Koncepcja stojąca za tą reprezentacją jest bardzo stara i powszechnie znana.
Przykładowo ksiązka w biblotece może zostać zaklasyfikowana jako \textit{Epika / Powieść / Powieść fantastyczna}.
Biolodzy dla opisu różnorodności życia na Ziemi stosują drzewo filogenetyczne.
Za pomocą niego organizm w nim może zostać opisany jako
\textit{eukarionty / zwierzęta / strunowce / kręgowce / ssaki / ssaki żyworodne / łożyskowce / drapieżne / kotowate / felis / kot domowy}\footnote{
    Ta reprezentacja może wyglądać podobnie do ścieżek dostępu.
    Jest to złudne podobieństwo.
    W tej metodzie z wartości każdej częsci ścieżki można wywnioskować wszystkich jej przodków.
    Przykładowo, mając doczynienia z kotem domowym wiemy, że jest ssakiem.
    Natomiast nazwa katalogu \texttt{bin} nie daje nam informacji o tym czy jesteśmy w katalogu \texttt{/bin}, \texttt{/usr/bin} czy \texttt{/home/user/bin}.
    %Nie należy mylić tej metody z 
}.
Stąd też wywodzi się popularna angielska nazwa tej reprezentacji --- \textit{lineage} --- oznaczająca rodowód, pochodzenie.





Metoda ta polega na przechowywaniu ciągu identyfikatorów wszystkich węzłów pomiędzy korzeniem a danym węzłem.

Są one przechowywane w pojedyńczym polu danego rekordu.
Identyfikatorem może być dowolne unikalne pole \todo{term:pole?} tabeli.
Dobrym rozwiązaniem jest numeryczny klucz główny, lecz często bywa też stosowana unikalna nazwa węzła.


Ten opis nie wymusza konkretnej implementacji.
Listę elementów można przechowywać na wiele sposobów.
Jedynym wymogiem jest to by można było dokonać prostego pobrania identyfikatorów poszczególnych węzłów.
Przykładowo ciąg identyfikatorów całkowitoliczbowych \texttt{4, 8, 15, 16, 23, 42} można przechować jako:
\begin{itemize}
 \item napis z elememtami rozdzielonymi separatorami. 
    Przykładowo dla separatora~\verb|'.'| będzie to napis \verb|'4.8.15.16.23.42'|. 
    Ważne jest aby znak lub ciąg znaków będący separatorem nie mogły występować w identyfikatorze.
    %W praktyce najczęsciej stosuje się do tego celu 
 \item napis z elememtami przekształconymi na napis stałej długości. 
    Przykładowo dla długości 3 będzie to \verb|'004008015016023042'|.
    Wadą tej metody jest konieczność jednorazowego wyboru długości napisu dla identyfikatora.
    Dla długości $n$ można przechować tylko $10^n-1$ węzłów%\footnote{zakładając, że wszystkie dopuszczalne wartości znajdą się w użyciu, a usuwanie elementow tworzy luki}.
    Natomiast zastosowanie dużej $n$ zwiększa zużycie zasobów SZDB oraz spowalnia działanie.
 \item tablicę \todo{term:array} np. \verb|{4, 8, 15, 16, 23, 42}|
    Możliwość przechowywania tablic wewnątrz rekordu nie jest powszechna wśród implementacji relacyjnych baz danych.
    Jako pozytywny wyjątek może posłórzyć PostgreSQL oferujący typ danych \verb|ARRAY|.
    % dostpne tylko w wybranych bazach danych, niedostpne dla ORM, potencjalne problemy z indeksami
    % http://www.postgresql.org/docs/9.0/interactive/arrays.html
\end{itemize}

W praktyce najczęściej stosowaną implementacją jest pierwsza z wymienionych,
więc poniżej zostanie to ona zostanie przedstawiona.



% TODO:
% - nie jest nawet w pierwszej postaci normmalnej
% - można zamiast sztucznych (primary key) identyfikatorów stosować unikalne nazwy
%   - przykładem tego są ścierzki plików, przykładowo /usr/bin/foo
%   - tak sugeruje się używać ltree




\operacja{Reprezentacja}

\begin{verbatim}[table] trie
>n1 _
id | path    | name
1  |         | Bazy Danych

>n2 n1
id | path    | name
2  | 1.      | Obiektowe

>n3 n2
id | path    | name
3  | 1.2.    | db4o

>n4 n1
id | path    | name
4  | 1.      | Relacyjne

>n5 n4
id | path    | name
5  | 1.4.    | Open Source

>n7 n5
id | path    | name
6  | 1.4.5.  | PostgreSQL

>n8 n5
id | path    | name
7  | 1.4.5.  | MySQL

>n9 n5
id | path    | name
8  | 1.4.5.  | SQLite

>n6 n4
id | path    | name
9  | 1.4.    | Komercyjne

>n10 n1
id | path    | name
10 | 1.      | XML

\end{verbatim}

\temat{Operacje}

\operacja{Reprezentacja w SQL}

W tej metodzie ważne jest by właściwie oszacować wymaganą długość pola \texttt{path}.
W prakryce nie powinna być ona mniejsza niż $(\lceil log_{10}(maksymalna ilość rekordów) \rceil + 1)(wysokość drzewa)$.

%! method-sql pathenum.create

\operacja{Wstawianie danych}
%! method-sql pathenum.insert

\operacja{Pobranie korzeni}



%! method-sql pathenum.roots

\operacja{Pobranie rodzica}
%! method-sql pathenum.parent

\operacja{Pobranie dzieci}
%! method-sql pathenum.children

\operacja{Pobranie przodków}
%! method-sql pathenum.ancestors

\operacja{Pobieranie potomków}
%! method-sql pathenum.descendants



\temat{Uwagi}

Dostosowania metody do przyjętego interfacu zmniejsza jej wydajność.
Praktycznie wszystkie zapytania zawierają podzapytanie które dla danego identyfikatora węzła \todo{term} pobierają jego ścieżkę.
Ta operacja jest w prawdzie bardzo szybka (kolumna \texttt{id} jest kluczem głównym) ale mimo wszystko mają wpływ na ogólną wydajność.
W razie stosowania tej metody warto sie zastanowić nad używaniem ścieżki (\texttt{path}) jako parametrów metod. 

Dodatkowo odwoływanie się do węzłów za pomocą ścieżek a nie identyfikatorów daje większe możliwości.
W takiej sytuacji można wykonać dodatkowe operacje bez konieczności wykonywania zapytań w bazie danych.
Są to:
\begin{itemize}
    \item
        Pobranie głebokości węzła w drzewie (wystarczy policzyć ilość elementów w ścieżce, zero oznacza korzeń\todo{dopisać})
    \item
        Wyznaczenie pierwszego wspólnego potomka
\end{itemize}

\temat{Wydajność}

\begin{qxtab}{pathenum}{Wydajność reprezentacji zmaterializowanych ścieżek}
%! result-table pathenum deep3
\end{qxtab}

\begin{qxfig}{pathenum}{Wydajność reprezentacji zmaterializowanych ścieżek}
%! result-chart pathenum deep3
\end{qxfig}


\index{metoda!drzew prefiksowych|)}
