\section{Metoda zmaterializowanych ścieżek}
\index{metoda!drzew prefiksowych|(textbf}\index{metoda!lineage|see{drzew prefiksowych}}

% Rodowód, pochodzenie
%Metoda jest też zwana Lineage.

Koncepcja stojąca za tą reprezentacją jest bardzo stara i powszechnie znana.
Przykładowo książka w bibliotece może zostać zaklasyfikowana jako \textit{Epika / Powieść / Powieść fantastyczna}.
Biolodzy dla opisu różnorodności życia na Ziemi stosują drzewo filogenetyczne.
Za pomocą niego organizm w nim może zostać opisany jako
\textit{eukarionty / zwierzęta / strunowce / kręgowce / ssaki / ssaki żyworodne / łożyskowce / drapieżne / kotowate / felis / kot domowy}.
Stąd też wywodzi się popularna angielska nazwa tej reprezentacji --- \textit{lineage} --- oznaczająca rodowód, pochodzenie.


    Ta reprezentacja może wyglądać podobnie do ścieżek dostępu w systemach plików.
    Jest to złudne podobieństwo.
    W tej metodzie z wartości każdej części ścieżki można wywnioskować wszystkich jej przodków.
    Przykładowo, mając do czynienia z kotem domowym wiemy, że jest ssakiem.
    Natomiast nazwa katalogu \texttt{bin} nie daje nam informacji o tym czy jesteśmy w katalogu \texttt{/bin}, \texttt{/usr/bin} czy \texttt{/home/user/bin}.
    %Nie należy mylić tej metody z 



Metoda ta polega na przechowywaniu ciągu identyfikatorów wszystkich węzłów pomiędzy korzeniem a danym węzłem.
Są one przechowywane w pojedynczym atrybucie danego rekordu.
Czyli de facto przechowuje się wynik zapytania o przodków czyli ścieżkę.
Stąd pochodzi najpopularniejsza jej nazwa: \emph{metoda zmaterializowanych ścieżek} \eng{materialized path}\footnote{
    Sama nazwa sugeruje podobieństwo do \emph{zmaterializowanych widoków} \eng{materialized views}.
    W obu przypadkach wynik zapytania jest przechowywany by przyśpieszyć kolejne operacje.
}.


Identyfikatorem może być dowolny, unikalny atrybut \todo{term:pole?} tabeli.
Dobrym rozwiązaniem jest numeryczny klucz główny, lecz często bywa też stosowana unikalna nazwa węzła.


Ten opis nie wymusza konkretnej implementacji.
Listę elementów można przechowywać na wiele sposobów.
Jedynym wymogiem jest to by można było dokonać prostego pobrania identyfikatorów poszczególnych węzłów.
Przykładowo ciąg identyfikatorów całkowitoliczbowych \texttt{4, 8, 15, 16, 23, 42} można przechować jako:
\begin{itemize}
 \item napis z elementami rozdzielonymi separatorami. 
    Przykładowo dla separatora~\verb|'.'| będzie to napis \verb|'4.8.15.16.23.42'|.
    Ważne jest aby znak lub ciąg znaków będący separatorem nie mogły występować w identyfikatorze.
    %W praktyce najczęsciej stosuje się do tego celu 
 \item napis z elementami przekształconymi na napis stałej długości. 
    Przykładowo dla długości 3 będzie to \verb|'004008015016023042'|.
    Wadą tej metody jest konieczność jednorazowego wyboru długości napisu dla identyfikatora.
    Dla długości $n$ można przechować tylko $10^n-1$ węzłów. %\footnote{zakładając, że wszystkie dopuszczalne wartości znajdą się w użyciu, a usuwanie elementow tworzy luki}.
    Natomiast zastosowanie dużej $n$ zwiększa zużycie zasobów SZDB oraz spowalnia działanie.
 \item tablicę \todo{term:array} np. \verb|{4, 8, 15, 16, 23, 42}|
    Możliwość przechowywania tablic wewnątrz rekordu nie jest powszechna wśród implementacji relacyjnych baz danych.
    Jako pozytywny wyjątek może posłużyć PostgreSQL oferujący typ danych \verb|ARRAY|.
    % dostpne tylko w wybranych bazach danych, niedostpne dla ORM, potencjalne problemy z indeksami
    % http://www.postgresql.org/docs/9.0/interactive/arrays.html
\end{itemize}

W praktyce najczęściej stosowanym podejściem jest pierwsze z wymienionych,
więc to ono zostanie przedstawione poniżej.

Dla ułatwienia implementacji i zmniejszenia duplikacji danych została wprowadzona mała zmiana.
Skoro w każdym węźle jest już atrybut zawierający jego identyfikator to można go pominąć w ścieżce.
Dzięki temu nie trzeba ręcznie pobierać z sekwencji nowego \texttt{id} a można pozostawić to zadanie bazie danych.
W efekcie ścieżka jest w postaci \verb|'4.8.15.16.23.'| a \verb|42| znajduje się w atrybucie \verb|id|.



% TODO:
% - nie jest nawet w pierwszej postaci normmalnej
% - można zamiast sztucznych (primary key) identyfikatorów stosować unikalne nazwy
%   - przykładem tego są ścierzki plików, przykładowo /usr/bin/foo
%   - tak sugeruje się używać ltree




%\operacja{Reprezentacja}

\begin{verbatim}[table] trie
>n1 _
id | path    | name
1  |         | Bazy Danych

>n2 n1
id | path    | name
2  | 1.      | Obiektowe

>n3 n2
id | path    | name
3  | 1.2.    | db4o

>n4 n1
id | path    | name
4  | 1.      | Relacyjne

>n5 n4
id | path    | name
5  | 1.4.    | Open Source

>n7 n5
id | path    | name
6  | 1.4.5.  | PostgreSQL

>n8 n5
id | path    | name
7  | 1.4.5.  | MySQL

>n9 n5
id | path    | name
8  | 1.4.5.  | SQLite

>n6 n4
id | path    | name
9  | 1.4.    | Komercyjne

>n10 n1
id | path    | name
10 | 1.      | XML

\end{verbatim}

\temat{Operacje}

\operacja{Reprezentacja w SQL}

W tej metodzie ważne jest by właściwie oszacować wymaganą długość atrybutu \texttt{path}.
W prakryce nie powinna być ona mniejsza niż:
\begin{displaymath}
    (\lceil log_{10}(|T|) \rceil + 2) \times height(T)
\end{displaymath}

%! method-sql pathenum.create

\operacja{Wstawianie danych}

%Wstawianie danych jest prostą operacją.

Podczas wstawiania potrzebne są informacje o wszystkich właściwych przodkach wstawianego węzła.
Na szczęście znajdują się one w pojedynczym rekordzie --- rodzicu.
Wystarczy więc je pobrać i odpowiednio sformatować.

%! method-sql pathenum.insert

\operacja{Pobranie korzeni}

Jako, że korzenie nie posiadają przodków to ich ścieżka jest pusta. 

%! method-sql pathenum.roots

\operacja{Pobranie rodzica}

%\todo{przedstawiona tu metoda }

%! method-sql pathenum.parent

Ta implementacja nie jest ekstremalnie szybka, ale ma dwie zalety ---
daje się zaimplementować w każdej z obsługiwanych baz danych oraz wymagane zapytanie jest wszędzie takie samo.



\operacja{Pobranie dzieci}

Podczas wstawiania danych ścieżka rodzica jest konkatenowana z jego identyfikatorem.
Wynik jest zapisywany w ścieżce dziecka.
Czyli z węzła można wyliczyć wartość ścieżki jego dzieci.
%W efekcie mając węzeł można stworzyć na jego podstawie dokładnie takąś samą ścierzkę,
%jaką mają jego 


%! method-sql pathenum.children

\operacja{Pobranie przodków}
%! method-sql pathenum.ancestors

\operacja{Pobieranie potomków}



%! method-sql pathenum.descendants



\temat{Uwagi}

Dostosowania metody do przyjętego interface zmniejsza jej wydajność.
Praktycznie wszystkie zapytania zawierają podzapytanie które dla danego identyfikatora węzła \todo{term} pobierają jego ścieżkę.
Ta operacja jest w prawdzie bardzo szybka (kolumna \texttt{id} jest kluczem głównym) ale mimo wszystko mają wpływ na ogólną wydajność.
W razie stosowania tej metody warto się zastanowić nad używaniem ścieżki (\texttt{path}) jako parametrów metod. 

Dodatkowo odwoływanie się do węzłów za pomocą ścieżek a nie identyfikatorów daje większe możliwości.
W takiej sytuacji można wykonać dodatkowe operacje bez konieczności wykonywania zapytań w bazie danych.
%Są to:
%\begin{itemize}
%    \item
%        Pobranie głębokości węzła w drzewie (wystarczy policzyć ilość elementów w ścieżce, zero oznacza korzeń\todo{dopisać})
%    \item
%        Wyznaczenie pierwszego wspólnego potomka
%\end{itemize}

\temat{Wydajność}

\begin{qxtab}{pathenum}{Wydajność reprezentacji zmaterializowanych ścieżek}
%! result-table pathenum
\end{qxtab}

\begin{qxfig}{pathenum}{Wydajność reprezentacji zmaterializowanych ścieżek}
%! result-chart pathenum
\end{qxfig}


\index{metoda!drzew prefiksowych|)}
