\section{PostgreSQL \texttt{ltree}}
\index{PostgreSQL}
\index{metoda!ltree@\texttt{ltree}|(textbf}

% Oleg Bartunov, Teodor Sigaev   Oleg Bartunow, Teodor Sigaew
% Rok 2002, wersja 7.2


Moduł \texttt{ltree} zawiera implementacje typu danych oraz funkcji umożliwiających przechowywanie danych hierarchicznych.
Jest on dostępny w bazie PostgreSQL począwszy od wersji 7.2 (wydanej w roku 2002).

Oparty jest na \emph{reprezentacji zmaterializowanych ścieżek}.
Jednak, w odróżnieniu od metody opisanej w jednym z poprzednich rozdziałów tej pracy, oferuje wiele udogodnień.
Chodzi tu przede wszystkim o kilkadziesiąt funkcji i operatorów ułatwiających pracę z tą metodą.
Są one dobrze opisane w oficjalnej dokumentacji \cite{ltree}, 
więc tutaj zostaną przedstawione wyłącznie najważniejsze funkcjonalności.


Jedną z metod korzystania z \texttt{ltree} jest wykorzystanie \texttt{lquery}.
Są to wyrażenia ścieżkowe, przypominające trochę wyrażenia regularne.
Poszczególne elementy zapytania rozdziela się za pomocą kropki.
Chcąc dopasować konkretną etykietę węzła po prostu się ją podaje.
Można też zastosować znak \verb|*|, który pasuje do każdego ciągu węzłów.
Jeśli zastosuje \texttt{\{n\}}, \texttt{\{n,\}}, \texttt{\{n,m\}} lub \texttt{\{,m\}} 
(działające jak w wyrażeniach regularnych)
można określić ile węzłów ma zostać dopasowanych.
Mając zbudowane zapytanie, używa się go następująco: \verb|ltree ~ lquery|.
Przykładowo \verb|t.node ~ '*.5.*'| spowoduje dopasowanie wszystkich węzłów, 
których przodkiem jest węzeł o identyfikatorze \texttt{5}.


Ponadto dostępnych jest wiele funkcji, z których najpotrzebniejsze są:

\begin{description}
%    \item[\texttt{ltree subltree(ltree, start, end)}] \index{subltree@\texttt{subltree}}
    \item[\texttt{ltree subpath(ltree, offset[, len])}]
    	pobiera część \texttt{ltree} zawierającą się pomiędzy elementami o pozycjach pomiędzy 
    	\texttt{offset} a \texttt{len}. 
%    \item[\texttt{int4 nlevel(ltree)}] \index{nlevel@\texttt{nlevel}}
%        zwraca poziom węzła, czyli ilość etykiet w ścieżce
%    \item[\texttt{int4 index(ltree, ltree[, offset])}] \index{index@\texttt{index}}
    \item[\texttt{ltree text2ltree(text)} oraz \texttt{text ltree2text(ltree)}] 
    	pozwalają dokonywać konwersji pomiędzy \texttt{ltree} a łańcuchami znaków
    \item[operator \texttt{ltree <@ ltree}] sprawdza, czy lewa strona jest potomkiem prawej
    \item[operator \texttt{ltree @> ltree}] sprawdza, czy lewa strona jest przodkiem prawej
\end{description}


%\begin{description}
%    \item[\texttt{ltree subltree(ltree, start, end)}] \index{subltree@\texttt{subltree}}
%    \item[\texttt{ltree subpath(ltree, offset[, len])}] \index{subpath@\texttt{subpath}}
%    \item[\texttt{int4 nlevel(ltree)}] \index{nlevel@\texttt{nlevel}}
%        zwraca poziom węzła, czyli ilość etykiet w ścieżce
%    \item[\texttt{int4 index(ltree, ltree[, offset])}] \index{index@\texttt{index}}
%    \item[\texttt{ltree text2ltree(text)}] \index{text2ltree@\texttt{text2ltree}}
%    \item[\texttt{text ltree2text(ltree)}] \index{ltree2text@\texttt{ltree2text}}
%    \item[\texttt{ltree lca(ltree[])}] \index{lca@\texttt{lca}}
%\end{description}


%\temat{Opis modułu \texttt{ltree}}
%
%\podtemat{Definicje}

%\begin{description}
%    \item[etykieta] \eng{label} węzła jest ciągiem jednego lub więcej słów rozdzielonych przez znak \verb|'_'|. 
%        Słowa mogą zawierać litery i liczby.
%        Długość etykiety nie może przekraczać 256 bajtów. 
%        \todo{bajtów? znaków? czy mogą być pl-literki?}
%    \item[ścieżka etykieta]
%        \eng{label path} -- ciąg jednej lub więcej rozdzielonych kropkami etykiet $l_1.l_2...l_n$. 
%        Reprezentuje ścieżkę korzenia do węzła. 
%        Długość ścieżki etykiet jest ograniczony do $2^{16} - 1 = 65535 \approx 64 Kb$. 
%        \todo{Kb? KB?} \todo{2Kb zalecane} 
%
%%   \item[etykieta] \eng{label} 
%\end{description}

%\podtemat{Typy danych}
%
%\begin{description}
%    \item[\texttt{ltree}] -- typ danych
%    \item[\texttt{lquery}] wyrażenie ściekowe
%    \begin{description}
%        \item[\texttt{\{n\}}] asdf
%        \item[\texttt{\{n,\}}] Dopasowuje dokładnie \emph{n} poziomów
%        \item[\texttt{\{n,m\}}] Dopasowuje dokładnie \emph{n} poziomów
%        \item[\texttt{\{,m\}}] Dopasowuje dokładnie \emph{n} poziomów
%    \end{description}
%\end{description}
%
%\podtemat{Funkcje i operatory}
%
%\begin{description}
%    \item[\texttt{ltree subltree(ltree, start, end)}] \index{subltree@\texttt{subltree}}
%    \item[\texttt{ltree subpath(ltree, offset[, len])}] \index{subpath@\texttt{subpath}}
%    \item[\texttt{int4 nlevel(ltree)}] \index{nlevel@\texttt{nlevel}}
%        zwraca poziom węzła, czyli ilość etykiet w ścieżce
%    \item[\texttt{int4 index(ltree, ltree[, offset])}] \index{index@\texttt{index}}
%    \item[\texttt{ltree text2ltree(text)}] \index{text2ltree@\texttt{text2ltree}}
%    \item[\texttt{text ltree2text(ltree)}] \index{ltree2text@\texttt{ltree2text}}
%    \item[\texttt{ltree lca(ltree[])}] \index{lca@\texttt{lca}}
%\end{description}
%
%\begin{description}
%    \item[\texttt{<}, \texttt{>}, \texttt{<=}, \texttt{>=}, \texttt{=}, \texttt{<>}]
%        have their usual meanings. Comparison is doing in the order of direct tree traversing, children of a node are sorted lexicographic.
%    \item[\texttt{ltree @> ltree}]
%        returns TRUE if left argument is an ancestor of right argument (or equal).
%    \item[\texttt{ltree <@ ltree}]
%        returns TRUE if left argument is a descendant of right argument (or equal).
%    \item[\texttt{ltree \~{} lquery}, \texttt{lquery \~{} ltree}]
%        returns TRUE if node represented by ltree satisfies lquery.
%    \item[\texttt{ltree || ltree}, \texttt{ltree || text}, \texttt{text || ltree}]
%        return concatenated ltree.
%\end{description}
%
%
%\podtemat{Indeksy}
%
%\begin{description}
%    \item[B-tree]\index{indeks!Btree@B-tree} na kolumnie \texttt{ltree} pozwala na skorzystanie z operatorów: \verb|<|, \verb|<=|, \verb|=|, \verb|>=|, \verb|>|
%    \item[GiST]\index{indeks!GiST} na kolumnie \texttt{ltree} pozwala na skorzystanie z operatorów: 
%        \verb|<|, \verb|<=|, \verb|=|, \verb|>=|, \verb|>|, \verb|<@|, \verb|@|, \verb|@>|, \verb|~|, \verb|?|.
%        Indeksy GiST \eng{Generalized Search Tree} umożliwiają stosowanie różnych strategii indeksowania w zależności od potrzeb.
%        Za ich pomocą może być zaimplementowane zarówno wyszukiwanie pełnotekstowe (\texttt{tsearch2}), 
%        indeksowanie typu \texttt{hstore} (pozwalające na przechowywanie w jednym atrybucie dowolnej ilości par klucz--wartość).
%        Oczywiście wspiera również typ \texttt{ltree}.
%\end{description}
%
%Dla przykładu można stworzyć następujące indeksy:
%\begin{verbatim}[sql]
%CREATE INDEX test_path_idx_btree ON test USING btree (path);
%CREATE INDEX test_path_idx_gist  ON test USING gost  (path);
%\end{verbatim}



\temat{Operacje}

\operacja{Reprezentacja w SQL}
%! method-sql ltree.create



\operacja{Wstawianie danych}
%! method-sql ltree.insert

\operacja{Pobranie korzeni}
%! method-sql ltree.roots

\operacja{Pobranie rodzica}
%! method-sql ltree.parent

\operacja{Pobranie dzieci}
%! method-sql ltree.children

\operacja{Pobranie przodków}
%! method-sql ltree.ancestors

\operacja{Pobieranie potomków}
%! method-sql ltree.descendants


\temat{Wydajność}

\begin{qxtab}{ltree}{Wydajność metody \texttt{ltree}}
%! result-table ltree
\end{qxtab}

\begin{qxfig}{ltree}{Wydajność metody \texttt{ltree}}
%! result-chart ltree
\end{qxfig}







\index{metoda!ltree@\texttt{ltree}|)}