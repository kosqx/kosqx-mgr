\chapter{Wstęp}
%\chapter*{Wstęp}
%\addcontentsline{toc}{chapter}{Wstęp}

% W ciągu ostatnich kilkunastu lat relacyjne bazy danych opanowały świat. 


%Put simply, this method is used to report, in order, the branches of a family tree. Such trees are
%encountered often—the genealogy of human families, livestock, horses; corporate management,
%company divisions, manufacturing; literature, ideas, evolution, scientific research, theory; and
%even views built upon views.


% hierarchie
% drzewo jako hierarchia
% o jakich drzewach nie mówimy
% cele pracy

% hierarchiczne bazy danych
% - w historii
% - LDAP (ale dane przechowyje w zwyklej bazie, http://en.wikipedia.org/wiki/OpenLDAP - Available Backends)
% - XML
% - rejestr windows

% Parodoksalnie stary, hierarchiczny model zaczyna powracać w bazach NoSQL. Przykładowo MongoDB czy CouchDB.
% Dane są w nich przechowywane jako \todo{sformuowanie} JSON, YAML lub inny rekurencyjny format danych.

%Hierarchie są bardzo użyteczną metodą reprezentacji danych, w~szczególności są bardzo intuicyjne dla ludzi.

Hierarchie są bardzo użyteczną metodą reprezentacji danych.
Zawdzięczają to głównie temu, że są bardzo intuicyjne, naturalne dla ludzi.
Nic dziwnego, że z~tej abstrakcji korzystają mapy myśli, systemy plików, kategorie produktów w sklepach czy katalogi książek w biblotece.
Używane są również do repezentacji struktór organizacyjnych, przykładowo odziałów firm czy wydziałów na uniwersytetach.
Nawet układy prac naukowych są hierarchiczne.


% Dlatego programista baz danych bardzo często ma potrzebę by zapisać drzewo do bazy danych.
% Przykładowo drzewo kategorii produktów w sklepie, katalog książek w bibliotece, stukturę szef--podwładny w firmie, itd.

Strukturą danych najlepiej nadającą się do przechowywania danych hierarchicznych są drzewa.

%Jego zadanie jest utrudnione gdyż --- obecnie dominujący --- relacyjny model danych nie został stworzony z myślą o takich zastosowaniach.
%Oczywiście nie oznacza to, że jest to niemożliwe.
%Wymaga jednak więcej pracy i wiedzy.



Każda współczesna baza danych korzysta z drzew. 
Indeksy są implementowanee jako \emph{B-drzewa}, \emph{R-drzewa}, itd.
% inne drzewa w bazach danych
Zapytania SQL są przetwarzane do \emph{drzew składniowych} \eng{AST --- Abstract Syntax Tree}. Następnie są przetwarzane do postaci planu zapytania który również jest drzewem.
Lecz to są drzewa zastosowane w implementacji, mające na celu zwiększenie szybkości działania bazy danych i nie są bezpośrednio widoczne dla użytkownika bazy danych.
Z tego powodu nie będą tu omawiane.

Warto wspomnieć, że istnieją modele danych wspierające przechowywanie danych hierarchicznych.
Oczywistym przykładem jest \emph{hierarchiczny model danych}.
\todo{Paul Beynon-Davies}
Korzysta on z dwóch struktur danych: rekordów oraz związków nadrzędny\dywiz{}podrzędny.
Najważniejszym jego reprezentantem jest \emph{IMS} \eng{Information Management System} firmy IBM.
Innymi bazami danych przechowywójącymi dane hierarchiczne są LDAP \eng{Lightweight Directory Access Protocol} oraz rejestr systemu operacyjnego Windows.

Alternatywnym podejciem do przechowywania danych hierarchicznych jest umieszczenie ich w wnętrzu pojedyńczego rekordu.
Można do tego zastosować różne formaty danych.
Obecnie najpopularniejszym używanym w tym celu jest XML.
Korzystają z niego zarówno \emph{natywne bazy danych XML} \eng{NXD --- Native XML databases} jak i relacyjne bazy danych z stosownymi rozszerzeniami.
Innym formatem jest JSON stosowany w obecnie zdobywających popularność bazach \emph{NoSQL}, takich jak \emph{CouchDB} czy \emph{MongoDB}. 
% formatem używanym do tego jest XML.




%Takie podejście jest całkiem popularne. Jako przykład naj


%




%Dlatego nie będą omawiane w tej pracy.

%Lecz --- najpopularniejsze obecnie --- relacyjne bazy danych nie zostały stworzone z myślą o obsłudze danych hierarchicznych.
%Nie oznacza to bynajmniej, że jest to niemożliwe.
%Należy za to spodziewać się pewnych utrudnień.
%Największych problemow należy się spodziewać w



% Celem tej pracy jest zaprezentowanie metod przechowywania danych w relacyjnych bazach danych. 


%{Sprawę dodatkowo komplikuje dążenie do przechowywani danych w 3 postaci normalnej. Niektóre metody nie są nawet w 1 postaci normalnej. \ask{Prawda? Pisać o tym?}


Dominujące obecnie relacyjne bazy danych nie zostały stworzone z myślą o obsłudze danych hierarchicznych.
Nie oznacza to bynajmniej, że jest to niemożliwe.
Już Edgar Frank Codd\index{Codd Edgar Frank} zaproponował jedno z rozwiązań.
\todo{sprawdzić gdzie} 
Miało być ono obroną relacyjnego modelu danych przed zarzutami,
że w takich bazach nie można przechowywać danych hierarchicznych.



\section{Cel pracy}





%Nie oznacza to bynajmniej, że jest to niemożliwe.
%Należy za to spodziewać się pewnych utrudnień.
%Największych problemow należy się spodziewać w

Celem tej pracy jest:
\begin{itemize*}
    \item przedstawienie popularnych lub godnych zainteresowania metod przechowywania danych hierarchicznych
    \item porównanie tych metod pod względem łatwości użycia, dostępności, możliwości
    \item sprawdzenie ich wydajności w najpopularniejszych bazach danych
    \item ocena metod, wskazanie ich mocnych i słabych stron
    \item dostarczenie programu umożliwiającego własnoręczne sprawdzenie wymienionych metod
        oraz ułatwiającego sprawdzenie własnych
\end{itemize*}


Zagadnienie przechowywania danych hierarchicznych w relacyjnych bazach danych nie jest tematem nowym.
Temat ten jest poruszany często w Internecie,
wiele książek na temat baz danych zawiera rozdziały na ten temat.
Joe Celko nawet napisał całą książkę poruszającą to zagadnienie\cite{celko-tree}.


Więc dlaczego powstała ta praca?
Otóż dostępne materiały często zawierają opisy tylko kilku wybranych metod.

Pozatym nie zawierają testów wydajności, lub jeśli zawierają to tylko w jednej bazie.



% relatywnie populanne - ale brakuje naprawdę kompleksowych porównań

%Nie należy się spodziewać idealnej metody, raczej należy oczekiwać, że metody będą się znacznie różniły między sobą w poszczególnych operacjach.
%najlepszed do danego zastosowania


