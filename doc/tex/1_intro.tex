\chapter*{Wstęp}
\addcontentsline{toc}{chapter}{Wstęp}

% W ciągu ostatnich kilkunastu lat relacyjne bazy danych opanowały świat. 



Hierarchie są bardzo użyteczną metodą reprezentacji danych. \todo{przepisać na polski}. 
Nic dziwnego, że z tej idei\todo{lepsze słowo: abstrakcji, pojęcia} korzystają \emph{mapy myśli}, drzewa katalogów, itd.

Niestety najpopularniejsze obecnie relacyjne bazy danych nie zostały stworzone do obsługi danych hierarchicznych.
Nie oznacza to bynajmniej, że jest to niemożliwe.
Należy za to spodziewać się pewnych utrudnień.

Celem tej pracy jest zaprezentowanie metod przechowywania danych w relacyjnych bazach danych. 

Jest to zagadnienie stare niemal jak same relacyjne bazy danych. Już Edgar Frank Codd\index{Codd Edgar Frank} zaproponował jedno z rozwiązań.
\todo{sprawdzić gdzie} 
Miało być ono obroną relacyjnego modelu danych przed zarzutami, że w takich bazach nie można przechowywać danych hierarchicznych.



Celem tej pracy jest \todo{wypisać}

% porównanie wydajności, mocnych stron, słabych stron
% nie należy się spodziewać idealnej metody, raczej najlepszed do danego zastosowania




