\chapter{Wstęp}
%\addcontentsline{toc}{chapter}{Wstęp}

% W ciągu ostatnich kilkunastu lat relacyjne bazy danych opanowały świat. 


%Put simply, this method is used to report, in order, the branches of a family tree. Such trees are
%encountered often—the genealogy of human families, livestock, horses; corporate management,
%company divisions, manufacturing; literature, ideas, evolution, scientific research, theory; and
%even views built upon views.


% hierarchie
% drzewo jako hierarchia
% o jakich drzewach nie mówimy
% cele pracy
% 


Hierarchie są bardzo użyteczną metodą reprezentacji danych, w~szczególności są bardzo intuicyjne dla ludzi.
Nic dziwnego, że z~tej abstrakcji korzystają \emph{mapy myśli}, systemy plików, kategorie produktów w sklepach czy katalogi książek w biblotece.
Używane są również do repezentacji struktór organizacyjnych, przykładowo odziałów firm czy wydziałów na uniwersytetach.
Nawet układy prac dyplomowych są hierarchiczne.


% Dlatego programista baz danych bardzo często ma potrzebę by zapisać drzewo do bazy danych.
% Przykładowo drzewo kategorii produktów w sklepie, katalog książek w bibliotece, stukturę szef--podwładny w firmie, itd.

Strukturą danych najlepiej nadającą się do przechowywania danych hierarchicznych są drzewa.

%Jego zadanie jest utrudnione gdyż --- obecnie dominujący --- relacyjny model danych nie został stworzony z myślą o takich zastosowaniach.
%Oczywiście nie oznacza to, że jest to niemożliwe.
%Wymaga jednak więcej pracy i wiedzy.



Każda współczesna baza danych korzysta z drzew. 
Indeksy są implementowanee jako \emph{B-drzewa}, \emph{R-drzewa}, itd.
% inne drzewa w bazach danych
Zapytania SQL są przetwarzane do \emph{drzew składniowych} \eng{AST --- Abstract Syntax Tree}. Następnie są przetwarzane do postaci planu zapytania który również jest drzewem.
Lecz to są drzewa zastosowane w implementacji, mające na celu zwiększenie szybkości działania bazy danych i nie są bezpośrednio widoczne dla użytkownika bazy danych.
Z tego powodu nie będą tu omawiane.

%Dlatego nie będą omawiane w tej pracy.

%Lecz --- najpopularniejsze obecnie relacyjne bazy danych --- nie zostały stworzone do obsługi danych hierarchicznych.
%Nie oznacza to bynajmniej, że jest to niemożliwe.
%Należy za to spodziewać się pewnych utrudnień.
%Największych problemow należy się spodziewać w



% Celem tej pracy jest zaprezentowanie metod przechowywania danych w relacyjnych bazach danych. 


%{Sprawę dodatkowo komplikuje dążenie do przechowywani danych w 3 postaci normalnej. Niektóre metody nie są nawet w 1 postaci normalnej. \ask{Prawda? Pisać o tym?}


Celem tej pracy jest:
\begin{itemize}
    \item przedstawienie popularnych metod przechowywania danych hierarchicznych
    \item porównanie tych metod pod względem łatwości użycia, dostępności, możliwości
    \item sprawdzenie ich wydajności w najpopularniejszych bazach danych
    \item ocena metod, wskazanie ich mocnych i słabych stron
\end{itemize}

Nie należy się spodziewać idealnej metody, raczej należy oczekiwać, że metody będą się znacznie różniły między sobą w poszczególnych operacjach.
%najlepszed do danego zastosowania


Jest to zagadnienie stare niemal jak same relacyjne bazy danych. 
Już Edgar Frank Codd\index{Codd Edgar Frank} zaproponował jedno z rozwiązań.
\todo{sprawdzić gdzie} 
Miało być ono obroną relacyjnego modelu danych przed zarzutami, 
że w takich bazach nie można przechowywać danych hierarchicznych.



