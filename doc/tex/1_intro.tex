\chapter*{Wstęp}
\addcontentsline{toc}{chapter}{Wstęp}

% W ciągu ostatnich kilkunastu lat relacyjne bazy danych opanowały świat. 


%Put simply, this method is used to report, in order, the branches of a family tree. Such trees are
%encountered often—the genealogy of human families, livestock, horses; corporate management,
%company divisions, manufacturing; literature, ideas, evolution, scientific research, theory; and
%even views built upon views.



Hierarchie --- a w szczególności drzewa --- są bardzo użyteczną metodą reprezentacji danych. \todo{przepisać na polski}. 
Nic dziwnego, że z tej abstrakcji korzystają \emph{mapy myśli}, drzewa katalogów, itd.






Każda współczesna baza danych korzysta z struktur danych jakimi są drzewa. 
Indeksy są implementowanee jako \emph{B-drzewa}, \emph{R-drzewa}, itd.
% inne drzewa w bazach danych
Zapytania SQL są przetwarzane do \emph{drzew składni} \eng{AST --- Abstract Syntax Tree}.
Lecz to są drzewa zastosowane w implementacji, mające na celu zwiększenie szybkości działania bazy danych i nie są widoczne dla użytkownika bazy danych.

Lecz --- najpopularniejsze obecnie relacyjne bazy danych --- nie zostały stworzone do obsługi danych hierarchicznych.
Nie oznacza to bynajmniej, że jest to niemożliwe.
Należy za to spodziewać się pewnych utrudnień.

% Celem tej pracy jest zaprezentowanie metod przechowywania danych w relacyjnych bazach danych. 





Celem tej pracy jest:
\begin{itemize}
    \item przedstawienie popularnych metod przechowywania danych hierarchicznych
    \item porównanie tych metod pod względem łatwości użycia, dostępności, możliwości
    \item sprawdzenie ich wydajności w najpopularniejszych bazach danych
    \item ocena metod, wskazanie ich mocnych i słabych stron
\end{itemize}

Nie należy się spodziewać idealnej metody, raczej należy oczekiwać, że metody będą się znacznie różniły między sobą w poszczególnych operacjach.
%najlepszed do danego zastosowania


Jest to zagadnienie stare niemal jak same relacyjne bazy danych. 
Już Edgar Frank Codd\index{Codd Edgar Frank} zaproponował jedno z rozwiązań.
\todo{sprawdzić gdzie} 
Miało być ono obroną relacyjnego modelu danych przed zarzutami, 
że w takich bazach nie można przechowywać danych hierarchicznych.



