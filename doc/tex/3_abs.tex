\chapter{Konstrukcje języka}
%     Opisane w tym rozdziale metody nie są uniwersalnymi reprezentacjami.

% \todo{Przepisać, bo stylistycznie to masakra}
% 
% Metody PL/SQL, CTEs, connect by korzystają z reprezentacji krawędziowej.
% Metoda krawędziowa posiada wiele zalet lecz ma też znaczącą wadę - 
% SQL (sciślej piszącL te jego dialetaki jaki są dostępny w wszystkich bazach danych)
% nie pozwala na pobranie potomków w jednym zapytaniu. 
% Podobnnie jest z przodkami. 
% Natomiast niektre bazy danych posiadają mechanizmy umożliwjące na wygodne wykonywanie zapytań rekurencyjnych.
% 
% 
% Pozostałe metody też korzystają z omówionych poprzednio reprezentacji.
% Oferują za to mechanizmy ułatwiające operowanie tymi reprezentacjami, 
% jak też optymalizacje zwiększające wydajność i zmniejszające zrozmiar pola.


Opisana w poprzednim rozdziale metoda krawędziowa ma szereg zalet.
Niestety ma też poważną wadę --- aby pobrać przodków lub potomków należy wykonać wiele zapytań.
W praktyce oznacza to większą czasochłonność takiej operacji. 
Ponadto wymaga bardziej skomplikowanego kodu.

Obejściem tego problemu jest pełniejsze wykorzystanie możliwości dawanych przez bazy danych.
Nie są one zwykłymi składami danych, lecz udostępniają coraz większe możliwości operowania na nich.

Wystarczy więc przepisać te dwie powolne metody tak by wykonywane były jako pojedyncze zapytanie w bazie danych.
Pozostałe operacje w tych metodach korzystają z implementacji \emph{metody krawędziowej}.

%Siłą rzeczy przedstawione tu metody będą dziediczyły z implementacji metody krawędziowej.
%Jedyne przedstawione operacje to  pobrania przodków i potomków.
%Pozostałem tetody są identyczne z oryginalną  metodą.