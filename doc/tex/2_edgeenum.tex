\section{Metoda zmaterializowanych krawędzi}
\index{metoda!zmaterializowanych krawędzi|(textbf}

Metoda ta polega na przechowywaniu ciągu numerów porządkowych wszystkich węzłów pomiędzy korzeniem a danym węzłem.

% Podobieństwo do drzew katalogów gdzie 'ala' nic nie znaczy, za to znaczenie ma cała ścieżka
% tzn
% Podobne jest też do wyboru ŧrasy w mieście gdzie mówimy najpierw w lewo, potem w prawo, prosto i wprawo i jesteśmu
% Metoda ta ma szansę być o wiele bardziej zwięzła, 

\temat{Reprezentacja}

\begin{verbatim}[table] edgeenum
>n1 _
id | path     | name
1  | /1       | Bazy Danych

>n2 n1
id | path     | name
2  | /1/1     | Obiektowe

>n3 n2
id | path     | name
3  | /1/1/1   | db4o

>n4 n1
id | path     | name
4  | /1/2     | Relacyjne

>n5 n4
id | path     | name
5  | /1/2/1   | Open Source

>n7 n5
id | path     | name
6  | /1/2/1/1 | PostgreSQL

>n8 n5
id | path     | name
7  | /1/2/1/2 | MySQL

>n9 n5
id | path     | name
8  | /1/2/1/3 | SQLite

>n6 n4
id | path     | name
9  | /1/2/2   | Komercyjne

>n10 n1
id | path     | name
10 | /1/3     | XML

\end{verbatim}


\temat{Operacje}

\operacja{Tworzenie tabeli}
%! method-sql edgeenum.create

\operacja{Wstawianie danych}
%! method-sql edgeenum.insert

\operacja{Pobranie korzeni}
%! method-sql edgeenum.roots

\operacja{Pobranie rodzica}
%! method-sql edgeenum.parent

\operacja{Pobranie dzieci}
%! method-sql edgeenum.children

\operacja{Pobranie przodków}
%! method-sql edgeenum.ancestors

\operacja{Pobieranie potomków}
%! method-sql edgeenum.descendants


\temat{Wydajność}

\begin{qxtab}{edgeenum}{Wydajność reprezentacji zmaterializowanych krawędzi}
%! result-table edgeenum deep3
\end{qxtab}

\begin{qxfig}{edgeenum}{Wydajność reprezentacji zmaterializowanych krawędzi}
%! result-chart edgeenum deep3
\end{qxfig}



\index{metoda!zmaterializowanych krawędzi|)}