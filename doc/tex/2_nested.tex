\section{Metoda zagnieżdzonych zbiorów}
\index{metoda!zagnieżdzonych zbiorów|(textbf}

Reprezentacja \emph{zagnieżdżonych zbiorów} \eng{nested sets} została spopularyzowana przez Joe Celko\cite{celko-sql}\index{Celko Joe}.
Dlatego czasem bywa nazywana ,,metodą Celko''.
Jak sam przyznaje, oparł się na opisie Donalda Knutha\cite{knuth}.

%Lecz to nie on jako pierwszy ją opisał.
%%Przed nim zrobił to Donald Knuth\cite{knuth}.

%Lecz to nie on jako pierwszy ją opisał.
%Przed nim zrobił to Donald Knuth\cite{knuth}.



%Nazwa tej metody pochodzi od 
%Drzewo można 
%
%W tej reprezentacji

%W tej metodzie każdy węzeł drzewa traktowany jest jako zbiór zbiorem.
%Jeśli nie ma podzbiorów to jest liściem.
%Natomiast jeśli ma podzbiory to staje się ich przodkiem.
%Jeśli dodatkowo jest najmniejszym z nadzbiorów danego węzła to jest jego rodzicem.
%Dodatkowym wymogiem jest to by każdy węzeł miał część wspólną tylko i~wyłącznie z swymi przodkami oraz potomkami (a~nie z~rodzeństwem).
%W efekcie każdy zbiór jest \emph{,,zagnieżdzony''} w swoim rodzicu%
%\footnote{
%    Za przykład z realnego świata może posłużyć metoda pakowania przedmiotów w celu przewozu.
%    W kontenerach zanajdują się palety, na których znajdują się kartony, wewnątrz których znajdują się opakowania indywidualne produktów.
%    Innym przykładem są matrioszki
%}.
%Stąd pochodzi nazwa tej reprezentacji.

W tej metodzie każdy węzeł drzewa traktowany jest jako zbiór.
Muszą one spełniać wymóg by \emph{każde dwa zbiory były rozłączne albo jeden był podzbiorem drugiego}.
W efekcie jeśli zbiór nie ma podzbiorów to jest liściem.
Natomiast jeśli ma podzbiory to staje się ich przodkiem.
Jeśli dodatkowo jest najmniejszym z nadzbiorów danego węzła to jest jego rodzicem.
W efekcie każdy zbiór jest \emph{,,zagnieżdżony''} w swoim rodzicu%
\footnote{
    Za przykład z realnego świata może posłużyć metoda pakowania przedmiotów w celu przewozu.
    W kontenerach znajdują się palety, na których znajdują się kartony, wewnątrz których znajdują się opakowania indywidualne produktów.
    Innym przykładem są matrioszki czy jednostki podziału administracyjnego (województwa, gminy, miejscowości, \ldots).
}.
Stąd pochodzi nazwa tej reprezentacji.

W bazie danych najprostszą metodą implementacji tej metody jest myślenie o zbiorach jako o odcinkach na prostej%
\footnote{W tym celu można by było użyć również rozszerzeń przestrzennych, przykładowo \emph{PostGIS} lub \emph{Oracle Spatial}}.
Odcinek taki jest opisywany przez parę liczb.
Jedna z nich określa lewy a druga prawy koniec odcinka.
Jeśli dodatkowo przyjmiemy, że zbiór tych wartości liczbowych końców odcinków
jest zbiorem kolejnych liczb naturalnych to zyskujemy możliwość łatwego wykonywania wielu operacji.

Ponieważ \texttt{left} oraz \texttt{right} są słowami kluczowymi SQL-92,
w~kodzie zostaną zastosowane ich skróty, odpowiednio \texttt{lft} oraz \texttt{rgt}.

Reprezentacja pozwala na przechowywanie drzew uporządkowanych.

%\temat{Reprezentacja}

\begin{verbatim}[table] nested
>n1 _
id | lft | rgt | name
1  | 1   | 20  | Bazy Danych

>n2 n1
id | lft | rgt | name
2  | 2   | 5   | Obiektowe

>n3 n2
id | lft | rgt | name
3  | 3   | 4   | db4o

>n4 n1
id | lft | rgt | name
4  | 6   | 17  | Relacyjne

>n5 n4
id | lft | rgt | name
5  | 7   | 14  | Open Source

>n7 n5
id | lft | rgt | name
6  | 8   | 9   | PostgreSQL

>n8 n5
id | lft | rgt | name
7  | 10  | 11  | MySQL

>n9 n5
id | lft | rgt | name
8  | 12  | 13  | SQLite

>n6 n4
id | lft | rgt | name
9  | 15  | 16  | Komercyjne

>n10 n1
id | lft | rgt | name
10 | 18  | 19  | XML

\end{verbatim}

\temat{Operacje}

\operacja{Reprezentacja w SQL}
%! method-sql nested.create

\operacja{Wstawianie danych}
%! method-python nested.insert



% \begin{verbatim}[sql]
% INSERT INTO simple (parent, name) VALUES (:parent, :name)
% \end{verbatim}

Wstawianie danych w tej metodzie jest jej najsłabszą stroną.
\todo{terminologia}
Wymaga ono zmiany wartości \texttt{lft} i \texttt{rgt} wielu rekordów.

W powyższym przypadku
--- jako że wstawiamy dane w kolejności \emph{in depth} ---
wymagana jest modyfikacja tylko tylu rekordów na jakim poziomie\todo{?} aktualnie wstawiamy nowy węzeł.
Lecz należy się liczyć z przypadkiem pesymistycznym --- dodaniem do korzenia potomka jako pierwszego elementu\todo{las}.
W takiej sytuacji wymagane jest zmodyfikowanie wszystkich istniejących rekordów.

%\textbf{Uwaga:} Jeśli chce się jednorazowo załadować cały las do bazy danych można postąpić bardziej optymalnie\footnote{
%    Powyższy kod tego nie robi gdyż ma być ogólny.
%}.

\textbf{Uwaga:} Jeśli chce się jednorazowo załadować całe drzewo do bazy danych można zrobić to bardziej wydajnie.
(powyższy kod tego nie robi gdyż ma być ogólny).

Mianowicie przed załadowaniem danych do bazy danych wstępnie przetworzyć dane.
W takiej sytuacji należy sie posłużyć algorytmem \emph{wyszukiwania w głąb}
i~kolejno numerować wartości \texttt{lft} węzła po wejściu do niego oraz \texttt{rgt} przed jego opuszczeniem.
Przykładowa implementacja tego algorytmu:

\begin{verbatim}[python]
def preprocess(node):
  node.lft = get_next_int()
  for n in children(node):
    preprocess(n)
  node.rgt = get_next_int()
\end{verbatim}

%
%\operacja{Pobranie węzłów --- cechy wspólne}
%
%Wszystkie implementacje operacje są do siebie bardzo podobne.

% JOIN box
% wewnątz/na zewnątez

\operacja{Pobranie korzeni}

W tej reprezentacji korzeń to węzeł, który nie zawiera się w żadnym innym.
Zapytanie próbuje pobrać rodzica danego rekordu, a dzięki złączeniu zewnętrznemu 

%! method-sql nested.roots

\operacja{Pobranie rodzica}
%! method-sql nested.parent

Ten zapytanie różni się od pobrania przodków w dwóch miejscach:
\begin{itemize}
    \item 
	\texttt{result.lft + 1} w warunku złączenia powoduje, że spełniają go wyłącznie przodkowie właściwi
    \item 
        klauzula \texttt{LIMIT 1} sprawia, że pobiera tylko jeden rekord 
\end{itemize}

\operacja{Pobranie dzieci}

Pobieranie dzieci to de facto pobieranie potomków z dodatkowym warunkiem.
Jest nim sprawdzenie czy 

%Podzapytanie skorelowane

%! method-sql nested.children



\operacja{Pobranie przodków}

Przodkowie to wszystkie węzły zawierające (w całości) dany węzeł.
Oznacza to, że wartość z węzła będącego parametrem znajduje się pomiędzy ich lewym i prawym końcem. 

%! method-sql nested.ancestors


Aby uzyskać wymaganą kolejność można by zastosować sortowanie \texttt{ORDER BY (result.rgt - result.lft) ASC}.
Jednak można zauważyć, że lewy koniec przodka ma mniejszą wartośc liczbową niż u potomka.
Prowadzi to do prostrzego \texttt{ORDER BY result.lft DESC}.

\operacja{Pobieranie potomków}

Pobieranie potomków jest tą operacją, dla której stosuje się tą reprezentację.
Metoda pozwala na bardzo proste i szybkie wykonywanie tej operacji.

%! method-sql nested.descendants

Dzięki sortowaniu \texttt{ORDER BY result.lft ASC} otrzymuje się potomków w kolejności wyszukiwania w głąb.

% Jeśli by box.lft + 0 to nie tylko potomkowie właściwi.

\operacja{Uwagi}

Metoda umożliwia bardzo proste pobieranie liści drzewa.
Jest ono możliwe dzięki temu, że $node.rgt - node.lft = 1$\footnote{
    Warto zauważyć, że warunek podany jako \texttt{node.rgt - node.lft = 1} mimo,
    że jest matematycznie równoważny z \texttt{node.lft + 1 = node.rgt} może być mniej wydajny.
    Wynika to z tego, że drugim przypadku optymalizator może wykorzystać indeks.
}
tylko i wyłącznie dla liści. 
%Jest ono możliwe dzięki temu, że $a.right - a.left = 2 * |a.descendants()| + 1$. 
%Skoro liść nie posiada żadnych potomków to da różnica wynosi $1$, co prowadzi do zapytania: 
\begin{verbatim}[sql]
SELECT *
  FROM nested_sets
  WHERE lft + 1 = rgt
\end{verbatim}




\temat{Wydajność}

\begin{qxtab}{nested}{Wydajność reprezentacji zagnieżdzonych zbiorów}
%! result-table nested deep3
\end{qxtab}

\begin{qxfig}{nested}{Wydajność reprezentacji zagnieżdzonych zbiorów}
%! result-chart nested deep3
\end{qxfig}

\index{metoda!zagnieżdzonych zbiorów|)}
