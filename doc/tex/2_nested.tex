\section{Metoda zagnieżdzonych zbiorów}
\index{metoda!zagnieżdzonych zbiorów|textbf}

Metoda została spopularyzowana przez Joe Celko\cite{celko-sql}\index{Celko Joe}.



\operacja{Reprezentacja w SQL}
%! method-sql nested.create

\operacja{Wstawianie danych}
% %! method-sql nested.insert

% \begin{verbatim}[sql]
% INSERT INTO simple (parent, name) VALUES (:parent, :name)
% \end{verbatim}

Wstawianie danych w tej metodzie jest jej najsłabszą stroną.


\operacja{Pobranie korzeni}
%! method-sql nested.roots

\operacja{Pobranie rodzica}
% %! method-sql nested.parent

\operacja{Pobranie dzieci}
% %! method-sql nested.children

\operacja{Pobranie przodków}
%! method-sql nested.ancestors

\operacja{Pobieranie potomków}
%! method-sql nested.descendants


\subsection{Uwagi}

Metoda umożliwia bardzo proste pobieranie liści drzewa. 
Jest ono możliwe dzięki temu, że $a.right - a.left = 2 * |a.descendants()| + 1$. 
Skoro liść nie posiada żadnych potomków to da różnica wynosi $1$, co prowadzi do zapytania: 
\begin{verbatim}[sql]
SELECT *
    FROM nested_sets
    WHERE left + 1 = right
\end{verbatim}


\begin{table}[h!]
  \caption{Wyniki reprezentacji zagnieżdzonych zbiorów}
   \begin{center}
%! result-table nested deep3
   \end{center}
\end{table}

\begin{figure}[h!t]
  \caption{Wyniki reprezentacji zagnieżdzonych zbiorów}
  \label{fig:img_chart_nested}
  \begin{center}
%! result-chart nested deep3
  \end{center}
\end{figure}

