\section{PL/SQL}
\index{metoda!PL/SQL|(textbf}

\temat{Opis PL/SQL}



\temat{Operacje}

\operacja{Utworzenie funkcji}

\begin{verbatim}[sql]
CREATE OR REPLACE FUNCTION tree_ancestors(
  start_id int
) RETURNS SETOF int AS $$
DECLARE
  cid int := start_id;
BEGIN
  WHILE cid IS NOT NULL LOOP
    RETURN NEXT cid;
    SELECT parent INTO cid FROM tree WHERE tree.id = cid;
  END LOOP;
END;
$$ LANGUAGE plpgsql strict;
\end{verbatim}

\begin{verbatim}[sql]
CREATE OR REPLACE FUNCTION tree_descendants(
  start_id int
) RETURNS SETOF int AS $$ 
DECLARE
  rec RECORD;
  current INT[];
  build INT[];
  tmp INT;
BEGIN
  build := ARRAY[0];
  current := ARRAY[0, start_id];
  WHILE current > ARRAY[0] LOOP
    build := ARRAY[0];  
    FOR i IN 2..array_upper(current, 1) LOOP
      tmp := current[i];  
      FOR rec IN SELECT * FROM tree WHERE parent = tmp LOOP
        RETURN NEXT rec.id;
        build := build || rec.id;
      END LOOP;
    END LOOP;
    current := build;
  END LOOP;
END;
$$ LANGUAGE plpgsql strict;
\end{verbatim}


\operacja{Pobranie przodków}
%! method-sql plsql.ancestors

\begin{verbatim}[sql]
SELECT * FROM tree_ancestors(111);
SELECT t FROM tree_ancestors(111) AS t;
SELECT count(*) FROM tree_ancestors(111);
SELECT id, parent, value
  FROM tree_ancestors(111) AS t
    JOIN tree ON t = tree.id;
\end{verbatim}


\operacja{Pobieranie potomków}
%! method-sql plsql.descendants

\begin{verbatim}[sql]
SELECT * FROM tree_descendants(1);
SELECT t FROM tree_descendants(1) AS t;
SELECT count(*) FROM tree_descendants(1);
SELECT id, parent, value
  FROM tree_descendants(1) AS t 
    JOIN tree ON t = tree.id;
\end{verbatim}


\temat{Wydajność}

\begin{qxtab}{plsql}{Wydajność metody PL/SQL}
%! result-table plsql deep3
\end{qxtab}

\begin{qxfig}{plsql}{Wydajność metody PL/SQL}
%! result-chart plsql deep3
\end{qxfig}

\operacja{Uwagi}






\index{metoda!PL/SQL|)}

