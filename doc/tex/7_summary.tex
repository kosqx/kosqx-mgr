\chapter{Podsumowanie}

Do tej pory wyniki testów były przedstawiane dla każdej metody oddzielnie, pokazując różnice pomiędzy systemami baz danych.
By móc porównać metody między sobą warto je przedstawić w jednym miejscu.
Przedstawione tu wyniki powstały poprzez uśrednienie wyników wszystkich SZBD umożliwiających implementację danej metody.
Wyniki są podane w ilości zapytań ma milisekundę.


\begin{qxfig}{summary}{Porównanie wydajności metod}
%! summary-chart
\end{qxfig}

%\begin{qxtab}{summary}{Porównanie wydajności metod}
%%! summary-table
%\end{qxtab}


%\section*{Wnioski}
%
%Negatywnym zaskoczeniem są wyniki typu \texttt{hierarchyid}.
%Okazał się być bardzo wolny we wszystkich operacjach.
%Ciekawa metoda pełnych ścieżek również okazała się być zbyt powolna.
%
%Natomiast metoda zmaterializowanych ścieżek oraz typ \texttt{ltree} wypadły bardzo dobrze.
%
%Warto zauważyć, że metoda krawędziowa, pomimo konieczności wykonywania wielu zapytań, całkiem dobrze radzi sobie z pobieraniem przodków i potomków.



\section*{Którą metodę wybrać?}



