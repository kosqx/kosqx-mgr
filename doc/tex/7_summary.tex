\chapter{Podsumowanie}

Do tej pory wyniki testów były prezentowane dla każdej metody oddzielnie, pokazując różnice pomiędzy systemami baz danych.
By móc porównać metody między sobą warto je ująć w jednym miejscu.
Przedstawione tu wartości powstały poprzez uśrednienie wyników wszystkich SZBD umożliwiających implementację danej metody.
%Są podane w postaci ilości zapytań ma milisekundę.


\begin{qxfig}{summary}{Porównanie wydajności metod}
%! summary-chart
\end{qxfig}

%\begin{qxtab}{summary}{Porównanie wydajności metod}
%%! summary-table
%\end{qxtab}


\section*{Którą metodę wybrać?}

W poprzednich rozdziałach zostały przedstawione popularne oraz warte zainteresowania metody przechowywania danych hierarchicznych.
Zaprezentowano ich implementacje oraz wyniki wydajności.

%Daje to podstawę do odpowiedzi na powyższe pytanie.

Jak można się domyślić, nie istnieje metoda najlepsza w wszystkich sytuacjach%
\footnote{Dowód nie wprost: jeśli by istniała perfekcyjna metoda to tylko ona była by stosowana}.
Czyli programista musi przewidzieć
\footnote{Jeszcze lepiej było by sprawdzić założenia, przeprowadzając symulację}
jakie operacje będą najczęściej wykorzystywane.


Poniżej zostaną podane rady ułatwiające wybór metody dla konkretnego, praktycznego zastosowania.

\begin{description}
	\item[Częste wstawianie węzłów]
		Jeśli struktura hierarchiczna jest często modyfikowana to należy wybrać metodę oferującą szybką operację wstawiania.
		Zdecydowanie najlepsza jest w tym metoda krawędziowa. 
		Ewentualnie można by użyć reprezentacji zmaterializowanych ścieżek.
		Przy okazji warto zauważyć, że oparty na niej typ \texttt{ltree} oferuje równie dobrą wydajność. 
		
	\item[Pobieranie przodków i potomków]
		Są to operacje w których reprezentacja krawędziowa radzi sobie słabo.
		Z tego wynika popularność metody zagnieżdżonych zbiorów, która jest uważana za najlepszą w tych operacjach.
		Jak pokazały wyniki, niesłusznie.
		Wprawdzie oferuje dobrą wydajność, ale metoda zmaterializowanych ścieżek jest w tej sytuacji jeszcze lepsza.
\end{description}

Jeśli nie jest możliwe przewidzenie jak będą używane dane hierarchiczne to najlepiej zastosować reprezentację krawędziową.
Przemawia za tym:
\begin{itemize}
  	\item
    	Jest najprostsza w implementacji.
  	\item 
  		Metoda ma dobrą wydajność. W kilku operacjach jest najszybsza, nigdzie nie jest bardzo powolna.
  	\item
  		W razie potrzeby można (bez zmian w strukturze tabel, wystarczy zastosować inne zapytania) przekształcić ją w metodę Wspólnych Wyrażeń Tabelowych lub \texttt{CONNECT BY}.
  	\item 
  		Opiera się na bardzo prostej koncepcji, więc przeniesienie danych z niej do dowolnej z omówionych w tej pracy metod powinno być proste.
\end{itemize}




