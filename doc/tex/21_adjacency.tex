\section{Metoda listy sąsiedztwa}


% Hierarchies like trees, organizational charts, ... are sometimes difficult to
% store in database tables. The most common database pattern to store 
% hierachical data is known as the adjacency model. It has been introduced 
% by the famous computer scientist Edgar Frank Codd. It's called the "adjacency"
% model because a reference to the parent data is stored in the same row as the
% child data, in an adjacent column. These kind of tables are also called self 
% referencing tables.
% http://www.scip.be/index.php?Page=ArticlesNET18


% It was Scott's pet cat called "tiger".
% And, who was Scott? His first name was Scott but Bruce. Bruce Scott was
% employee number #4 at the then Software Development Laboratories that
% eventually became Oracle. He co-authored and co-architected Oracle V1, V2 & V3.
% http://www.dba-oracle.com/t_scott_tiger.htm

% http://pl.wikipedia.org/wiki/Reprezentacja_grafu#Reprezentacja_przez_listy_s.C4.85siedztwa



Najprostszą, najbardziej intuicyjną i zapewne najpopularniejszą metodą jest metoda przyległej listy.
Została ona spopularyzowana przez Edgara Franka Codda. 
On też nadał im nazwę, która odnosi się do tego, że informacja o rodzicu elementu znajduje się w tej samej krotce co dane.

Zasługa spopularyzowania tej metody przypada Oracle. 
Dołączył on do swojego produktu przykładową bazę danych, nazywaną ,,Scott/Tiger'' 
korzystającą z tej metody.
\footnote{
    Nazwa tej bazy danych pochodzi od metody autoryzacji w bazie Oracle (login/hasło).
    Login pochodził z nazwiska jednego z pierwszych pracowników 
    Software Development Laboratories (przekształconych ostatecznie w Oracle) Bruce'a Scott'a. 
    Natomiast hasło to imię jego kota.
}

Na popularność metody przekłada się również jej znaczące podobieństwo do 
używanego między innymi w~językach C i~C++ sposobu przechowywania list i drzew.
Mianowicie każdy węzeł zawiera wskaźnik na rodzica. 
W relacyjnych bazach danych odpowiednikiem wskaźnika jest klucz obcy. 

\operacja{Definicja danych}

\begin{verbatim}[sql]
CREATE TABLE tree (
  id INTEGER PRIMARY KEY,
  parent INTEGER REFERENCES tree(id) ON DELETE CASCADE,
  value VARCHAR(100)
);
\end{verbatim}

\operacja{Pobieranie potomków}

Należy zwrócić uwagę na wykorzystanie operatora \emph{IN}. Bez niego wydajność metody spada znacząco. 

\operacja{Wstawianie danych}

\operacja{Wyniki}


\begin{table}[h!]
  \caption{Wyniki reprezentacji krawędziowej}
  \begin{center}
%! result-table simple deep3
  \end{center}
\end{table}

\begin{figure}[h!t]
  \caption{Wyniki reprezentacji krawędziowej}
  \label{fig:img_chart_simple}
  \begin{center}
%! result-chart simple deep3
  \end{center}
\end{figure}

\clearpage



