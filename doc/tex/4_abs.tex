\chapter{Typy danych}

%\chapter{Metody specyficzne dla bazy danych}

%    Opisane w tym rozdziale metody nie są uniwersalnymi reprezentacjami.
%
%\todo{Przepisać, bo stylistycznie to masakra}
%
%Metody PL/SQL, CTEs, connect by korzystają z reprezentacji krawędziowej.
%Metoda krawędziowa posiada wiele zalet lecz ma też znaczącą wadę - 
%SQL (sciślej piszącL te jego dialetaki jaki są dostępny w wszystkich bazach danych)
%nie pozwala na pobranie potomków w jednym zapytaniu. 
%Podobnnie jest z przodkami. 
%Natomiast niektre bazy danych posiadają mechanizmy umożliwjące na wygodne wykonywanie zapytań rekurencyjnych.
%
%
%Pozostałe metody też korzystają z omówionych poprzednio reprezentacji.
%Oferują za to mechanizmy ułatwiające operowanie tymi reprezentacjami, 
%jak też optymalizacje zwiększające wydajność i zmniejszające zrozmiar pola.


Standardowo bazy danych umożliwiają przechowywanie wielu typów danych.
Do powszechnie występujących można zaliczyć typy\footnote{Wymienione nazwy konkretnych typów pochodzą z bazy PostgreSQL}:
\begin{itemize}
    \item numeryczne
        (\verb|integer|, \verb|numeric|, \verb|real|, \ldots)
    \item znakowe
        (\verb|char|, \verb|varchar|, \verb|text|, \ldots)
    \item daty i czasu
        (\verb|date|, \verb|time|, \verb|timestamp|, \verb|interval|, \ldots)
    \item binarne
        (\verb|bytea|)
%    \item numeryczne
%        (\verb||, \verb||, \verb||, \verb||, \verb||, \verb||, )
%    \item numeryczne
%        (\verb||, \verb||, \verb||, \verb||, \verb||, \verb||, )
\end{itemize}

Ponadto bazy danych oferują własne, specyficzne dla implementacji typy.
Wśród nich warto wymienić typy tablicowe, sieciowe, logiczne, monetarne, wyliczeniowe oraz złożone.
Coraz częściej stosowane są też typy umożliwiające przechowywanie dokumentów XML.


Co najważniejsze --- z punktu widzenia tej pracy --- istnieją typy ułatwiające przechowywanie danych hierarchicznych.


 