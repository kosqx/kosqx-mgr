\chapter{Typy danych}

%\chapter{Metody specyficzne dla bazy danych}

%    Opisane w tym rozdziale metody nie są uniwersalnymi reprezentacjami.
%
%\todo{Przepisać, bo stylistycznie to masakra}
%
%Metody PL/SQL, CTEs, connect by korzystają z reprezentacji krawędziowej.
%Metoda krawędziowa posiada wiele zalet lecz ma też znaczącą wadę - 
%SQL (sciślej piszącL te jego dialetaki jaki są dostępny w wszystkich bazach danych)
%nie pozwala na pobranie potomków w jednym zapytaniu. 
%Podobnnie jest z przodkami. 
%Natomiast niektre bazy danych posiadają mechanizmy umożliwjące na wygodne wykonywanie zapytań rekurencyjnych.
%
%
%Pozostałe metody też korzystają z omówionych poprzednio reprezentacji.
%Oferują za to mechanizmy ułatwiające operowanie tymi reprezentacjami, 
%jak też optymalizacje zwiększające wydajność i zmniejszające zrozmiar pola.