\section{PostgreSQL \texttt{ltree}}
\index{PostgreSQL}
\index{metoda!ltree@\texttt{ltree}|(textbf}


\begin{description}
    \item[etykieta] \eng{label} węzła 
    \item[ścieżka etykieta]
        \eng{label path} -- ciąg jednej lub więcej rozdzielonych kropkami etykiet $l_1.l_2...l_n$. 
        Reprezentuje ścieżkę korzenia do węzła. 
        Długość ścieżki etykiet jest ograniczony do $2^{16} - 1 = 65535 \approx 64 Kb$. 
        \todo{Kb? KB?} \todo{2Kb zalecane} 

%   \item[etykieta] \eng{label} 
\end{description}


\begin{description}
    \item[\texttt{ltree}] -- typ danych
    \item[\texttt{ltree[]}]
    \item[\texttt{lquery}] wyrażenie ściekowe
    \begin{description}
        \item[\texttt{\{n\}}] asdf
        \item[\texttt{\{n,\}}] Dopasowuje dokładnie \emph{n} poziomów
        \item[\texttt{\{n,m\}}] Dopasowuje dokładnie \emph{n} poziomów
        \item[\texttt{\{,m\}}] Dopasowuje dokładnie \emph{n} poziomów
    \end{description}
\end{description}

\subsection*{Indeksy}

\begin{description}
    \item[B-tree] na kolumnie \texttt{ltree} pozwala na skorzystanie z operatorów: \verb|<|, \verb|<=|, \verb|=|, \verb|>=|, \verb|>|
    \item[GiST] na kolumnie \texttt{ltree} pozwala na skorzystanie z operatorów: 
        \verb|<|, \verb|<=|, \verb|=|, \verb|>=|, \verb|>|, \verb|<@|, \verb|@|, \verb|@>|, \verb|~|, \verb|?|.
        Indeksy GiST \eng{Generalized Search Tree} umożliwiają stosowanie różnych strategii indeksowania w zależności od potrzeb.
        Za ich pomocą może być zaimplementowane zarówno wyszukiwanie pełnotekstowe (\texttt{tsearch2}), 
        indeksowanie typu \texttt{hstore} (pozwalające na przechowywanie w jednym polu dowolnej ilości par klucz--wartość).
        Oczywiście wspiera również typ \texttt{ltree}.
\end{description}

Dla przykładu można stworzyć następujące indeksy:
\begin{verbatim}[sql]
CREATE INDEX test_path_idx_btree ON test USING btree (path);
CREATE INDEX test_path_idx_gist  ON test USING gost  (path);
\end{verbatim}



\subsection*{Operacje}

\operacja{Reprezentacja w SQL}
%! method-sql ltree.create

\operacja{Wstawianie danych}
%! method-sql ltree.insert

\operacja{Pobranie korzeni}
%! method-sql ltree.roots

\operacja{Pobranie rodzica}
%! method-sql ltree.parent

\operacja{Pobranie dzieci}
%! method-sql ltree.children

\operacja{Pobranie przodków}
%! method-sql ltree.ancestors

\operacja{Pobieranie potomków}
%! method-sql ltree.descendants



\operacja{Wyniki}

\begin{table}[h!]
  \caption{Wyniki \texttt{ltree}}
  \begin{center}
%! result-table ltree deep3
  \end{center}
\end{table}

\begin{figure}[h!t]
  \caption{Wyniki \texttt{ltree}}
  \label{fig:img_chart_simple}
  \begin{center}
%! result-chart ltree deep3
  \end{center}
\end{figure}






\index{metoda!ltree@\texttt{ltree}|)}