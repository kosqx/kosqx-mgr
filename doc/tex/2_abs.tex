\chapter{Podstawowe reprezentacje}

W tym rozdziale zostaną przedstawione podstawowe reprezentacje drzew.
Reprezentacje --- czyli sposoby przedstawienia struktury drzewa.
Wiele reprezentacji zostało wymienione w \cite{knuth}.
Tu zostaną opisane te najbardziej przydatne w bazach danych.
% Różnią się one stopniem duplikacji danych.
% nie są ortogonalne

% TODO:
% Reprezentacje --- równowarzne formy przedstawienia danych
%   Oparte na podobnych definicjach
% Programowanie dynamiczne - w pathenum i fulltree są już wyliczone dane
%   strata na ortogonalnościd

%Tylko pierwsza z wymienionych metod jest w 3 postaci normalnej.
%Pozostałe wykorzystują uprzednio wyliczone dane by przyśpieszyć wykonywane na nich operacje.
%W ten sposób upodobniają się do programowania dynamicznego.


Tym co różni wymienione tu metody to sposób w jaki przetwarzają dane przed zapisaniem ich do bazy danych.
Tylko pierwsza z wymienionych metod wstawia dane w najprostszej postaci.
Pozostałe stosują (zwykle dosyć kosztowne) wstępne przetwarzanie, 
mające na celu przyśpieszenie późniejszych zapytań.
%Jako, że wstawianie danych jest zwykle o wiele rzadziej wykonywane niż ich pobierania to takie zachowanie jest opłacalne 
%gdyż zmniejsza koszt średni operacji.
Wstawianie danych jest zwykle o wiele rzadziej wykonywane niż ich pobieranie.
W efekcie takie zachowanie jest opłacalne gdyż zmniejsza koszt średni operacji.


Metody wymienione w tym rozdziale są uniwersalne.
Daje się je zaimplementować w każdej relacyjnej bazie danych. 
Można również (przy odrobinie wysiłku) korzystać z nich również w \emph{odwzorowaniach obiektowo-relacyjnych}\index{OR/M}.
Nic nie stoi na przeszkodzie aby wykorzystać te metody również w innych niż relacyjne bazach danych.

% Przedstawione zostaną tu różne reprezentacje lasów.
% Czym czym są reprezentacje?
% Otórz drzewo można zozumieć na wiele różnych sposobów.
% jako graf, jako strukturę rekurecyjną