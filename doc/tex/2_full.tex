\section{Metoda pełnych ścieżek}
\index{metoda!pełnych ścieżek|(textbf}
% http://troels.arvin.dk/db/rdbms/links/#hierarchical
% 
% http://en.wikipedia.org/wiki/Transitive_closure
% http://pl.wikipedia.org/wiki/Domknięcie_przechodnie

% Domknięcie_przechodnie Cormen:644

% Reprezentacja ta opiera się na

% Ta metoda

Ta reprezentacja jest najmniej rozpowszechniona spośród zaprezentowanych w tym rozdziale.
Przy czym w Polsce jest bardziej popularna niż poza granicami kraju.
Jest to zasługa \emph{Huberta Lubaczewskiego}\index{Lubaczewski Hubert} lepiej znanego pod pseudonimem \emph{depesz}\index{depesz|see{Lubaczewski Hubert}}.
Stąd w polskim internecie często można spotkać się z tą metodą pod nazwą \emph{metoda depesza}.
On sam nazywa tę reprezentację \emph{metodą pełnych ścieżek}.
% na świecie można się spotkać z nazwami odwołującymi się do Domknięcie_przechodnie \emg{Transitive closure}

%Metoda został spopularyzowana w Polskim Internecie przez \emph{Huberta Lubaczewskiego}\index{Lubaczewski Hubert} lepiej znanego pod pseudonimem \emph{depesz}\index{depesz|see{Lubaczewski Hubert}}.

\todo{napisać czym moja implementacja różni się od implementacji Depesza}

\index{drzewo!domknięcie przechodnie}

Idea metody jest prosta. 
Głowna tabela z danymi nie zawiera żadnych informacji o hierarchii danych. 
Jedynym wymogiem jest istnienie w niej klucza głównego\index{klucz!główny}.

Cała informacja potrzebna do operowania na drzewie zawiera się w dodatkowej tabeli. 
Zawiera ona informacje o odległości pomiędzy każdym elementem a wszystkimi jego przodkami.


Przedstawiona tutaj implementacja została lekko zmieniona względem oryginalnej ,,metody depesza''.
Mianowicie zawiera dodatkowo jeden specjalny rekord dla każdego węzła.
Przechowuje on odległość pomiędzy węzłem a korzeniem.
Posiada on atrybut \verb|top_id| ustawiony na \verb|NULL|.


Ilość wierszy drugiej tabeli wynosi:
\begin{displaymath}
    \sum_{t \in T} level(t) + 2 \leq |T|(height(T) + 2)
\end{displaymath}
Jak widać, metoda wymaga znacząca ilości rekordów (które trzeba wstawić i przechowywać).
Jednak nie jest to duży problem gdyż pojedyńczy rekord zawiera wyłącznie trzy liczby całkowite, więc rozmiar rekordu jest mały.

Poniższy graf pokazuje wyłącznie zawartość dodatkowej tabeli zawierającej strukturę drzewa.

%\temat{Reprezentacja}

\begin{verbatim}[table] full

>n1 _
top_id | bottom_id | distance
NULL   | 1         | 0
1      | 1         | 0

>n2 n1
top_id | bottom_id | distance
NULL   | 2         | 1
1      | 2         | 1
2      | 2         | 0

>n3 n2
top_id | bottom_id | distance
NULL   | 3         | 2
1      | 3         | 2
2      | 3         | 1
3      | 3         | 0

>n4 n1
top_id | bottom_id | distance
NULL   | 4         | 1
1      | 4         | 1
4      | 4         | 0

>n5 n4
top_id | bottom_id | distance
NULL   | 5         | 2
1      | 5         | 2
4      | 5         | 1
5      | 5         | 0


>n6 n5
top_id | bottom_id | distance
NULL   | 6         | 3
1      | 6         | 3
4      | 6         | 2
5      | 6         | 1
6      | 6         | 0

>n7 n5
top_id | bottom_id | distance
NULL   | 7         | 3
1      | 7         | 3
4      | 7         | 2
5      | 7         | 1
7      | 7         | 0

>n8 n5
top_id | bottom_id | distance
NULL   | 8         | 3
1      | 8         | 3
4      | 8         | 2
5      | 8         | 1
8      | 8         | 0

>n9 n4
top_id | bottom_id | distance
NULL   | 9         | 2
1      | 9         | 2
4      | 9         | 1
9      | 9         | 0

>n10 n1
top_id | bottom_id | distance
NULL   | 10        | 1
1      | 10        | 1
10     | 10        | 0

\end{verbatim}


\temat{Operacje}

\operacja{Reprezentacja w SQL}
%! method-sql full.create

\operacja{Wstawianie danych}

%Mimo, że w tej metodzie trzeba wstawić sporo węzłów

W bazach dających możliwość wstawiania danych wygenerowanych przez podzapytanie ta operacja jest bardzo łatwa do wykonania.

%! method-sql full.insert


Jak widać podzapytanie składa się z 3 części, których wyniki są łączone za pomocą operatora \texttt{UNION ALL}.
Pierwszy fragment przetwarza rekordy odpowiadające za położenie rodzica w drzewie.
Drugi odpowiada za sytuację gdy węzeł jest korzeniem i trzeba wstawić specjalny rekord z \texttt{top\_id = NULL}.
Natomiast trzeci odpowiada za rekord z \texttt{distance = 0}.

\operacja{Pobranie węzłów}

W tej metodzie wszystkie analizowane operacje są do siebie bardzo podobne.
By nie omawiać ich wielokrotnie zostanie tu zaprezentowany ogólny mechanizm.

\begin{verbatim}[sql]
SELECT d.*
    FROM full_data d 
        JOIN full_tree t 
            ON (d.id = $JOIN-ATTR$)
    WHERE
        $WHERE-ATTR$ = :id AND
        t.distance $DISTANCE$
\end{verbatim}

Jak widać w powyższym kodze pojawiły się ,,bloki'' (zapisywane \verb|$NAZWA-BLOKU$|).
Należy je zastąpić odpowiednimi fragmentami SQL.

Bloki \verb|$JOIN-ATTR$| i \verb|$WHERE-ATTR$| są od siebie zależne.
Gdy jeden z nich jest postaci \verb|t.top_id| drugi musi być \verb|t.bottom_id|.
Jeśli \verb|$JOIN-ATTR$| ma wartość \verb|t.bottom_id| to zapytanie będzie pobierało potomków (węzły ,,pod'').
Analogicznie przodków w przeciwnej sytuacji.

Natomiast blok \verb|$DISTANCE$| pozwala sprecyzować z których poziomów zostaną pobrane węzły.
I tak dla \verb|>= 0| zostaną pobrani przodkowie/potomkowie%
\footnote{W tej sytuacji można by było pominąć ten warunek, gdyż wszystkie wartości atrybutu \texttt{distance} są większe lub równe zero},
dla \verb|> 0| zostaną pobrani przodkowie/potomkowie właściwi.
Natomiast \verb|= 1| spowoduje pobranie rodzica/potomków.
Oczywiście można zastosować dowolny warunek lub ich kombinację, co stanowi siłę tej reprezentacji. 


\operacja{Pobranie korzeni}
%! method-sql full.roots


\operacja{Pobranie rodzica}
%! method-sql full.parent

%Tabela \verb|full_tree| zawiera rekordy z 

\operacja{Pobranie dzieci}
%! method-sql full.children

\operacja{Pobranie przodków}
%! method-sql full.ancestors

\operacja{Pobieranie potomków}
%! method-sql full.descendants

Jak widać w klauzuli \texttt{SELECT} pojawiło się podzapytanie
zwracające identyfikator rodzica bieżącego węzła.
Jeśli operacja ma pobierać tylko zbiór potomków to można je pominąć.
Natomiast jeśli wynik zapytania ma zostać przekształcony w drzewo to jest ono konieczne.
Wynika to z faktu, że sama tabela \verb|full_tree| nie zawiera informacji o strukturze hierarchicznej.

%\operacja{Uwagi}
%
%Ta metoda jest bardzo elastyczna, pozwala łatwo wykonywać bardzo bardzo różnorodne zapytania.
%
%
%\begin{verbatim}[sql]
%SELECT *
%  FROM none
%\end{verbatim}


\temat{Wydajność}

\begin{qxtab}{full}{Wydajność reprezentacji pełnych ścieżek}
%! result-table full
\end{qxtab}

\begin{qxfig}{full}{Wydajność reprezentacji pełnych ścieżek}
%! result-chart full
\end{qxfig}





\index{metoda!pełnych ścieżek|)}


