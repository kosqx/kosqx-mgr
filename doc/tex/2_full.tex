\section{Metoda pełnych ścieżek}
\index{metoda!pełnych ścieżek|textbf}
% http://troels.arvin.dk/db/rdbms/links/#hierarchical
% 
% http://en.wikipedia.org/wiki/Transitive_closure
% http://pl.wikipedia.org/wiki/Domknięcie_przechodnie

% Domknięcie_przechodnie Cormen:644

Metoda został spopularyzowana w Polskim Internecie przez \emph{Huberta Lubaczewskiego}\index{Lubaczewski Hubert} lepiej znanego pod pseudonimem \emph{depesz}\index{depesz|see{Lubaczewski Hubert}}.

\index{drzewo!domknięcie przechodnie}

Idea metody jest prosta. 
Głowna tabela z danymi nie zawiera żadnych informacji o hierarchi danych. 
Jedynym wymogiem jest istnienie w niej klucza głównego\index{klucz!główny}.

Cała informacja potrzebna do operowania na drzewie zawiera się w dodatkowej tabeli. 
Zawiera ona informacje o odległości pomiędzy każdym elementem a wszystkimi jego potomkami.
 

\operacja{Reprezentacja w SQL}
%! method-sql full.create

\operacja{Wstawianie danych}
% %! method-sql full.insert

% \begin{verbatim}[sql]
% INSERT INTO simple (parent, name) VALUES (:parent, :name)
% \end{verbatim}


\operacja{Pobranie korzeni}
%! method-sql full.roots

\operacja{Pobranie rodzica}
%! method-sql full.parent

\operacja{Pobranie dzieci}
%! method-sql full.children

\operacja{Pobranie przodków}
%! method-sql full.ancestors

\operacja{Pobieranie potomków}
%! method-sql full.descendants


\begin{table}[h!]
  \caption{Wyniki reprezentacji pełnych ścieżek}
   \begin{center}
%! result-table full deep3
   \end{center}
\end{table}

\begin{figure}[h!t]
  \caption{Wyniki reprezentacji pełnych ścieżek}
  \label{fig:img_chart_nested}
  \begin{center}
%! result-chart full deep3
  \end{center}
\end{figure}
