\section{Microsoft SQL Server \texttt{hierarchyid}}
	% http://technet.microsoft.com/en-us/library/bb677173.aspx
	% http://blogs.msdn.com/manisblog/archive/2007/08/17/sql-server-2008-hierarchyid.aspx
	% http://www.microsoft.com/poland/technet/bazawiedzy/centrumrozwiazan/cr314_01.mspx


% Apparently the HierarchyID uses the ORDPATH algorithm (as far as I'm concerned). 
% Just found the following document that elaborates somewhat on the algorithm and other things related to hierarchical storage:
% http://sites.computer.org/debull/a07mar/kumaran.pdf


% SQL Server 2008 adds a new feature to help with modeling hierarchical relationships: the HIERARCHYID data type. It provides compact storage and convenient methods to manipulate hierarchies. In a way it is very much like optimized materialized path. In addition the SqlHierarchyId CLR data type is available for client applications. 
% 
% While HIERARCHYID has a lot to offer in terms of operations with hierarchical data, it is important to understand a few basic concepts:
% 
% - HIERARCHYID can have only a single root (although easy to work around by adding sub-roots)
% - It does not automatically represent a tree, the application has to define the relationships and enforce all rules 
% - The application needs to maintain the consistency
% http://pratchev.blogspot.com/2008/05/hierarchies-in-sql-server-2008.html

% http://technet.microsoft.com/en-us/library/cc721270.aspx  !! TODO
% The new HIERARCHYID data type in SQL Server 2008 is a system-supplied CLR UDT that can be useful for storing and manipulating hierarchies. This type is internally stored as a VARBINARY value that represents the position of the current node in the hierarchy (both in terms of parent-child position and position among siblings). You can perform manipulations on the type by using either Transact-SQL or client APIs to invoke methods exposed by the type. Let’s look at indexing strategies for the HIERARCHYID type, how to use the type to insert new nodes into a hierarchy, and how to query hierarchies.

%% TODO: 
% - opis algorytmu (zoptymalizowane ścieżki zmaterializowane)
% - opis wszystkich funkcji
% - http://www.cs.umb.edu/~poneil/ordpath.pdf google:ORDPATH
% - http://msdn.microsoft.com/en-us/library/bb677173.aspx

Jedną z najciekawszych nowości jakie Microsoft wprowadził w SQL Server 2008 jest nowy typ danych \texttt{hierarchyid}. 
Pozwala on na wygodne przechowywanie danych hierarchicznych.

W budowie wewnętrznej przypomina reprezentację wyliczonych krawędzi, lecz jest zapisany w


\subsection*{Opis typu \texttt{hierarchyid}}

\paragraph{Budowa}

Typ \texttt{hierarchyid} jest przechowywany wewnętrznie jako \texttt{VARBINARY}. 


\paragraph{Indeksowanie}



\subsection{Urzyteczne funkcje}

\begin{description}
  \item[\texttt{child.GetAncestor(n)}]
    Zwraca węzeł będący  $n-tym$ przodkiem danego węzła.
	%This method is useful to find the (nth ancestor of the given child node.

  \item[\texttt{parent.GetDescendant(child1, child2)}]
	This method is very useful to get the descendant of a given node. 
	It has a great significance in terms of finding the new descendant position get the descendants etc. 

	This function returns one child node that is a descendant of the parent. 
	If parent is NULL, returns NULL. 
	If parent is not NULL, and both child1 and child2 are NULL, returns a child of parent. 
	If parent and child1 are not NULL, and child2 is NULL, returns a child of parent greater than child1. 
	If parent and child2 are not NULL and child1 is NULL, returns a child of parent less than child2. 
	If parent, child1, and child2 are all not NULL, returns a child of parent greater than child1 and less than child2. 
	If child1 or child2 is not NULL but is not a child of parent, an exception is raised. 
	If child1 >= child2, an exception is raised.

  \item[\texttt{node.GetLevel()}]
	Zwraca liczbę całowitą będącą poziomem danego węzła w drzewie\todo{słownik}. 
	Korzeń ma poziom równy $0$. 
	Jako, że typ \texttt{hierarchyid} nie obsługuje lasów to aby to kompensować tworzy się sztuczny korzeń, a korzenie obsługiwanych drzew znajdują się na poziomie $1$.

	% This function will return an integer that represents the depth of this node in the current tree. 

  \item[\texttt{hierarchyid::GetRoot()}]
	This method will return the root of the hierarchy tree and this is a static method if you are using it within CLR. 
	It will return the data type hierarchyID. 

  \item[\texttt{parent.IsDescendant(child)}]
	This method returns true/false (BIT) if the node is a descendant of the parent. 

  \item[\texttt{hierarchyid::Parse (input)}]
	Parse converts the canonical string representation of a hierarchyid to a hierarchyid value. 
	Parse is called implicitly when a conversion from a string type to hierarchyid occurs. 
	Acts as the opposite of ToString(). 
	Parse() is a static method. 

  \item[\texttt{void Read( BinaryReader r )}]
	Read reads binary representation of SqlHierarchyId from the passed-in BinaryReader and sets the SqlHierarchyId object to that value. 
	Read cannot be called by using Transact-SQL. Use CAST or CONVERT instead.

  \item[\texttt{node.Reparent(oldRoot, newRoot)}]
	This is a very useful method which helps you to reparent a node i.e. suppose if we want to align an existing node 
	to a new parent or any other existing parent then this method is very useful. 

  \item[\texttt{node.ToString()}]
	This method is useful to get the string representation of the HierarchyID. 
	The method returns a string that is a nvarchar(4000) data type.


  \item[\texttt{void Write( BinaryWriter w )}]
	Write writes out a binary representation of SqlHierarchyId to the passed-in BinaryWriter. 
	Write cannot be called by using Transact-SQL. Use CAST or CONVERT instead.

 \end{description}


\operacja{Reprezentacja}

Typ \texttt{hierarchyid} jest przeznaczony do przechowywania pojedyńczego drzewa drzewa.
Aby umożliwić mu przechowywanie lasu należy połączyć korzenie wszystkich drzew z dodatkowym węzłem.
Stanie się on korzeniem przechowywanego drzewa.

\begin{verbatim}[table] hierarchyid

>n0 _
id | node      | name
1  | /         | ROOT

>n1 n0
id | node      | name
2  | /1/       | Bazy Danych

>n2 n1
id | node      | name
3  | /1/1/     | Obiektowe

>n3 n2
id | node      | name
4  | /1/1/1/   | db4o

>n4 n1
id | node      | name
5  | /1/2/     | Relacyjne

>n5 n4
id | node      | name
6  | /1/2/1/   | Open Source

>n7 n5
id | node      | name
7  | /1/2/1/1/ | PostgreSQL

>n8 n5
id | node      | name
8  | /1/2/1/2/ | MySQL

>n9 n5
id | node      | name
9  | /1/2/1/3/ | SQLite

>n6 n4
id | node      | name
10  | /1/2/2/  | Komercyjne

>n10 n1
id | node      | name
11 | /1/3/     | XML

\end{verbatim}

\subsection*{Operacje}

\operacja{Reprezentacja w SQL}
%! method-sql hierarchyid.create

\operacja{Wstawianie danych}
%! method-sql hierarchyid.insert

\operacja{Pobranie korzeni}
%! method-sql hierarchyid.roots

\operacja{Pobranie rodzica}
%! method-sql hierarchyid.parent

\operacja{Pobranie dzieci}
%! method-sql hierarchyid.children

\operacja{Pobranie przodków}
%! method-sql hierarchyid.ancestors

\operacja{Pobieranie potomków}
%! method-sql hierarchyid.descendants



\operacja{Wyniki}

\begin{table}[h!]
  \caption{Wyniki \texttt{hierarchyid}}
  \begin{center}
%! result-table hierarchyid deep3
  \end{center}
\end{table}

\begin{figure}[h!t]
  \caption{Wyniki \texttt{hierarchyid}}
  \label{fig:img_chart_simple}
  \begin{center}
%! result-chart hierarchyid deep3
  \end{center}
\end{figure}


\operacja{Uwagi}

SQL Server udostępnia \texttt{PERSISTED} --- umożliwiające dynamiczne tworzenie dynamicznych kolumn. \todo{lepiej to sformuować}
Przykładowo może zostać to użyte do łatwego pobrania 

\begin{verbatim}[sql]
CREATE TABLE tree (
  node hierarchyid PRIMARY KEY CLUSTERED,
  level AS node.GetLevel() PERSISTED,
  name varchar(50)
);
SELECT level, name FROM tree;
\end{verbatim}


% P main.py sql sqlserver  'SELECT *, master.dbo.fn_varbintohexstr(cast(node as varbinary)) a, node.ToString() t FROM herid'
% +----+------+-------------+----------+-----------+
% | id | node | name        | a        | t         |
% +----+------+-------------+----------+-----------+
% | 1  |      | ROOT        | None     | /         |
% | 2  | X    | Bazy Danych | 0x58     | /1/       |
% | 3  | Z�   | Obiektowe   | 0x5ac0   | /1/1/     |
% | 4  | Z�   | db4o        | 0x5ad6   | /1/1/1/   |
% | 5  | [@   | Relacyjne   | 0x5b40   | /1/2/     |
% | 6  | [V   | Komercyjne  | 0x5b56   | /1/2/1/   |
% | 7  | [Z   | Open Source | 0x5b5a   | /1/2/2/   |
% | 8  | [Z�  | PostgreSQL  | 0x5b5ab0 | /1/2/2/1/ |
% | 9  | [Z�  | MySQL       | 0x5b5ad0 | /1/2/2/2/ |
% | 10 | [Z�  | SQLite      | 0x5b5af0 | /1/2/2/3/ |
% | 11 | [�   | XML         | 0x5bc0   | /1/3/     |
% +----+------+-------------+----------+-----------+

