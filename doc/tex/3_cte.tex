\section{Wspólne Wyrażenia Tabelowe}
\index{metoda!Wspolnych@Wspólnych Wyrażeń Tabelowych|(textbf}
\index{IBM DB2}\index{SQL Server}
% http://www.ibm.com/developerworks/data/library/techarticle/0307steinbach/0307steinbach.html


% http://publib.boulder.ibm.com/infocenter/dzichelp/v2r2/index.jsp?topic=/com.ibm.db29.doc.apsg/db2z_createcte.htm
% http://www.4guysfromrolla.com/webtech/071906-1.shtml - dobry opis, napisane do czego może się jeszcze przydać
% http://www.depesz.com/index.php/2008/10/07/waiting-for-84-common-table-expressions-with-queries/#more-1287

% Opis standardowej metody with która pojawiła się w standardzie SQL:1999. Porównanie z connect by
% UNION ALL (should allow UNION too), and we don't have SEARCH or CYCLE clauses
% Common Table Expressions - Wspólne wyrażenie tabelowe
% http://www.google.com/search?client=opera&rls=en&q=Wsp%C3%B3lne+wyra%C5%BCenie+tabelowe&sourceid=opera&ie=utf-8&oe=utf-8

%% TODO:
% - napisać http://en.wikipedia.org/wiki/Common_table_expressions
% - przerobić tytuły i opisy na CTE

% rekurencyjne wyrażenie tabelowe. Wspólne wyrażenie tabelowe odwołujące się do siebie w klauzuli FROM pełnej selekcji.
% Rekurencyjne wyrażenia tabelowe są używane do konstruowania zapytań rekurencyjnych.

% Common table expressions, or CTEs, are new to DB2 as of Version 8 and they greatly expand the useability of SQL. 
% A common table expression can be thought of as a named temporary table within a SQL statement that is retained for the duration of a SQL statement. 
% There can be many CTEs in a single SQL statement, but each must have a unique name and be defined only once.


% PGSQL doc:
% Tip: The recursive query evaluation algorithm produces its output in breadth-first search order.
% You can display the results in depth-first search order by making the outer query ORDER BY a
% “path” column constructed in this way.



W standardzie SQL:1999\index{SQL!SQL:1999} dodano \emph{Wspólne Wyrażenia Tabelowe} \eng{CTE --- Common Table Expressions}.
Głównym celem ich wprowadzenia było umożliwienie pisania bardziej zwięzłego, czytelnego a przede wszystkim prostszego kodu.

Ich działanie przypomina stworzenie tymczasowego widoku\index{widok}. 
%istniejącego tylko na czas wykonania danego zapytania.
Po zdefiniowaniu jest on dostępny dla zapytania.
Jednak w odróżnieniu od widoku jest on tworzony w samym zapytaniu i istnieje tylko na czas jego wykonywania.


Przykładowym zastosowaniem CTE jest uniknięcie wielokrotnego używania tego samego podzapytania\cite{apress-sqlserver}.
Nie tylko zwiększy to czytelność kodu, ale poprawi jego wydajność.
% http://www.postgresonline.com/journal/archives/127-PostgresQL-8.4-Common-Table-Expressions-CTE,-performance-improvement,-precalculated-functions-revisited.html



%Pozwala ono również na rekurencyjne wykonywanie zapytań w bazie danych.

Wspólne Wyrażenia Tabelowe mają też inne, ciekawsze z punktu widzenia tej pracy, zastosowanie.
Pozwalają one na rekurencyjne wykonywanie zapytań w bazie danych.

W chwili obecnej to rozszerzenie jest dostępne w:
\begin{itemize}
 \item \textbf{IBM DB2} począwszy od  wersji 8
 \item \textbf{Microsoft SQL Server}
	% http://www.mssqltips.com/tip.asp?tip=1520
	począwszy od SQL Server 2005
 \item \textbf{PostgreSQL}
	% http://developer.postgresql.org/pgdocs/postgres/queries-with.html
	począwszy od 8.4
% \item \textbf{Firebird}
%	wersja 2.1 
\end{itemize}

%\begin{description}
% \item[IBM DB2]
%    wprowadzone w wersji 8
% \item[Microsoft SQL Server]
%	% http://www.mssqltips.com/tip.asp?tip=1520
%	w wersjach począwszy od SQL Server 2005
% \item[PostgreSQL]
%	% http://developer.postgresql.org/pgdocs/postgres/queries-with.html
%	ma się pojawić w wersji 8.4
% \item[Firebird]
%	wersja 2.1 
%\end{description}


\temat{Opis Wspólnych Wyrażeń Tabelowych}
\index{Wspolne@Wspólne Wyrażenia Tabelowe|textbf}
\index{CTEs|see{Wspólne Wyrażenia Tabelowe}}
\index{Common Table Expressions|see{Wspólne Wyrażenia Tabelowe}}
\index{WITH@\texttt{WITH}|see{Wspólne Wyrażenia Tabelowe}}


Składnia Rekurencyjnych Wspólnych Wyrażeń Tabelowych prezentuje się następująco:


\index{WITH RECURSIVE@\texttt{WITH RECURSIVE}|see{Wspólne Wyrażenia Tabelowe}}


\begin{verbatim}[sql]
WITH <<nazwa widoku>>(<<nazwy kolumn>>) AS
(
  <<zapytanie startowe>>
UNION ALL
  <<zapytanie rekurencyjne>>
)
SELECT <<nazwy kolumn>> FROM <<nazwa widoku>>
\end{verbatim}

Działanie tej konstrukcji można streścić jako:
\begin{enumerate}
    \item
        Na początku wykonywane jest \texttt{zapytanie startowe}.
        Jego wynik jest widoczny pod nazwą \texttt{nazwa widoku}.
    \item
        Korzystając z wygenerowanych w poprzednim kroku rekordów
        (wyłącznie w ostatnim kroku, wyniki poprzednich wywołań nie są tu dostępne)
        \texttt{zapytanie rekurencyjne} zwraca nowe wyniki.
        %Jeśli nie został pobrany żaden nowy element
        Jeśli jest ich więcej niż zero to ten etap się powtarza.
    \item
        Po zakończeniu pracy części rekurencyjnej jej \emph{wszystkie} wyniki są dostępne dla zapytania znajdującego się po nim.
\end{enumerate}

Ten algorytm można przedstawić również w poniższy sposób:
\begin{verbatim}[python]
widok = []
tmp = zapytanie_startowe()
while tmp:
  widok.extend(tmp)
  tmp = zapytanie_rekurencyjne(tmp)
return widok
\end{verbatim}

Najprostszy przykład wykorzystania zapytań rekurencyjnych to wygenerowanie ciągu liczb. 
W tym przykładzie ciągu arytmetycznego zaczynającego się od $0$ (co wynika z zapytania startowego \texttt{SELECT \textbf{0} AS i}), 
różnica ciągu wynosi $1$ (\texttt{SELECT \textbf{i + 1} AS i})

\begin{verbatim}[sql]
WITH RECURSIVE a(i) AS (
    SELECT 0 AS i
  UNION ALL
    SELECT i + 1 AS i
      FROM a
      WHERE i < 10
)
SELECT i FROM a;
\end{verbatim}


%Rozszerzenie to można wykorzystać do pracy z reprezentacją krawędziową. 
%Umożliwi ono wysłanie do bazy danych tylko jednego zapytania podczas pobierania potomków i przodków. 
%Bez niego konieczne było wykonywanie odzielnego zapytania podczas przechodzenia przez każdy poziom drzewa.

\temat{Operacje}

%W bazie PostgreSQL należy urzyć \texttt{}

Niestety implementacje CTE różnią się od siebie.
PostgreSQL wymaga jawnego podania, że chodzi o zapytanie rekurencyjne czyli \texttt{WITH RECURSIVE}.
Pozostałe bazy wymagają samego \texttt{WITH} a słowo \texttt{RECURSIVE} powoduje zgłoszenie błędu.
Ponadto DB2 nie zezwala na jawne użycie złączenia (czyli z \texttt{JOIN}).
Czyli trzeba przenieść warunek złączenia do klauzuli \texttt{WHERE}.


\operacja{Pobranie przodków}
%! method-sql with.ancestors

\operacja{Pobieranie potomków}
%! method-sql with.descendants

By zapytanie zwracało potomków właściwych dodano dodatkową kolumnę \texttt{level}.
Jej wartość odpowiada poziomowi węzła w pobieranym poddrzewie.

Pomijając tą kolumnę, kod jest niemal identyczny jak w pobieraniu przodków.
Tym co je różni jest warunek złączenia.

%Warto zwrócić uwagę, że ta konstrukcja przeszukuje drzwewo wszerz. Można można to dostrzec w tym jak zapytanie jest sformuowane. 
%
%Jeśli by zaszła potrzeba pobrania całego drzewa wystarczy w zapytaniu startowym zmienić \texttt{WHERE root.id = :id} na \texttt{WHERE root.parent IS NULL}


\temat{Wydajność}

\begin{qxtab}{with}{Wydajność metody Wspólnych Wyrażeń Tabelowych}
%! result-table with
\end{qxtab}

\begin{qxfig}{with}{Wydajność metody Wspólnych Wyrażeń Tabelowych}
%! result-chart with
\end{qxfig}



\index{metoda!Wspolnych@Wspólnych Wyrażeń Tabelowych|)}





% \begin{verbatim}[sql]
% WITH temptab(deptid, empcount, superdept) AS
%    (    SELECT root.deptid, root.empcount, root.superdept
%             FROM departments root
%             WHERE deptname='Production'
%      UNION ALL
%         SELECT sub.deptid, sub.empcount, sub.superdept
%             FROM departments sub, temptab super
%             WHERE sub.superdept = super.deptid
%    )
% SELECT sum(empcount) FROM temptab
% \end{verbatim}



% P main.py sql db2 'WITH temptab(level, id, parent, name) AS (SELECT 0, r.id, r.parent, r.name FROM simple r WHERE r.id = 1 UNION ALL SELECT t.level + 1, s.id, s.parent, s.name FROM simple s, temptab t WHERE s.parent = t.id) SELECT level, id, parent, name FROM temptab'
% +-------+----+--------+-------------+
% | level | id | parent | name        |
% +-------+----+--------+-------------+
% | 0     | 1  | None   | Bazy Danych |
% | 1     | 2  | 1      | Obiektowe   |
% | 1     | 4  | 1      | Relacyjne   |
% | 1     | 10 | 1      | XML         |
% | 2     | 3  | 2      | db4o        |
% | 2     | 5  | 4      | Komercyjne  |
% | 2     | 6  | 4      | Open Source |
% | 3     | 7  | 6      | PostgreSQL  |
% | 3     | 8  | 6      | MySQL       |
% | 3     | 9  | 6      | SQLite      |
% +-------+----+--------+-------------+



% P main.py sql db2 'WITH temptab(level, id, parent, name) AS (SELECT 0, r.id, r.parent, r.name FROM simple r WHERE r.id = 7 UNION ALL SELECT t.level + 1, s.id, s.parent, s.name FROM simple s, temptab t WHERE s.id = t.parent) SELECT level, id, parent, name FROM temptab'
% +-------+----+--------+-------------+
% | level | id | parent | name        |
% +-------+----+--------+-------------+
% | 0     | 7  | 6      | PostgreSQL  |
% | 1     | 6  | 4      | Open Source |
% | 2     | 4  | 1      | Relacyjne   |
% | 3     | 1  | None   | Bazy Danych |
% +-------+----+--------+-------------+
