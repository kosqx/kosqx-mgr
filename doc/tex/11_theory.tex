\chapter{Wprowadzenie do tematu}

\section{Czym jest drzewo}
\index{drzewo|textbf}
Drzewo to bardzo powszechnie używane w informatyce pojęcie. W zależności od zastosowania może być różnie zdefiniowane.

\paragraph{Drzewo jako graf}
\index{graf}
\paragraph{Drzewo jako struktura rekurencyjna}
\index{rekurencja}


Tu: matematyczny opis drzew, ich podstawowe własności, omówienie terminów:
\begin{itemize}
    \item drzewo
    \item las
    \item rodzic
    \item przodek
    \item korzeń
    \item dziecko
    \item potomek
    \item głębokość, mini
\end{itemize}

\section{Podział drzew}
\subsection{Jednorodność}
Warto by dodać o drzewach jednorodnych oraz niejednorodnych.

\subsection{Kolejność}


\section{Podstawowe algorytmy dla drzew}
\subsection{Wyszukiwanie w głąb}
\index{drzewo!wyszukiwanie!w głąb}
\subsection{Wyszukiwanie wszerz}
\index{drzewo!wyszukiwanie!wszerz}


\section{Różnice pomiędzy drzewami w algorytmice a w bazach danych}

Nie ma sensu coś takiego jak drzewo czerwono-czarne gdyż w drzewach w bazach danych chodzi o strukturę a nie o optymalizację czasu dostępu.

\section{Tematy porównania - operacje}



\subsection{Operacje}
\subsection{Pobranie korzeni}
\index{drzewo!korzeń}
\subsection{Pobranie przodków}
\index{drzewo!przodekowie}
\subsection{Pobranie rodzica}
\index{drzewo!rodzic}
\subsection{Pobranie dzieci}
\index{drzewo!dzieci}
\subsection{Pobranie potomków}
\index{drzewo!potomkowie}
\subsection{Pobranie (najmłodszego) wspólnego przodka}



% \begin{description}
%   \item[pobranie rodzica] polega na pobraniu rodzica bieżącego elementu
%   \item[pobranie przodków] polega na pobraniu rodzica, rodzica rodzica aż do korzenia elementów
% \end{description}

\section{Dostępność}
\subsection{Mapowanie relacyjno-obiektowe}
\subsection{W zależności od bazy danych}


\section{Przyjęte założenia}
\todo{więcej wypisać, zastanowić się co z tym zrobić}

\paragraph{Tabela zawiera tylko jedną kolumnę z danymi użytkownika} Bez problemu można dodać więcej kolumn do tabeli. Obecność wyłącznie kolumny \texttt{name} zwiększa czytelność przykładów.
\paragraph{Zapytania wyłącznie związane z hierarchiczną strukturą danych}