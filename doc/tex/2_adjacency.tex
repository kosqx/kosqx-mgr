\section{Metoda krawędziowa}
\index{metoda!krawędziowa|(textbf}


% Hierarchies like trees, organizational charts, ... are sometimes difficult to
% store in database tables. The most common database pattern to store 
% hierachical data is known as the adjacency model. It has been introduced 
% by the famous computer scientist Edgar Frank Codd. It's called the "adjacency"
% model because a reference to the parent data is stored in the same row as the
% child data, in an adjacent column. These kind of tables are also called self 
% referencing tables.
% http://www.scip.be/index.php?Page=ArticlesNET18


% It was Scott's pet cat called "tiger".
% And, who was Scott? His first name was Scott but Bruce. Bruce Scott was
% employee number #4 at the then Software Development Laboratories that
% eventually became Oracle. He co-authored and co-architected Oracle V1, V2 & V3.
% http://www.dba-oracle.com/t_scott_tiger.htm

% http://pl.wikipedia.org/wiki/Reprezentacja_grafu#Reprezentacja_przez_listy_s.C4.85siedztwa



Najprostszą, najbardziej intuicyjną i zapewne najpopularniejszą metodą jest metoda krawędziowa \eng{adjacency}\index{adjacency}.
Została ona pierwszy raz zaprezentowana przez Edgara Franka Codda\index{Codd Edgar Frank}.
%On też nadał im nazwę, która odnosi się do tego, że informacja o rodzicu elementu znajduje się w tej samej krotce\todo{krotka?} co dane.
%Nadał też jej nazwę. Wynika ona z tego, że każdy rekord zawiera własny identyfikator oraz identyfikator rodzica.
%Takie dane są wystarczające aby w drzewie opisać krawędz.


Zasługa spopularyzowania tej metody przypada Oracle.
Dołączył on do swojego produktu przykładową bazę danych, nazywaną ,,Scott/Tiger'' 
\footnote{
    Nazwa tej bazy danych pochodzi od metody autoryzacji w bazie Oracle (login/hasło).
    Login pochodził z nazwiska jednego z pierwszych pracowników 
    Software Development Laboratories (przekształconych ostatecznie w Oracle)
    % Bruce'a Scott'a.
    Bruca Scotta. 
    Natomiast hasło to imię jego kota.
}
korzystającą z tej metody.

Na popularność metody przekłada się również jej znaczące podobieństwo do 
używanego między innymi w~językach C i~C++ sposobu przechowywania list i drzew.
Mianowicie każdy węzeł zawiera wskaźnik na rodzica.

\begin{verbatim}[c]
typedef struct item {
    struct item* parent;
    char*        name;
} treeitem;
\end{verbatim}

W relacyjnych bazach danych odpowiednikiem wskaźnika jest klucz obcy%
%\index{klucz!obcy}%
. 


Ta metoda jest na tyle popularna, że w bazach danych pojawiły się specjalne konstrukcje do jej obsługi. 
Zostaną one przedstawione w rozdziale \todo{wstawić jakoś nazwę rozdziału} o konstrukcjach języka specyficznych dla systemu zarządzania bazą danych.


%\temat{Reprezentacja}

\begin{verbatim}[table] adjacency
>n1 _
id | parent | name
1  | NULL   | Bazy Danych

>n2 n1
id | parent | name
2  | 1      | Obiektowe

>n3 n2
id | parent | name
3  | 2      | db4o

>n4 n1
id | parent | name
4  | 1      | Relacyjne

>n5 n4
id | parent | name
5  | 4      | Open Source

>n7 n5
id | parent | name
6  | 5      | PostgreSQL

>n8 n5
id | parent | name
7  | 5      | MySQL

>n9 n5
id | parent | name
8  | 5      | SQLite

>n6 n4
id | parent | name
9  | 4      | Komercyjne

>n10 n1
id | parent | name
10 | 1      | XML

\end{verbatim}


\operacja{Reprezentacja w SQL}
%\vspace{-0.5cm}
%! method-sql simple.create

Należy zwrócić uwagę na \todo{term}warunek \texttt{ON DELETE CASCADE}. 
Sprawia on, że w razie usunięcia węzła \todo{ustalić terminologię} 
automatycznie zostaną usunięci wszyscy jego potomkowie.


\operacja{Wstawianie danych}

Ta metoda --- jako jedyna w tym rozdziale --- nie wymaga żadnego wstępnego przetwarzania danych.
W efekcie wstawianie węzłów jest bardzo proste.

%! method-sql simple.insert



\operacja{Pobranie korzeni}
Cechą charakterystyczną korzenia jest to, że jego rodzic jest ustawiony na \texttt{NULL}.
Więc pobranie rodziców sprowadza się do tego zapytania:
%! method-sql simple.roots


\operacja{Pobranie rodzica}
\todo{a może by zrobić to złączeniem?}

Identyfikator rodzica znajduje się w każdym węźle.
%Ponieważ przyjęty interface 
Więc zapytanie musi pobrać najpierw rekord o podanym jako parametr identyfikatorze a następnie
--- korzystając z tego identyfikatora rodzica --- wynikowy rekord.
Użyte został SQL z podzapytaniem, lecz równie dobrze można by było zastosować złączenie. 

%! method-sql simple.parent


\operacja{Pobranie dzieci}

Pobieranie dzieci jest bardzo prostą i szybką operacją
--- każdy rekord zawiera identyfikator rodzica.

%! method-sql simple.children


\operacja{Pobranie przodków}
Tu pojawia się po raz pierwszy największa wada tej reprezentacji --- 
dla kilku operacji wymaga wykonania wielu zapytań w bazie danych.

W tym przypadku widać, że pobranie przodków sprowadza się do pobrania aktualnego węzła,
a następnie w kolejnym zapytaniu jego rodzica,
potem rodzica jego rodzica\ldots
aż dotrze się do korzenia drzewa.

%! method-python simple.ancestors



\operacja{Pobieranie potomków}

Algorytm pobieranie potomków jest podobny do pobierania przodków --- też pobiera się po jednym poziomie.
%Różnica polega na tym, że w tym przypadku poziom do więcej niż jeden rekord, a 

%! method-python simple.descendants

Należy zwrócić uwagę na równoczesne pobieranie wszystkich węzłów na danym poziomie.
Wybrany algorytm wymaga wykonania tylko tylu zapytań ile wynosi wysokość poddrzewa,
którego korzeniem jest dany element.


Gdyby zastosować naiwny algorytm pobierania dzieci jednego węzła,
a następnie, rekurencyjnie, dzieci jego dzieci
to wymagał by on wykonania w sumie tylu zapytań ile węzłów ma poddrzewo,
którego korzeniem jest dany element.


\subsection{Uwagi}

Czasem nie ma porzeby pobierać wszystkich potomków a tylko tych różniących się poziomem o nie więcej niż $N$.
Czyli dla $N = 1$ dzieci, dla $N = 2$, dzieci oraz wnuki, itd.
Mając taką dodatkową wiedzę program można wygenerować bardziej optymalne zapytanie.
Zasada działania się nie zmienia (dalej jesto wyszukiwanie wszerz)
ale zamiast łączyć wyniki w wnętrzu programu są one łaczone wewnątrz bazy danych za pomocą operatora \texttt{UNION ALL}.

Poniżej przykład dla $N = 3$.

\begin{verbatim}[sql]
    SELECT * FROM simple WHERE parent = :id
UNION ALL 
    SELECT * FROM simple WHERE parent IN (
        SELECT id FROM simple WHERE parent = :id
    )
UNION ALL
    SELECT * FROM simple WHERE parent IN (
        SELECT id FROM simple WHERE parent IN (
            SELECT id FROM simple WHERE parent = :id
        )
    )
\end{verbatim}

W analogiczny sposób można pobierać przodków.


%SELECT * FROM [kosqx].[dbo].[simple] WHERE parent=1
%UNION ALL 
%SELECT * FROM [kosqx].[dbo].[simple] WHERE parent IN (SELECT id FROM [kosqx].[dbo].[simple] WHERE parent=1)
%UNION ALL
%SELECT * FROM [kosqx].[dbo].[simple] WHERE parent IN (SELECT id FROM [kosqx].[dbo].[simple] WHERE parent IN (SELECT id FROM [kosqx].[dbo].[simple] WHERE parent=1))





%Należy zwrócić uwagę na wykorzystanie operatora \texttt{IN}. Bez niego wydajność metody spada znacząco.

%Należy zwrócić uwagę na równoczesne pobieranie wszystkich węzłów na danym poziomie.
%Bez niego wydajność metody spada znacząco.
%
%Gdyby zastosować naiwny algorytm pobierania dzieci jednego węzła,
%a następnie, rekurencyjnie, dzieci jego dzieci
%%to wymagał by on wykonania w sumie tylu zapytań ile potomków ma dany element plus jeden.
%to wymagał by on wykonania w sumie tylu zapytań ile węzłów ma poddrzewo,
%którego korzeniem jest dany element.
%
%Wybrany algorytm wymaga wykonania tylko tylu zapytań ile wynosi wysokość poddrzewa,
%którego korzeniem jest dany element.

\temat{Wydajność}

\begin{qxtab}{simple}{Wydajność reprezentacji krawędziowej}
%! result-table simple deep3
\end{qxtab}

\begin{qxfig}{simple}{Wydajność reprezentacji krawędziowej}
%! result-chart simple deep3
\end{qxfig}

\index{metoda!krawędziowa|)}
