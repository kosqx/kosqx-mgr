\section{\texttt{CONNECT BY}}
\index{metoda!connect by@\texttt{connect by}|(textbf}\index{Oracle}
% napisać czy to przeszukiwanie w głąb czy wszerz

% http://download.oracle.com/docs/cd/B19306_01/server.102/b14200/queries003.htm
% http://www.dba-oracle.com/t_sql99_with_clause.htm

%% TODO:
% - napisać o sortowaniu wyników
% - operatory, zwłaszcza LEVEL, SYS_CONNECT_BY_PATH,
% - trochę o historii
% - literatura


% prior forces reporting to be from the root out toward the leaves (if the prior column is the
% parent) or from a leaf toward the root (if the prior column is the child).


System zarządzania bazą danych Oracle nie posiada możliwości przetwarzania danych hierarchicznych za pomocą klauzuli \texttt{WITH}.
\index{Wspolne@Wspólne Wyrażenia Tabelowe}
W prawdzie jest ona dostępna od wersji \emph{Oracle 9i release 2}, ale służy wyłacznie do pracy z podzapytaniami.

% po ludzku: ma CTE ale tylko WITH, ale bez WITH RECURSIVE

W to miejsce \emph{Oracle} udostępnia własne rozszerzenie \texttt{CONNECT BY}. 
Jest ono dobrze opisane w \cite{oracle-ref11},
wiec tu zostaną przedstawione tylko podstawowe i najczęściej używane jego możliwości.


\temat{Opis klauzuli \texttt{CONNECT BY}}

\begin{verbatim}[sql]
SELECT expression [,expression]...
    FROM [user.]table
    WHERE condition
    CONNECT BY [PRIOR] expression = [PRIOR] expression
    START WITH expression = expression
    ORDER BY expression
\end{verbatim}


Sposób działania da się w skrócie opisać regułami:
\begin{itemize}
    \item \texttt{START WITH}\index{START WITH@\texttt{START WITH}} wskazuje w którym miejscu drzewa rozpocząć działanie. 
        Klauzula może się znajdować zarówno przed jak i po \texttt{CONNECT BY}.
    \item kierunek przechodzenia po węzłach zależny jest od tego przed którym wyrażeniem stoi \texttt{PRIOR}\index{PRIOR@\texttt{PRIOR}}.
        Nie ma różnicy \verb|CONNECT BY PRIOR a = b| a \verb|CONNECT BY b = PRIOR a|.
    \item klauzula \texttt{WHERE} pozwala na wyeliminowanie z wyniku pojedyńczych rekordów, 
        ale nie usuwa rekordów, które są dostępne po przejściu przez ten pominięty rekord.
    \item zastosowanie zwykłej klauzuli \texttt{ORDER BY} niszczy hierarchiczny układ danych 
        (więc w praktyce rzadko bywa używana).
        Aby umożliwić kontrolowanie kolejności odwiedzania węzłów, 
        w \emph{Oracle 9i} wprowadzono klauzulę \texttt{ORDER SIBLINGS BY}\index{ORDER SIBLINGS BY@\texttt{ORDER SIBLINGS BY}}.
\end{itemize}


% START WITH tells where in the tree to begin. These are the rules:
%  - The position of PRIOR with respect to the CONNECT BY expressions determines which
%          expression identifies the root and which identifies the branches of the tree.
%  - A WHERE clause will eliminate individuals from the tree, but not their descendants (or
%          ancestors, depending on the location of PRIOR).
%  - A qualification in the CONNECT BY (particularly a not equal sign instead of the equal sign)
%          will eliminate both an individual and all of its descendants.
% - CONNECT BY cannot be used with a table join in the WHERE clause.
% 


Przydatne mechanizmy:
\begin{description}
    \item[\texttt{LEVEL}] \index{LEVEL@\texttt{LEVEL}}
        pseudokolumna,
        jest równa $1$ dla korzenia (lub węzła wskazanego przez \texttt{START WITH}), 
        dla dzieci tego węzła jest równa $2$, dla dzieci tych dzieci $3$, i tak dalej. 
    \item[\texttt{CONNECT\_BY\_ISLEAF}] \index{CONNECTBYISLEAF@\texttt{CONNECT\_BY\_ISLEAF}}
        pseudokolumna o wartości $1$ jeśli rekord jest liściem, $0$ w przeciwnym wypadku
    \item[\texttt{CONNECT\_BY\_ISCYCLE}]  \index{CONNECTBYISCYCLE@\texttt{CONNECT\_BY\_ISCYCLE}}
        pseudokolumna o wartości $1$ jeśli rekord ma potomka, który też jest jego przodkiem, $0$ w przeciwnym wypadku
    \item[\texttt{SYS\_CONNECT\_BY\_PATH(kolumna\_wartosci, znak\_separujacy)}] \index{SYSCONNECTBYPATH@\texttt{SYS\_CONNECT\_BY\_PATH()}}
        funkcja zwraca złączoną w listę wartości 
        z kolumny \texttt{kolumna\_wartosci} wchodzące w skład ścieżki pomiędzy korzeniem a aktualnym węzłem.
        Każda wartość jest poprzedzona znakiem \texttt{znak\_separujacy}.
        Dla przykładu \texttt{SYS\_CONNECT\_BY\_PATH(name, '/')} może zwrócić \texttt{/Bazy danych/Obiektowe/db4o}.

        % It returns the path of a column value from root to node, with column values separated by char for each row returned by the CONNECT BY condition.
\end{description}


\temat{Operacje}

\operacja{Pobranie przodków}
%! method-sql connectby.ancestors

% \begin{verbatim}[sql]
% SELECT level, id, parent, name
%   FROM simple
%   START WITH id=7
%   CONNECT BY PRIOR parent =  id
% \end{verbatim}

\operacja{Pobieranie potomków}
%! method-sql connectby.descendants

% \begin{verbatim}[sql]
% SELECT level, id, parent, name
%   FROM simple
%   START WITH id=1
%   CONNECT BY parent = PRIOR id
% \end{verbatim}


\temat{Wydajność}

\begin{qxtab}{connectby}{Wydajność metody \texttt{CONNECT BY}}
%! result-table connectby
\end{qxtab}

\begin{qxfig}{connectby}{Wydajność metody \texttt{CONNECT BY}}
%! result-chart connectby
\end{qxfig}

\temat{Uwagi}

Oracle w wersjach wcześniejszych od 11g wykonując zapytanie korzystające
z klauzuli \texttt{CONNECT BY} korzysta z \emph{full table scan}.
Dlatego wydajność tych metod nie jest dobra.



% http://www.postgresql.org/docs/current/static/tablefunc.html
% wajig install postgresql-contrib-8.3

%\podtemat{PostgreSQL \texttt{connectby()}}
%
%PostgreSQL dostarcza moduł \texttt{tablefunc}\index{tablefunc} zawierający różne funkcje zwracające tabele.
%% Ta funkcjonalność pojawiła się w wersji 7.4, natomiast dokumentacja dal niej dopiero w wersji 8.3.
%Są one zaimplementowane w C\index{C} dla większej wydajności.
%
%Jedną z dostępnych funkcji jest \texttt{connectby()}, będąca odpowiednikiem wyrażenia \texttt{CONNECT BY}. Jej składnia\todo{składnia?} wygląda następująco:
%
%\begin{verbatim}[sql]
%connectby(
%  text relname,          -- nazwa tabeli
%  text keyid_fld,        -- 
%  text parent_keyid_fld, -- 
%  [text orderby_fld,]    -- 
%  text start_with,       -- 
%  int max_depth,         -- 
%  [text branch_delim])   -- 
%\end{verbatim}




\index{metoda!connect by@\texttt{connect by}|)}




% \begin{verbatim}[sql]
% SELECT sum(empcount) FROM STRUCREL
%    CONNECT BY PRIOR superdept = deptid
%      START WITH deptname = 'Production';
% \end{verbatim}



% P main.py sql oracle 'SELECT level, id, parent, name FROM simple START WITH id=1 CONNECT BY parent = PRIOR id'
% +-------+----+--------+-------------+
% | level | id | parent | name        |
% +-------+----+--------+-------------+
% | 1     | 1  | None   | Bazy Danych |
% | 2     | 2  | 1      | Obiektowe   |
% | 3     | 3  | 2      | db4o        |
% | 2     | 4  | 1      | Relacyjne   |
% | 3     | 5  | 4      | Komercyjne  |
% | 3     | 6  | 4      | Open Source |
% | 4     | 7  | 6      | PostgreSQL  |
% | 4     | 8  | 6      | MySQL       |
% | 4     | 9  | 6      | SQLite      |
% | 2     | 10 | 1      | XML         |
% +-------+----+--------+-------------+
% 

% P main.py sql oracle 'SELECT level, id, parent, name FROM simple START WITH id=7 CONNECT BY PRIOR parent =  id'
% +-------+----+--------+-------------+
% | level | id | parent | name        |
% +-------+----+--------+-------------+
% | 1     | 7  | 6      | PostgreSQL  |
% | 2     | 6  | 4      | Open Source |
% | 3     | 4  | 1      | Relacyjne   |
% | 4     | 1  | None   | Bazy Danych |
% +-------+----+--------+-------------+



