\section{Oracle \texttt{connect by}}
\index{metoda!connect by@\texttt{connect by}|textbf}\index{Oracle}
% napisać czy to przeszukiwanie w głąb czy wszerz

% http://download.oracle.com/docs/cd/B19306_01/server.102/b14200/queries003.htm
% http://www.dba-oracle.com/t_sql99_with_clause.htm

%% TODO:
% - napisać o sortowaniu wyników
% - operatory, zwłaszcza LEVEL, SYS_CONNECT_BY_PATH,
% - trochę o historii
% - literatura

System zarządzania bazą danych Oracle nie posiada możliwości przetwarzania danych hierarchicznych za pomocą klauzuli \texttt{with}.
W prawdzie jest ona dostępna od wersji \emph{Oracle 9i release 2} ale służy wyłacznie do pracy z podzapytaniami.

W to miejsce \emph{Oracle} udostępnia własne rozszerzenie.



\operacja{Pobranie przodków}

%! method-sql connectby.ancestors

% \begin{verbatim}[sql]
% SELECT level, id, parent, name
%   FROM simple
%   START WITH id=7
%   CONNECT BY PRIOR parent =  id
% \end{verbatim}

\operacja{Pobieranie potomków}
%! method-sql connectby.descendants

% \begin{verbatim}[sql]
% SELECT level, id, parent, name
%   FROM simple
%   START WITH id=1
%   CONNECT BY parent = PRIOR id
% \end{verbatim}


\operacja{Wyniki}

\begin{table}[h!]
  \caption{Wyniki \texttt{CONNECT BY}}
  \begin{center}
%! result-table connectby deep3
  \end{center}
\end{table}

\begin{figure}[h!t]
  \caption{Wyniki \texttt{CONNECT BY}}
  \label{fig:img_chart_simple}
  \begin{center}
%! result-chart connectby deep3
  \end{center}
\end{figure}


\operacja{Uwagi}

Oracle w wersjach wcześniejszych od 11g wykonując zapytanie korzystające z klauzuli \texttt{CONNECT BY}



% http://www.postgresql.org/docs/current/static/tablefunc.html
% wajig install postgresql-contrib-8.3

\subsection*{PostgreSQL \texttt{connectby()}}

PostgreSQL dostarcza moduł \texttt{tablefunc}\index{tablefunc} zawierający różne funkcje zwracające tabele.
% Ta funkcjonalność pojawiła się w wersji 7.4, natomiast dokumentacja dal niej dopiero w wersji 8.3.
Są one zaimplementowane w C\index{C} dla większej wydajności.

Jedną z dostępnych funkcji jest \texttt{connectby()}, będąca odpowiednikiem wyrażenia \texttt{CONNECT BY}. Jej składnia\todo{składnia?} wygląda następująco:

\begin{verbatim}[sql]
connectby(
  text relname,          -- 
  text keyid_fld,        -- 
  text parent_keyid_fld, -- 
  [text orderby_fld,]    -- 
  text start_with,       -- 
  int max_depth,         -- 
  [text branch_delim])   -- 
\end{verbatim}


% \begin{verbatim}[sql]
% SELECT sum(empcount) FROM STRUCREL
%    CONNECT BY PRIOR superdept = deptid
%      START WITH deptname = 'Production';
% \end{verbatim}



% P main.py sql oracle 'SELECT level, id, parent, name FROM simple START WITH id=1 CONNECT BY parent = PRIOR id'
% +-------+----+--------+-------------+
% | level | id | parent | name        |
% +-------+----+--------+-------------+
% | 1     | 1  | None   | Bazy Danych |
% | 2     | 2  | 1      | Obiektowe   |
% | 3     | 3  | 2      | db4o        |
% | 2     | 4  | 1      | Relacyjne   |
% | 3     | 5  | 4      | Komercyjne  |
% | 3     | 6  | 4      | Open Source |
% | 4     | 7  | 6      | PostgreSQL  |
% | 4     | 8  | 6      | MySQL       |
% | 4     | 9  | 6      | SQLite      |
% | 2     | 10 | 1      | XML         |
% +-------+----+--------+-------------+
% 

% P main.py sql oracle 'SELECT level, id, parent, name FROM simple START WITH id=7 CONNECT BY PRIOR parent =  id'
% +-------+----+--------+-------------+
% | level | id | parent | name        |
% +-------+----+--------+-------------+
% | 1     | 7  | 6      | PostgreSQL  |
% | 2     | 6  | 4      | Open Source |
% | 3     | 4  | 1      | Relacyjne   |
% | 4     | 1  | None   | Bazy Danych |
% +-------+----+--------+-------------+



