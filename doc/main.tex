\documentclass[12pt,a4paper,oneside]{report}
%\documentclass[10pt,a4paper,oneside]{book}
%\documentclass[10pt,a4paper,draft]{book}
%\documentclass[sfheadings,a4paper,11pt]{mwbk}

\usepackage{polski}
\usepackage[utf8]{inputenc}

\usepackage{ifpdf}

\usepackage{geometry}
\geometry{verbose,a4paper,tmargin=2.5cm,bmargin=2.5cm,lmargin=3.5cm,rmargin=2.5cm}

% \newif\ifpdf
% \ifx\pdfoutput\undefined
%    \pdffalse
% \else
%    \pdfoutput=1
%    \pdftrue
% \fi

\ifpdf
   \usepackage[pdftex]{graphics}
   \usepackage[pdftex]{color}
   \pdfcompresslevel9
\else
   \usepackage{graphics}
   \usepackage{color}
\fi

\usepackage{indentfirst}
\sloppy
\clubpenalty = 10000
\widowpenalty = 10000
\frenchspacing

% dla łatwiejszego sprawdzania
\linespread{1.3}
% \linespread{1.0}
\setcounter{tocdepth}{1}

\usepackage{amsmath}
\usepackage{amsfonts}
\usepackage{amssymb}
\usepackage{makeidx}

%\usepackage{showidx}
\usepackage{makeidx}
\makeindex

\author{Krzysztof Kosyl}
\title{Przechowywanie danych hierarchicznych w relacyjnych bazach danych}


\newcommand{\HRule}{\rule{\linewidth}{0.5mm}}
\newcommand{\operacja}[1]{\subsubsection*{#1}}

% wyświetla TODO: wrazie potrzeby zmodyfikować
\newcommand{\todo}[1]{(\textcolor{red}{\textbf{#1}})}

% numery stron u góry
\pagestyle{headings}

% rotacje w nagłówku kolumny
\usepackage{rotating}

% dla pygments // \begin{Verbatim}
\usepackage{fancyvrb}
%! pygments-style


\usepackage[unicode=true,pdftex,backref]{hyperref}
\hypersetup{
%kolory
  colorlinks=true,
  linkcolor=blue,
  %anchorcolor=black,
  %citecolor=green,
  %filecolor=magenta,
  %menucolor=red,
  %pagecolor=red,
  %urlcolor=cyan,
%zakladki
  bookmarks=true,
  bookmarksopen=true,
  bookmarksnumbered=true,
%metadane
  pdfauthor = {Krzysztof Kosyl},
  pdftitle = {Przechowywanie danych hierarchicznych w relacyjnych bazach danych},
  pdfsubject = {Trees in relational DB},
  pdfkeywords = {tree, db},
}




\begin{document}

%% desc: strona tytułowa

% To już nie jest używane
% \maketitle{}


\begin{titlepage}
	\begin{center}
		\textsc{Uniwersytet Mikołaja Kopernika -- Wydział Matematyki i Informatyki}
	\end{center}

	\vfill

	\begin{flushleft}
		Krzysztof Kosyl\\
		Numer albumu: 187411
	\end{flushleft}

	\vfill\vfill

	\begin{center}
		\HRule \\[0.4cm]
		\huge{\textbf{Przechowywanie danych hierarchicznych w relacyjnych bazach danych}}
		\HRule \\[0.4cm]
	\end{center}

	\vfill\vfill\vfill\vfill

	\begin{flushright}
		%Praca magisterska na kierunku:\\
		%Programowanie i Przetwarzanie Informacji
		Promotor:\\
		prof. dr hab Krzysztof Stencel
	\end{flushright}

	\vfill
	\begin{center}
		Toruń 2008
	\end{center}
\end{titlepage}

%% desc: abstract

% \begin{abstract}
% 	O czym jest ten dokument.
% \end{abstract} 

\tableofcontents{}

% \chapter*{Wstęp}
\chapter*{Wstęp}
\addcontentsline{toc}{chapter}{Wstęp}

% W ciągu ostatnich kilkunastu lat relacyjne bazy danych opanowały świat. 

Celem tej pracy jest zaprezentowanie metod przechowywania danych w relacyjnych bazach danych. 

Jest to zagadnienie stare niemal jak same relacyjne bazy danych. Już Edgar Frank Codd zaproponował jedno z rozwiązań.
\todo{sprawdzić gdzie} 
Miało być ono obroną relacyjnego modelu danych przed zarzutami, że w takich bazach nie można przechowywać danych hierarchicznych.



% porównanie wydajności, mocnych stron, słabych stron
% nie należy się spodziewać idealnej metody, raczej najlepszed do danego zastosowania





% \chapter{Wprowadzenie do tematu}
\chapter{Wprowadzenie do tematu}

\section{Czym jest drzewo}
\index{drzewo|textbf}
Drzewo to bardzo powszechnie używane w informatyce pojęcie. W zależności od zastosowania może być różnie zdefiniowane.

\paragraph{Drzewo jako graf}
\index{graf}
\paragraph{Drzewo jako struktura rekurencyjna}
\index{rekurencja}


Tu: matematyczny opis drzew, ich podstawowe własności, omówienie terminów:
\begin{itemize}
    \item drzewo
    \item las
    \item rodzic
    \item przodek
    \item korzeń
    \item dziecko
    \item potomek
    \item głębokość, mini
\end{itemize}

\section{Podział drzew}
\subsection{Jednorodność}
Warto by dodać o drzewach jednorodnych oraz niejednorodnych.

\subsection{Kolejność}


\section{Podstawowe algorytmy dla drzew}
\subsection{Wyszukiwanie w głąb}
\index{drzewo!wyszukiwanie!w głąb}
\subsection{Wyszukiwanie wszerz}
\index{drzewo!wyszukiwanie!wszerz}


\section{Różnice pomiędzy drzewami w algorytmice a w bazach danych}

Nie ma sensu coś takiego jak drzewo czerwono-czarne gdyż w drzewach w bazach danych chodzi o strukturę a nie o optymalizację czasu dostępu.

\section{Tematy porównania - operacje}



\subsection{Operacje}
\subsection{Pobranie korzeni}
\index{drzewo!korzeń}
\subsection{Pobranie przodków}
\index{drzewo!przodekowie}
\subsection{Pobranie rodzica}
\index{drzewo!rodzic}
\subsection{Pobranie dzieci}
\index{drzewo!dzieci}
\subsection{Pobranie potomków}
\index{drzewo!potomkowie}
\subsection{Pobranie (najmłodszego) wspólnego przodka}



% \begin{description}
%   \item[pobranie rodzica] polega na pobraniu rodzica bieżącego elementu
%   \item[pobranie przodków] polega na pobraniu rodzica, rodzica rodzica aż do korzenia elementów
% \end{description}

\section{Dostępność}
\subsection{Mapowanie relacyjno-obiektowe}
\subsection{W zależności od bazy danych}


\section{Przyjęte założenia}
\todo{więcej wypisać, zastanowić się co z tym zrobić}

\paragraph{Tabela zawiera tylko jedną kolumnę z danymi użytkownika} Bez problemu można dodać więcej kolumn do tabeli. Obecność wyłącznie kolumny \texttt{name} zwiększa czytelność przykładów.
\paragraph{Zapytania wyłącznie związane z hierarchiczną strukturą danych}

\chapter{Podstawowe metody przechowywania danych}

Metody wymienione w tym rozdziale są uniwersalne. Daje się je zaimplementować w każdej relacyjnej bazie danych. 
Można również (przy odrobinie wysiłku) kożystać z nich również w \emph{mapowaniach obiektowo-relacyjnych}\index{ORM}.
Nic nie stoi na przeszkodzie aby wykożystać te metody również w obiektowych bazach danych.

	% \section{Metoda stałej wysokości drzewa}
	% Przenieść do sekcji o metodach, o których nie będziemy mówili. Takich jak również, Dział -- pracownik.
	
	% \section{Metoda klucza obcego do rodzica}
	\section{Metoda listy sąsiedztwa}


% Hierarchies like trees, organizational charts, ... are sometimes difficult to
% store in database tables. The most common database pattern to store 
% hierachical data is known as the adjacency model. It has been introduced 
% by the famous computer scientist Edgar Frank Codd. It's called the "adjacency"
% model because a reference to the parent data is stored in the same row as the
% child data, in an adjacent column. These kind of tables are also called self 
% referencing tables.
% http://www.scip.be/index.php?Page=ArticlesNET18


% It was Scott's pet cat called "tiger".
% And, who was Scott? His first name was Scott but Bruce. Bruce Scott was
% employee number #4 at the then Software Development Laboratories that
% eventually became Oracle. He co-authored and co-architected Oracle V1, V2 & V3.
% http://www.dba-oracle.com/t_scott_tiger.htm

% http://pl.wikipedia.org/wiki/Reprezentacja_grafu#Reprezentacja_przez_listy_s.C4.85siedztwa



Najprostszą, najbardziej intuicyjną i zapewne najpopularniejszą metodą jest metoda przyległej listy.
Została ona spopularyzowana przez Edgara Franka Codda. 
On też nadał im nazwę, która odnosi się do tego, że informacja o rodzicu elementu znajduje się w tej samej krotce co dane.

Zasługa spopularyzowania tej metody przypada Oracle. 
Dołączył on do swojego produktu przykładową bazę danych, nazywaną ,,Scott/Tiger'' 
korzystającą z tej metody.
\footnote{
    Nazwa tej bazy danych pochodzi od metody autoryzacji w bazie Oracle (login/hasło).
    Login pochodził z nazwiska jednego z pierwszych pracowników 
    Software Development Laboratories (przekształconych ostatecznie w Oracle) Bruce'a Scott'a. 
    Natomiast hasło to imię jego kota.
}

Na popularność metody przekłada się również jej znaczące podobieństwo do 
używanego między innymi w~językach C i~C++ sposobu przechowywania list i drzew.
Mianowicie każdy węzeł zawiera wskaźnik na rodzica. 
W relacyjnych bazach danych odpowiednikiem wskaźnika jest klucz obcy. 

\paragraph{Definicja danych}

\begin{verbatim}[sql]
CREATE TABLE tree (  
  id INTEGER PRIMARY KEY,  
  parent INTEGER  
    REFERENCES tree(id)  
    ON DELETE CASCADE,  
  value VARCHAR(100)  
);
\end{verbatim}

\paragraph{Pobieranie potomków}

Należy zwrócić uwagę na wykorzystanie operatora \emph{IN}. Bez niego wydajność metody spada znacząco. 

\paragraph{Wstawianie danych}



	% \section{Metoda zagnieżdzonych zbiorów}
	\section{Metoda zagnieżdzonych zbiorów}
\index{metoda!zagnieżdzonych zbiorów|textbf}

Metoda została spopularyzowana przez Joe Celko\cite{celkosql}\index{Celko Joe}.



\operacja{Reprezentacja w SQL}
%! method-sql nested.create

\operacja{Wstawianie danych}
% %! method-sql nested.insert

% \begin{verbatim}[sql]
% INSERT INTO simple (parent, name) VALUES (:parent, :name)
% \end{verbatim}

Wstawianie danych w tej metodzie jest jej najsłabszą stroną.


\operacja{Pobranie korzeni}
%! method-sql nested.roots

\operacja{Pobranie rodzica}
% %! method-sql nested.parent

\operacja{Pobranie dzieci}
% %! method-sql nested.children

\operacja{Pobranie przodków}
%! method-sql nested.ancestors

\operacja{Pobieranie potomków}
%! method-sql nested.descendants


\begin{table}[h!]
  \caption{Wyniki reprezentacji zagnieżdzonych zbiorów}
   \begin{center}
%! result-table nested deep3
   \end{center}
\end{table}

\begin{figure}[h!t]
  \caption{Wyniki reprezentacji zagnieżdzonych zbiorów}
  \label{fig:img_chart_nested}
  \begin{center}
%! result-chart nested deep3
  \end{center}
\end{figure}



	% \section{Metoda pełnych ścieżek}
	\section{Metoda pełnych ścieżek}
\index{metoda!pełnych ścieżek|textbf}
% http://troels.arvin.dk/db/rdbms/links/#hierarchical
% 
% http://en.wikipedia.org/wiki/Transitive_closure
% http://pl.wikipedia.org/wiki/Domknięcie_przechodnie

% Domknięcie_przechodnie Cormen:644

Metoda został spopularyzowana w Polskim Internecie przez \emph{Huberta Lubaczewskiego}\index{Lubaczewski Hubert} lepiej znanego pod pseudonimem \emph{depesz}\index{depesz|see{Lubaczewski Hubert}}.

\index{drzewo!domknięcie przechodnie}

Idea metody jest prosta. 
Głowna tabela z danymi nie zawiera żadnych informacji o hierarchi danych. 
Jedynym wymogiem jest istnienie w niej klucza głównego\index{klucz!główny}.

Cała informacja potrzebna do operowania na drzewie zawiera się w dodatkowej tabeli. 
Zawiera ona informacje o odległości pomiędzy każdym elementem a wszystkimi jego potomkami.
 

\begin{table}[h!]
  \caption{Wyniki reprezentacji pełnych ścieżek}
   \begin{center}
%! result-table full deep3
   \end{center}
\end{table}

\begin{figure}[h!t]
  \caption{Wyniki reprezentacji pełnych ścieżek}
  \label{fig:img_chart_nested}
  \begin{center}
%! result-chart full deep3
  \end{center}
\end{figure}

	
	% \section{Metoda drzew prefiksowych (trie)}
	\section{Metoda drzew prefiksowych (trie)}


	% drzewa binarne
	% http://commons.wikimedia.org/wiki/File:Binary_tree_in_array.svg
	% http://en.wikipedia.org/wiki/K-ary_tree


\chapter{Modifikacje metod przechowywania danych}
	% \section{Metoda łączona}


\chapter{Metody specyficzne dla bazy danych}
	\section{PL/SQL}

	% \section{IBM DB2 \texttt{with}}
	\section{IBM DB2 \texttt{with}}

% http://www.ibm.com/developerworks/data/library/techarticle/0307steinbach/0307steinbach.html

Opis standardowej metody with która pojawiła się w standardzie SQL??. Porównanie z connect by

\begin{verbatim}
WITH temptab(deptid, empcount, superdept) AS
   (    SELECT root.deptid, root.empcount, root.superdept
            FROM departments root
            WHERE deptname='Production'
     UNION ALL
        SELECT sub.deptid, sub.empcount, sub.superdept
            FROM departments sub, temptab super
            WHERE sub.superdept = super.deptid
   )
SELECT sum(empcount) FROM temptab
\end{verbatim}



	% \section{Oracle \texttt{connect by}}
	\section{Oracle \texttt{connect by}}

% napisać czy to przeszukiwanie w głąb czy wszerz

% http://download.oracle.com/docs/cd/B19306_01/server.102/b14200/queries003.htm

\begin{verbatim}[sql]
SELECT sum(empcount) FROM STRUCREL
   CONNECT BY PRIOR superdept = deptid
     START WITH deptname = 'Production';
\end{verbatim}


	% \section{PostgreSQL \texttt{ltree}}
	\section{PostgreSQL \texttt{ltree}}
\index{metoda!ltree@\texttt{ltree}|textbf}\index{PostgreSQL}
	
	% \section{Microsoft SQL Server \texttt{hierarchyid}}
	\section{Microsoft SQL Server \texttt{hierarchyid}}
	% http://technet.microsoft.com/en-us/library/bb677173.aspx
	% http://blogs.msdn.com/manisblog/archive/2007/08/17/sql-server-2008-hierarchyid.aspx
	% http://www.microsoft.com/poland/technet/bazawiedzy/centrumrozwiazan/cr314_01.mspx


% Apparently the HierarchyID uses the ORDPATH algorithm (as far as I'm concerned). 
% Just found the following document that elaborates somewhat on the algorithm and other things related to hierarchical storage:
% http://sites.computer.org/debull/a07mar/kumaran.pdf


% SQL Server 2008 adds a new feature to help with modeling hierarchical relationships: the HIERARCHYID data type. It provides compact storage and convenient methods to manipulate hierarchies. In a way it is very much like optimized materialized path. In addition the SqlHierarchyId CLR data type is available for client applications. 
% 
% While HIERARCHYID has a lot to offer in terms of operations with hierarchical data, it is important to understand a few basic concepts:
% 
% - HIERARCHYID can have only a single root (although easy to work around by adding sub-roots)
% - It does not automatically represent a tree, the application has to define the relationships and enforce all rules 
% - The application needs to maintain the consistency
% http://pratchev.blogspot.com/2008/05/hierarchies-in-sql-server-2008.html

% http://technet.microsoft.com/en-us/library/cc721270.aspx  !! TODO
% The new HIERARCHYID data type in SQL Server 2008 is a system-supplied CLR UDT that can be useful for storing and manipulating hierarchies. This type is internally stored as a VARBINARY value that represents the position of the current node in the hierarchy (both in terms of parent-child position and position among siblings). You can perform manipulations on the type by using either Transact-SQL or client APIs to invoke methods exposed by the type. Let’s look at indexing strategies for the HIERARCHYID type, how to use the type to insert new nodes into a hierarchy, and how to query hierarchies.

%% TODO: 
% - opis algorytmu (zoptymalizowane ścieżki zmaterializowane)
% - opis wszystkich funkcji
% - http://www.cs.umb.edu/~poneil/ordpath.pdf google:ORDPATH
% - http://msdn.microsoft.com/en-us/library/bb677173.aspx

Jedną z najciekawszych nowości jakie Microsoft wprowadził w SQL Server 2008 jest nowy typ danych \texttt{hierarchyid}. 
Pozwala on na wygodne przechowywanie danych hierarchicznych.

W budowie wewnętrznej przypomina 


\subsection*{Opis typu \texttt{hierarchyid}}

\paragraph{Budowa}

Typ \texttt{hierarchyid} jest przechowywany wewnętrznie jako \texttt{VARBINARY}. 


\paragraph{Indeksowanie}



\begin{description}
  \item[\texttt{child.GetAncestor(n)}]
	This method is useful to find the (nth ancestor of the given child node.

  \item[\texttt{parent.GetDescendant(child1, child2)}]
	This method is very useful to get the descendant of a given node. 
	It has a great significance in terms of finding the new descendant position get the descendants etc. 

	This function returns one child node that is a descendant of the parent. 
	If parent is NULL, returns NULL. 
	If parent is not NULL, and both child1 and child2 are NULL, returns a child of parent. 
	If parent and child1 are not NULL, and child2 is NULL, returns a child of parent greater than child1. 
	If parent and child2 are not NULL and child1 is NULL, returns a child of parent less than child2. 
	If parent, child1, and child2 are all not NULL, returns a child of parent greater than child1 and less than child2. 
	If child1 or child2 is not NULL but is not a child of parent, an exception is raised. 
	If child1 >= child2, an exception is raised.

  \item[\texttt{node.GetLevel()}]
	Zwraca liczbę całowitą będącą poziomem danego węzła w drzewie\todo{słownik}. 
	Korzeń ma poziom równy $0$. 
	Jako, że typ \texttt{hierarchyid} nie obsługuje lasów to aby to kompensować tworzy się sztuczny korzeń, a korzenie obsługiwanych drzew znajdują się na poziomie $1$.

	% This function will return an integer that represents the depth of this node in the current tree. 

  \item[\texttt{hierarchyid::GetRoot()}]
	This method will return the root of the hierarchy tree and this is a static method if you are using it within CLR. 
	It will return the data type hierarchyID. 

  \item[\texttt{parent.IsDescendant(child)}]
	This method returns true/false (BIT) if the node is a descendant of the parent. 

  \item[\texttt{hierarchyid::Parse (input)}]
	Parse converts the canonical string representation of a hierarchyid to a hierarchyid value. 
	Parse is called implicitly when a conversion from a string type to hierarchyid occurs. 
	Acts as the opposite of ToString(). 
	Parse() is a static method. 

  \item[\texttt{void Read( BinaryReader r )}]
	Read reads binary representation of SqlHierarchyId from the passed-in BinaryReader and sets the SqlHierarchyId object to that value. 
	Read cannot be called by using Transact-SQL. Use CAST or CONVERT instead.

  \item[\texttt{node.Reparent(oldRoot, newRoot)}]
	This is a very useful method which helps you to reparent a node i.e. suppose if we want to align an existing node 
	to a new parent or any other existing parent then this method is very useful. 

  \item[\texttt{node.ToString()}]
	This method is useful to get the string representation of the HierarchyID. 
	The method returns a string that is a nvarchar(4000) data type.


  \item[\texttt{void Write( BinaryWriter w )}]
	Write writes out a binary representation of SqlHierarchyId to the passed-in BinaryWriter. 
	Write cannot be called by using Transact-SQL. Use CAST or CONVERT instead.

 \end{description}

\subsection*{Operacje}

\operacja{Reprezentacja w SQL}
%! method-sql hierarchyid.create

\operacja{Wstawianie danych}
%! method-sql hierarchyid.insert

\operacja{Pobranie korzeni}
%! method-sql hierarchyid.roots

\operacja{Pobranie rodzica}
%! method-sql hierarchyid.parent

\operacja{Pobranie dzieci}
%! method-sql hierarchyid.children

\operacja{Pobranie przodków}
%! method-sql hierarchyid.ancestors

\operacja{Pobieranie potomków}
%! method-sql hierarchyid.descendants

\operacja{Uwagi}

SQL Server udostępnia \texttt{PERSISTED} --- umożliwiające dynamiczne tworzenie dynamicznych kolumn. \todo{lepiej to sformuować}
Przykładowo może zostać to użyte do łatwego pobrania

\begin{verbatim}[sql]
CREATE TABLE tree (
  node hierarchyid PRIMARY KEY CLUSTERED,
  level AS node.GetLevel() PERSISTED,
  name varchar(50)
);
SELECT level, name FROM tree;
\end{verbatim}


% P main.py sql sqlserver  'SELECT *, master.dbo.fn_varbintohexstr(cast(node as varbinary)) a, node.ToString() t FROM herid'
% +----+------+-------------+----------+-----------+
% | id | node | name        | a        | t         |
% +----+------+-------------+----------+-----------+
% | 1  |      | ROOT        | None     | /         |
% | 2  | X    | Bazy Danych | 0x58     | /1/       |
% | 3  | Z�   | Obiektowe   | 0x5ac0   | /1/1/     |
% | 4  | Z�   | db4o        | 0x5ad6   | /1/1/1/   |
% | 5  | [@   | Relacyjne   | 0x5b40   | /1/2/     |
% | 6  | [V   | Komercyjne  | 0x5b56   | /1/2/1/   |
% | 7  | [Z   | Open Source | 0x5b5a   | /1/2/2/   |
% | 8  | [Z�  | PostgreSQL  | 0x5b5ab0 | /1/2/2/1/ |
% | 9  | [Z�  | MySQL       | 0x5b5ad0 | /1/2/2/2/ |
% | 10 | [Z�  | SQLite      | 0x5b5af0 | /1/2/2/3/ |
% | 11 | [�   | XML         | 0x5bc0   | /1/3/     |
% +----+------+-------------+----------+-----------+





% \chapter{Porównanie wydajności}
% \section{Po czym porównywać?}
% \subsection{Metodzie}
% \subsection{Bazie danych}



Dane użyte podczas testów zostały wygenerowane losowo w sposób automatyczny. Program który zajmuje się tym zadaniem został dołączony do tej pracy.

% \chapter{Możliwości korzystania z danych hierarchicznych w O/RM}
% \section{Active Record}
% \section{Hibernate}
% \subsection{HQL}
% Ten akapit opisuje HQL\index{HQL}.
% Przydatność: - jeśli przy move trzeba aktualizować mnóstwo elementów to metoda będzie bardzo czasochłonna dla OR/M


% \chapter{Zastosowania i przykłady}

% \section{LDAP}
% \section{DNS?}
% \section{XML}
% \subsection{Różnice pomiędzy XML a drzewami}
% \begin{itemize}
%  \item kolejność jest istotna
%  \item język XPath i ścieżki w dokumencie
% \end{itemize}

\appendix

\chapter{Metodyka testów}

\section{Wybrane SZDB}

Do testów zostały wybrane popularne, dostępne bezpłatnie\footnote{również do zastosowań komercyjnych} bazy danych. 
W przypadku baz Open Source wykorzystano najnowsze, stabilne wersje.
Dla baz komercyjnych zostały wybrane ich darmowe edycje. Posiadają one
ograniczenia co do wielkości baz danych, wykorzystania zasobów oraz zmniejszoną funkcjonalność. 
Specyfika testów sprawia jednak, że te ograniczenia nie miały znaczenia podczas testów.


\begin{itemize}
 \item PostgreSQL\index{PostgreSQL}
 \item MySQL\index{MySQL}
 \item SQLite
 \item Oracle Database 10g Express Edition
 \item IBM DB2 Express-C
 \item Microsoft SQL Server 2008 Express
\end{itemize}





% Wybrane zostały SZDB spełniające następujące cechy:
% \begin{itemize}
%  \item popularne
%  \item darmowe do użytku domowego 
% \end{itemize}


% Dla porównania innych rozwiązań zostały dobrane następujące bazy danych:
% 
% \begin{itemize}
%  \item db4o
%  \item berkeley DB
%  \item ??
% \end{itemize}



\section{Dane testowe}



Aby przetestować przedstawione rozwiązania wykożystane zostały następujące zestawy danych:
\begin{itemize}
 \item mały test 100 dwupoziomowo
 \item test mocnego zagłębienia
 \item test małego zagłebienia
 \item test 6 pozimów po 3 zagłębienia 
\end{itemize}


Parametry generatora danych:
\begin{itemize}
 \item minimalny poziom zagłębienia
 \item maksymalny poziom zagłębienia
 \item rozklad prawdopodobieństwa wygenerowania dzieci (\cite{asdf}, \cite[Ala]{asdf})
\end{itemize}


\section{Środowisko testowe}

Aby zminimalizować potencjalne problemy wynikające z instalacji wielu baz na jednej maszynie dla potrzeb testów zostały stworzone dla nich odzielne maszyny wirtualne. Wykorzystano VirtualBox 2.1. Każdej maszynie został przyznany 10GB virtualny dysk o stałym rozmiarze oraz 512 MB RAM. Program testujący działał również na maszynie wirtualnej, więc ominięto problem przepustowości interfacu sieciowego. \todo{lepsze nazwy; wersja; usunac powtórzenia}

Bazy posiadające wersję dla systemu GNU/Linux zostały zainstalowane na nim. Baza Microsoft SQL Server została zainstalowana na Windows XP Service Pack 2.


\chapter{Program testujący}

Do tej pracy jest dołączone oprogramowanie umożliwiające łatwe sprawdzanie
opisanych metod. 

% Aby zautomatyzować przeprowadzanie testów został stworzony program testujący. 


%\bibliographystyle{plain}
%\bibliography{main}

\clearpage
\phantomsection
\addcontentsline{toc}{chapter}{Literatura}
\begin{thebibliography}{99}
\thispagestyle{empty}

\bibitem[Cor04]{cormen} Thomas H. Cormen, Charles E. Leiserson, Ronald L. Rivest, Clifford Stein, 
    \emph{Wprowadzenie do algorytmów},
    Wydawnictwa Naukowo-Techniczne, Warszawa 2004

\bibitem[Dro04]{drozdek} Adam Drozdek
    \emph{C++. Algorytmy i struktury danych},
    Helion, Gliwice 2004

\bibitem[Knu02]{knuth} Donald E. Knuth
    \emph{Sztuka programowania. Tomy 1-3},
    Wydawnictwa Naukowo-Techniczne, Warszawa 2002

\bibitem[Cel00]{calko-sql} Joe Celko,
    \emph{SQL. Zaawansowane techniki programowania},
    MIKOM, 2000

\bibitem[Cel04]{celko-tree} Joe Celko,
    \emph{Joe Celko`s Trees and Hierarchies in SQL for Smarties}, 
    Morgan Kaufmann, 2004

\bibitem[Wal08]{apress-sqlserver} Robert E. Walters, Michael Coles, Robert Rae, Fabio Ferracchiati, Donald Farmer
    \emph{Accelerated SQL Server 2008},
    Apress, 2008

% \bibitem[]{} 
%     \emph{},
%     
% \bibitem[]{} 
%     \emph{},
%     

\end{thebibliography} 



%\listoffimakegures
%\listoftables

\clearpage
\phantomsection
\addcontentsline{toc}{chapter}{\indexname}
\printindex


\end{document}


%% snippets

% \begin{table}[h!]
%   \caption{Wyniki rmeprezentacji zagnieżdzonych zbiorów}
%    \begin{center}
% %! result-table nested deep3
%    \end{center}
% \end{table}
% 
% \begin{figure}[h!t]
%   \caption{Wyniki reprezentacji zagnieżdzonych zbiorów}
%   \label{fig:img_chart_nested}
%   \begin{center}
% %! result-chart nested deep3
%   \end{center}
% \end{figure}