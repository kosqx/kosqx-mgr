\documentclass[10pt,a4paper,oneside]{book}
%\documentclass[10pt,a4paper,draft]{book}
%\documentclass[sfheadings,a4paper,11pt]{mwbk}

\usepackage{polski}
\usepackage[utf8]{inputenc}

\usepackage{ifpdf}

% \setcounter{tocdepth}{1}
\usepackage{geometry}
\geometry{verbose,a4paper,tmargin=2.5cm,bmargin=2.5cm,lmargin=3.5cm,rmargin=2.5cm}

%\usepackage[utf8x]{inputenc}
%\usepackage{ucs}

% \newif\ifpdf
% \ifx\pdfoutput\undefined
%    \pdffalse
% \else
%    \pdfoutput=1
%    \pdftrue
% \fi

\ifpdf
   \usepackage[pdftex]{graphics}
   \usepackage[pdftex]{color}
   \pdfcompresslevel9
\else
   \usepackage{graphics}
   \usepackage{color}
\fi

\usepackage{indentfirst}
\sloppy
\clubpenalty = 10000
\widowpenalty = 10000
\frenchspacing

\linespread{1.3}

\usepackage{amsmath}
\usepackage{amsfonts}
\usepackage{amssymb}
\usepackage{makeidx}


\author{Krzysztof Kosyl}
\title{Przechowywanie danych hierarchicznych w relacyjnych bazach danych}

\makeindex

\newcommand{\HRule}{\rule{\linewidth}{0.5mm}}

\usepackage[unicode=true,pdftex,backref]{hyperref}
\hypersetup{
%kolory
  colorlinks=true,
  linkcolor=blue,
  %anchorcolor=black,
  %citecolor=green,
  %filecolor=magenta,
  %menucolor=red,
  %pagecolor=red,
  %urlcolor=cyan,
%zakladki
  bookmarks=true,
  bookmarksopen=true,
  bookmarksnumbered=true,
%metadane
  pdfauthor = {Krzysztof Kosyl},
  pdftitle = {Przechowywanie danych hierarchicznych w relacyjnych bazach danych},
  pdfsubject = {Trees in relational DB},
  pdfkeywords = {tree, db},
}





\begin{document}

%% desc: strona tytułowa

% To już nie jest używane
% \maketitle{}


\begin{titlepage}
	\begin{center}
		\textsc{Uniwersytet Mikołaja Kopernika -- Wydział Matematyki i Informatyki}
	\end{center}

	\vfill

	\begin{flushleft}
		Krzysztof Kosyl\\
		Numer albumu: 187411
	\end{flushleft}

	\vfill\vfill

	\begin{center}
		\HRule \\[0.4cm]
		\huge{\textbf{Przechowywanie danych hierarchicznych w relacyjnych bazach danych}}
		\HRule \\[0.4cm]
	\end{center}

	\vfill\vfill\vfill\vfill

	\begin{flushright}
		%Praca magisterska na kierunku:\\
		%Programowanie i Przetwarzanie Informacji
		Promotor:\\
		prof. dr hab Krzysztof Stencel
	\end{flushright}

	\vfill
	\begin{center}
		Toruń 2008
	\end{center}
\end{titlepage}

%% desc: abstract

% \begin{abstract}
% 	O czym jest ten dokument.
% \end{abstract} 

\tableofcontents{}

\chapter*{Wstęp}
\addcontentsline{toc}{chapter}{Wstęp}

\chapter*{Wstęp}
\addcontentsline{toc}{chapter}{Wstęp}

% W ciągu ostatnich kilkunastu lat relacyjne bazy danych opanowały świat. 

Celem tej pracy jest zaprezentowanie metod przechowywania danych w relacyjnych bazach danych. 

Jest to zagadnienie stare niemal jak same relacyjne bazy danych. Już Edgar Frank Codd zaproponował jedno z rozwiązań.
\todo{sprawdzić gdzie} 
Miało być ono obroną relacyjnego modelu danych przed zarzutami, że w takich bazach nie można przechowywać danych hierarchicznych.



% porównanie wydajności, mocnych stron, słabych stron
% nie należy się spodziewać idealnej metody, raczej najlepszed do danego zastosowania





\chapter{Wprowadzenie do tematu}

\section{Co to jest drzewo}

Tu: matematyczny opis drzew, ich podstawowe własności, omówienie terminów:
\begin{itemize}
	\item drzewo
	\item las
	\item rodzic
	\item przodek
	\item korzeń
	\item dziecko
	\item potomek
	\item głębokość, mini
\end{itemize}



\section{Podstawowe algorytmy dla drzew}
\subsection{Wyszukiwanie w głąb}
\subsection{Wyszukiwanie w szerz}


\section{Różnice pomiędzy drzewami w algorytmice a w bazach danych}

Nie ma sensu coś takiego jak drzewo czerwono-czarne gdyż w drzewach w bazach danych chodzi o strukturę a nie o optywamizację czasu dostępu.

\section{Tematy porównania - operacje}



\subsection{Operacje}
\subsection{Pobranie korzeni}
\subsection{Pobranie przodków}
\subsection{Pobranie rodzica}
\subsection{Pobranie dzieci}
\subsection{Pobranie potomków}
\subsection{Pobranie (najmłodszego) wspólnego przodka}



% \begin{description}
% 	\item[pobranie rodzica] polega na pobraniu rodzica bieżącego elementu
% 	\item[pobranie przodków] polega na pobraniu rodzica, rodzica rodzica aż do korzenia elementów
% \end{description}

\section{Dostępność}
\subsection{Mapowanie relacyjno-obiektowe}
\subsection{W zależności od bazy danych}

\chapter{Metody przechowywania danych}

\section{Metoda stałej wysokości drzewa}

Przenieść do sekcji o metodach, o których nie będziemy mówili. Takich jak również, Dział -- pracownik.

\section{Metoda klucza obcego do rodzica}

\section{Metoda zagnieżdzonych zbiorów}

\section{Metoda łączona}

\section{Metoda pełnych ścieżek}

\section{Metoda drzew prefiksowych (trie)}

\section{Metody specyficzne dla bazy danych}
\subsection{PostgreSQL \texttt{ltree}}
\subsection{Oracle \texttt{connect by}}
\subsection{DB2 \texttt{with}}
Opis standardowej metody with która pojawiła się w standardzie SQL??. Porównanie z connect by

\chapter{Porównanie wydajności}
\section{Po czym porównywać?}
\subsection{Metodzie}
\subsection{Bazie danych}



Dane użyte podczas testów zostały wygenerowane losowo w sposób automatyczny. Program który zajmuje się tym zadaniem został dołączony do tej pracy.

\chapter{Możliwości korzystania z danych hierarchicznych w O/RM}
\section{Active Record}
\section{Hibernate}
\subsection{HQL}

Ten akapit opisuje HQL\index{HQL}.

\chapter{Zastosowania i przykłady}

\section{LDAP}
\section{DNS?}
\section{XML}
\subsection{Różnice pomiędzy XML a drzewami}
\begin{itemize}
 \item kolejność jest istotna
 \item język XPath i ścieżki w dokumencie
\end{itemize}

\appendix

\chapter{Opis Python DB API}

\chapter{Załączone programy}

\chapter{Metodyka testów}
\section{Wybrane SZDB}
\begin{itemize}
 \item PostgreSQL\index{PostgreSQL}
 \item MySQL\index{MySQL}
 \item SQLite
 \item Firebirld?
 \item Oracle
 \item DB2
 \item MS SQL Server
\end{itemize}


Wybrane zostały SZDB spełniające następujące cechy:
\begin{itemize}
 \item popularne
 \item darmowe do użytku domowego 
\end{itemize}


Dla porównania innych rozwiązań zostały dobrane następujące bazy danych:

\begin{itemize}
 \item db4o
 \item berkeley DB
 \item ??
\end{itemize}

\section{Dane testowe}

Aby przetestować przedstawione rozwiązania wykożystane zostały następujące zestawy danych:
\begin{itemize}
 \item mały test 100 dwupoziomowo
 \item test mocnego zagłębienia
 \item test małego zagłebienia
 \item test 6 pozimów po 3 zagłębienia 
\end{itemize}


Parametry generatora danych:
\begin{itemize}
 \item minimalny poziom zagłębienia
 \item maksymalny poziom zagłębienia
 \item rozklad prawdopodobieństwa wygenerowania dzieci (\cite{asdf}, \cite[Ala]{asdf})
\end{itemize}


%\bibliographystyle{plain}
%\bibliography{main}

\clearpage
\phantomsection
\addcontentsline{toc}{chapter}{Literatura}
\begin{thebibliography}{99}
\thispagestyle{empty}

\bibitem[Cor04]{cormen} Thomas H. Cormen, Charles E. Leiserson, Ronald L. Rivest, Clifford Stein, 
    \emph{Wprowadzenie do algorytmów},
    Wydawnictwa Naukowo-Techniczne, Warszawa 2004

\bibitem[Dro04]{drozdek} Adam Drozdek
    \emph{C++. Algorytmy i struktury danych},
    Helion, Gliwice 2004

\bibitem[Knu02]{knuth} Donald E. Knuth
    \emph{Sztuka programowania. Tomy 1-3},
    Wydawnictwa Naukowo-Techniczne, Warszawa 2002

\bibitem[Cel00]{calko-sql} Joe Celko,
    \emph{SQL. Zaawansowane techniki programowania},
    MIKOM, 2000

\bibitem[Cel04]{celko-tree} Joe Celko,
    \emph{Joe Celko`s Trees and Hierarchies in SQL for Smarties}, 
    Morgan Kaufmann, 2004

\bibitem[Wal08]{apress-sqlserver} Robert E. Walters, Michael Coles, Robert Rae, Fabio Ferracchiati, Donald Farmer
    \emph{Accelerated SQL Server 2008},
    Apress, 2008

% \bibitem[]{} 
%     \emph{},
%     
% \bibitem[]{} 
%     \emph{},
%     

\end{thebibliography} 





%\listoffigures
%\listoftables

\clearpage
\phantomsection
\addcontentsline{toc}{chapter}{\indexname}
\printindex


\end{document}
