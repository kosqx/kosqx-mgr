\documentclass[11pt,a4paper,oneside]{report}
%\documentclass[10pt,a4paper,oneside]{book}
%\documentclass[10pt,a4paper,draft]{book}
%\documentclass[sfheadings,a4paper,11pt]{mwbk}

\usepackage{polski}
\usepackage[utf8]{inputenc}

\usepackage{ifpdf}

\usepackage{geometry}
\geometry{verbose,a4paper,tmargin=2.5cm,bmargin=2.5cm,lmargin=3.5cm,rmargin=2.5cm}

% \newif\ifpdf
% \ifx\pdfoutput\undefined
%    \pdffalse
% \else
%    \pdfoutput=1
%    \pdftrue
% \fi

\ifpdf
   \usepackage[pdftex]{graphics}
   \usepackage[pdftex]{color}
   \pdfcompresslevel9
\else
   \usepackage{graphics}
   \usepackage{color}
\fi

\usepackage{indentfirst}
\sloppy
\clubpenalty = 10000
\widowpenalty = 10000
\frenchspacing

% dla łatwiejszego sprawdzania
%\linespread{1.3}
\linespread{1.0}
\setcounter{tocdepth}{1}

\usepackage{amsmath}
\usepackage{amsfonts}
\usepackage{amssymb}
\usepackage{makeidx}

%\usepackage{showidx}
\usepackage{makeidx}
\makeindex

\author{Krzysztof Kosyl}
\title{Przechowywanie danych hierarchicznych w relacyjnych bazach danych}


\newcommand{\HRule}{\rule{\linewidth}{0.5mm}}
\newcommand{\temat}[1]{\subsection*{#1}}
\newcommand{\podtemat}[1]{\subsubsection*{#1}}
\newcommand{\operacja}[1]{\subsubsection*{#1}}


% wyświetla TODO: wrazie potrzeby zmodyfikować
\newcounter{kosqxtodo}
%\newcommand{\todo}[1]{\addtocounter{kosqxtodo}{1}(\textcolor{red}{\arabic{kosqxtodo}: \textbf{#1}})}
\newcommand{\todo}[1]{}
\newcounter{kosqxask}
\newcommand{\ask}[1]{\addtocounter{kosqxask}{1}(\textcolor{blue}{\arabic{kosqxask}: \textbf{#1}})}

\newcommand{\eng}[1]{(ang. \textit{#1})}

\newenvironment{qxtab}[2]{%
\begin{table}[h]%
  \caption{#2}%
  \label{tab:#1}%
  \begin{center}%
}{%
  \end{center}%
\end{table}%
}

\newenvironment{qxfig}[2]{%
\begin{figure}[h]%
  \caption{#2}%
  \label{fig:#1}%
  \begin{center}%
}{%
  \end{center}%
\end{figure}%
}

% numery stron u góry
\pagestyle{headings}

% rotacje w nagłówku kolumny
\usepackage{rotating}

\usepackage{wrapfig}

% dla pygments // \begin{Verbatim}
\usepackage{fancyvrb}
%! pygments-style


%\usepackage[unicode=true,pdftex,backref]{hyperref}
\usepackage[unicode=true,pdftex]{hyperref}
\hypersetup{
%kolory
  colorlinks=true,
  linkcolor=black,
  citecolor=black,
  %linkcolor=blue,
  %anchorcolor=black,
  %citecolor=green,
  %filecolor=magenta,
  %menucolor=red,
  %pagecolor=red,
  %urlcolor=cyan,
%zakladki
  bookmarks=true,
  bookmarksopen=true,
  bookmarksnumbered=true,
%metadane
  pdfauthor = {Krzysztof Kosyl},
  pdftitle = {Przechowywanie danych hierarchicznych w relacyjnych bazach danych},
  pdfsubject = {Trees in relational DB},
  pdfkeywords = {tree, db},
}


%%%%%%%%%%%%%%%%%%%%%%%%%%%%%%%%%%%%%%%%%%%%%%%%%%%%%%%%%%%%%%%%%%%%%
% document

\begin{document}

% \bibliographystyle{alpha}
\bibliographystyle{plalpha}

%% desc: strona tytułowa

% To już nie jest używane
% \maketitle{}


\begin{titlepage}
    \begin{center}
        \Large\scshape Uniwersytet Mikołaja Kopernika\\
        Wydział Matematyki i Informatyki
    \end{center}
    \vspace{22ex}
    \begin{center}
        \Huge\bfseries
        Przechowywanie danych hierarchicznych w relacyjnych bazach danych
    \end{center}
    \vspace{22ex}
    \begin{center}
        \Large\bfseries
        Krzysztof Kosyl \\
        \vspace{0.5ex}
        \normalsize
        Nr albumu: 187411
    \end{center}
    \vspace{19ex}
    \hspace*{\fill}
    \parbox{0.50\textwidth}{\setlength{\parindent}{1em}
            \normalsize
            \noindent
            Praca magisterska\\
            napisana pod kierunkiem\\
            \textbf{dra hab Krzysztofa Stencla, prof. UMK}\\
            na Wydziale Matematyki i Informatyki}
    \vfill
    \centerline{\textsc{Toruń} 2009}
    \addtocounter{page}{-1}
\end{titlepage}



% \begin{titlepage}
% 	\begin{center}
% 		\textsc{Uniwersytet Mikołaja Kopernika -- Wydział Matematyki i Informatyki}
% 	\end{center}
% 
% 	\vfill
% 
% 	\begin{flushleft}
% 		Krzysztof Kosyl\\
% 		Numer albumu: 187411
% 	\end{flushleft}
% 
% 	\vfill\vfill
% 
% 	\begin{center}
% 		\HRule \\[0.4cm]
% 		\huge{\textbf{Przechowywanie danych hierarchicznych w relacyjnych bazach danych}}
% 		\HRule \\[0.4cm]
% 	\end{center}
% 
% 	\vfill\vfill\vfill\vfill
% 
% 	\begin{flushright}
% 		%Praca magisterska na kierunku:\\
% 		%Programowanie i Przetwarzanie Informacji
% 		Promotor:\\
% 		prof. dr hab Krzysztof Stencel
% 	\end{flushright}
% 
% 	\vfill
% 	\begin{center}
% 		Toruń 2008
% 	\end{center}
% \end{titlepage}

%% desc: abstract

\begin{abstract}

Celem pracy jest przedstawienie różnych metod przechowywania danych hierarchicznych w relacyjnych bazach danych.
W porównaniu szczególna uwaga została zwrócona na wydajność oraz łatwość wykonywania standardowych operacji.
Przedstawione zostały metody zarówno uniwersalne, dostępne dla każdej bazy danych, jak i specyficzne dla konkretnych SZBD.
%Ważnym elementem pracy jest ocena przydatności poszczególnych metod do współpracy z O/RM, wraz z przedstawieniem przykładowych rozwiązań.
Uzupełnieniem pracy są przykładowe implementacje zaprezentowanych metod.

\end{abstract} 

\tableofcontents{}

% \chapter*{Wstęp}
\chapter*{Wstęp}
\addcontentsline{toc}{chapter}{Wstęp}

% W ciągu ostatnich kilkunastu lat relacyjne bazy danych opanowały świat. 


%Put simply, this method is used to report, in order, the branches of a family tree. Such trees are
%encountered often—the genealogy of human families, livestock, horses; corporate management,
%company divisions, manufacturing; literature, ideas, evolution, scientific research, theory; and
%even views built upon views.



Hierarchie --- a w szczególności drzewa --- są bardzo użyteczną metodą reprezentacji danych. \todo{przepisać na polski}. 
Nic dziwnego, że z tej abstrakcji korzystają \emph{mapy myśli}, drzewa katalogów, itd.






Każda współczesna baza danych korzysta z struktur danych jakimi są drzewa. 
Indeksy są implementowanee jako \emph{B-drzewa}, \emph{R-drzewa}, itd.
% inne drzewa w bazach danych
Zapytania SQL są przetwarzane do \emph{drzew składni} \eng{AST --- Abstract Syntax Tree}.
Lecz to są drzewa zastosowane w implementacji, mające na celu zwiększenie szybkości działania bazy danych i nie są widoczne dla użytkownika bazy danych.

Lecz --- najpopularniejsze obecnie relacyjne bazy danych --- nie zostały stworzone do obsługi danych hierarchicznych.
Nie oznacza to bynajmniej, że jest to niemożliwe.
Należy za to spodziewać się pewnych utrudnień.

% Celem tej pracy jest zaprezentowanie metod przechowywania danych w relacyjnych bazach danych. 





Celem tej pracy jest:
\begin{itemize}
    \item przedstawienie popularnych metod przechowywania danych hierarchicznych
    \item porównanie tych metod pod względem łatwości użycia, dostępności, możliwości
    \item sprawdzenie ich wydajności w najpopularniejszych bazach danych
    \item ocena metod, wskazanie ich mocnych i słabych stron
\end{itemize}

Nie należy się spodziewać idealnej metody, raczej należy oczekiwać, że metody będą się znacznie różniły między sobą w poszczególnych operacjach.
%najlepszed do danego zastosowania


Jest to zagadnienie stare niemal jak same relacyjne bazy danych. 
Już Edgar Frank Codd\index{Codd Edgar Frank} zaproponował jedno z rozwiązań.
\todo{sprawdzić gdzie} 
Miało być ono obroną relacyjnego modelu danych przed zarzutami, 
że w takich bazach nie można przechowywać danych hierarchicznych.






% \chapter{Wprowadzenie do tematu}
\chapter{Wprowadzenie do tematu}


% The hierarchical data model organizes data in a tree structure. There is a hierarchy of parent and child data segments

\section{Podstawowe definicje}

Drzewa można traktować i definiować na wiele równowarznych sposobów.
Przykładowo w~teorii grafów jest to acykliczny i spójny graf.
W informatyce częściej stosuje się definicję rekurencyjną%
\footnote{
    Wynika to z częstego w informatyce wymogu by drzewo było uporządkowane.
    Taki twór formalnie nie jest grafem.
}.


\begin{quote}

\emph{Drzewo} zdefiniujemy formalnie jako zbiór $T$ jednego lub więcej elementów zwanych \emph{węzłami}, takich że:

\begin{enumerate}
 \item istnieje jeden wyróżniony węzeł zwany \treedef{korzeniem} drzewa, $root(T)$; oraz
 \item pozostałe węzły (z wyłączeniem korzenia) są podzielone na $m \geq 0$ rozłącznych zbiorów $T_{1},\ldots, T_{m}$,
	z których każdy jest drzewem. Drzewa $T_{1},\ldots, T_{m}$ nazywane są \treedef{poddrzewami} korzenia.
\end{enumerate}

Z naszej definicji wynika, że każdy węzeł drzewa jest korzeniem pewnego poddrzewa zawartego w większym drzewie.
Liczna poddrzew węzła jest nazywana \treedef{stopniem} tego węzła.
Węzeł o stopniu zero nazywamy \treedef{liściem} lub \treedef{węzłem zewnętrznym}. 
Węzeł nie będący liściem nazywamy \treedef{węzłem wewnętrznym}. 
Poziom węzła jest zdefiniowany rekurencyjnie: poziom korzenia $root(T)$ równa się zero,
a poziom każdego innego węzła jest o jeden wiekszy niż poziom korzenia w najmniejszym 
\todo{w sensie ilości węzłów} zawierającym go poddrzewie.
%
% TODO: wysokość
%



Jeśli względny porządek poddrzew $T_{1},\ldots, T_{m}$ w części (2) definicji jest istotny,
to mówimy, że drzewo jest \treedef{uporządkowane}. \todo{...}
Jeśli nie chcemy rozróżniać drzew, które różnią się jedynie kolejnością poddrzew, mówimy o drzewach \treedef{zorientowanych}.

\treedef{Lasem} nazywamy zbiór zera lub więcej rozłącznych drzew. 
Innym sposobem wyprowadzenia części (2) definicji jest stwierdzenie: \textit{węzły drzewa z wyłączeniem korzenia tworzą las}.

Abstrakcyjnie drzewa i lasy niewiele się różnią.
Jeśli usuniemy korzeń drzewa, otrzymamy las.
W drugą stronę, jeśli wszystkie drzewa w lesie uczynimy poddrzewami jednego dodatkowego węzła, to otrzymamy drzewo.
Z tego powodu słowa ,,drzewa'' i ,,lasy'' są przy nieformalnym omawianiu struktur danych używane niemal wymiennie.

\end{quote}

%\begin{quote}
%
%\emph{Drzewo} zdefiniujemy formalnie jako zbiór $T$ jednego lub więcej elementów zwanych \emph{węzłami}, takich że:
%
%\begin{enumerate}
% \item istnieje jeden wyróżniony węzeł zwany \emph{korzeniem} drzewa, $root(T)$; oraz
% \item pozostałe węzły (z wyłączeniem korzenia) są podzielone na $m \geq 0$ rozłącznych zbiorów $T_{1},\ldots, T_{m}$,
%	z których każdy jest drzewem. Drzewa $T_{1},\ldots, T_{m}$ nazywane są \emph{poddrzewami} korzenia.
%\end{enumerate}
%
%Z naszej definicji wynika, że każdy węzeł drzewa jest korzeniem pewnego poddrzewa zawartego w większym drzewie.
%Liczna poddrzew węzła jest nazywana stopniem tego węzła.
%Węzeł o stopniu zero nazywamy \emph{liściem} lub \emph{węzłem zewnętrznym}. 
%Węzeł nie będący liściem nazywamy \emph{węzłem wewnętrznym}. 
%Poziom węzła jest zdefiniowany rekurencyjnie: poziom korzenia $root(T)$ równa się zero,
%a poziom każdego innego węzła jest o jeden wiekszy niż poziom korzenia w najmniejszym 
%\todo{w sensie ilości węzłów} zawierającym go poddrzewie.
%
%
%
%Jeśli względny porządek poddrzew $T_{1},\ldots, T_{m}$ w części (2) definicji jest istotny,
%to mówimy, że drzewo jest \emph{uporządkowane}. \todo{...}
%Jeśli nie chcemy rozróżniać drzew, które różnią się jedynie kolejnością poddrzew, mówimy o drzewach \emph{zorientowanych}.
%
%\emph{Lasem} nazywamy zbiór zera lub więcej rozłącznych drzew. 
%Innym sposobem wyprowadzenia części (2) definicji jest stwierdzenie: \textit{węzły drzewa z wyłączeniem korzenia tworzą las}.
%
%Abstrakcyjnie drzewa i lasy niewiele się różnią.
%Jeśli usuniemy korzeń drzewa, otrzymamy las.
%W drugą stronę, jeśli wszystkie drzewa w lesie uczynimy poddrzewami jednego dodatkowego węzła, to otrzymamy drzewo.
%Z tego powodu słowa ,,drzewa'' i ,,lasy'' są przy nieformalnym omawianiu struktur danych używane niemal wymiennie.
%
%\end{quote}

%
%
%\section{Czym jest drzewo}
%\index{drzewo|textbf}
%Drzewo to bardzo powszechnie używane w informatyce pojęcie. W zależności od zastosowania może być różnie zdefiniowane.
%
%\paragraph{Drzewo jako graf}
%\index{graf}
%\paragraph{Drzewo jako struktura rekurencyjna}
%\index{rekurencja}
%
%
%Tu: matematyczny opis drzew, ich podstawowe własności, omówienie terminów:
%\begin{description}
%    \item
%    \item[drzewo]
%    \item[las]
%    \item[rodzic]
%    \item[przodek]
%    \item[korzeń]
%    \item[dziecko]
%    \item[potomek]
%    \item[głębokość, mini]
%\end{description}
%
%\section{Podział drzew}
%\subsection{Jednorodność}
%Warto by dodać o drzewach jednorodnych oraz niejednorodnych.
%
%\subsection{Kolejność}
%
%
%\section{Podstawowe algorytmy dla drzew}
%\subsection{Wyszukiwanie w głąb}
%\index{drzewo!wyszukiwanie!w głąb}
%\subsection{Wyszukiwanie wszerz}
%\index{drzewo!wyszukiwanie!wszerz}
%
%
%\section{Różnice pomiędzy drzewami w algorytmice a w bazach danych}
%
%Nie ma sensu coś takiego jak drzewo czerwono-czarne gdyż w drzewach w bazach danych chodzi o strukturę a nie o optymalizację czasu dostępu.
%
%\section{Tematy porównania - operacje}
%
%
%
%\subsection{Operacje}
%\subsection{Pobranie korzeni}
%\index{drzewo!korzeń}
%\subsection{Pobranie przodków}
%\index{drzewo!przodekowie}
%\subsection{Pobranie rodzica}
%\index{drzewo!rodzic}
%\subsection{Pobranie dzieci}
%\index{drzewo!dzieci}
%\subsection{Pobranie potomków}
%\index{drzewo!potomkowie}
%\subsection{Pobranie (najmłodszego) wspólnego przodka}



% \begin{description}
%   \item[pobranie rodzica] polega na pobraniu rodzica bieżącego elementu
%   \item[pobranie przodków] polega na pobraniu rodzica, rodzica rodzica aż do korzenia elementów
% \end{description}

%\section{Dostępność}
%\subsection{Mapowanie relacyjno-obiektowe}
%\subsection{W zależności od bazy danych}
%

\section{Przyjęte założenia}
\todo{więcej wypisać, zastanowić się co z tym zrobić}

\paragraph{Przechowywany jest las}
Zgodnie z definicją z poprzedniego rozdziału las jest bardzo podobny do drzewa.
Jeśli nie potrzebny jest las a drzewo to wystarczy zauważyć, że chodzi nam o las składający się z tylko jednego drzewa.\todo{przepisać}


\paragraph{Drzewa uporządkowane}
Nie wymagane jest by metoda odpowiadała za przechowywanie porządku poddrzew.
Jeśli zaistnieje taki wymóg to można dodać kolumnę przeznaczoną do przechowywania wartości odpowiedzialnej za relację porządku.
Czasem istnieje też możliwośc sortowania po istniejących już kolumnach, przykładowo nazwie kategorii.


\paragraph{Tabela zawiera tylko jedną kolumnę z danymi użytkownika} 
Bez problemu można dodać więcej kolumn do tabeli. Obecność wyłącznie kolumny \texttt{name} zwiększa czytelność przykładów.



\paragraph{Zapytania wyłącznie związane z hierarchiczną strukturą danych}
\begin{itemize*}
    \item \treedef{Pobranie korzeni}
    \item \treedef{Pobranie rodzica}
    \item \treedef{Pobranie przodków}
    \item \treedef{Pobranie dzieci}
    \item \treedef{Pobranie potomków}
\end{itemize*}




\paragraph{Przedstawione fragmenty kodu} 
W opisach metod znajdują się fragmenty kodu je realizujące operacje na nich. 
Tam gdzie to jest możliwe (czyli metoda jest dostępna ) została zaprezentowana składnia PostgreSQL\index{PostgreSQL}.
Fragmenty kodu \todo{napisać o tym że gdzieniegdzie pojawiają się \texttt{:name}} wymagające parametrów...

Część operacji nie daje się wykonać wyłącznie za pomocą jednego (lub więcej) zapytań SQL.
W takiej sytuacji fragmenty kodu zawierają kod języka \emph{Python}\index{Python}.
Został on wybrany do tego zadania ze względu na jego ekspresyjność, zwięzłość oraz wysoki poziom abstrakcji\todo{dobre sformuowanie?}.
Z powodu tych cech jest często nazywany \emph{wykonywalnym pseudokodem}.

Aby polepszyć zwięzłość fragmęty kodu korzystają z bibloteki PADA, stworzonej aby uprościć pisanie implementacji opisanych metod. 
Więcej informacji o tej biblotece znajduje się w \todo{Załączniku 2}.



\paragraph{Operacje z wykorzystaniem \texttt{id}}
Wszystkie operacje wykorzystują sztucznie utworzony klucz główny\index{klucz!główny} \texttt{id}. 



\paragraph{Terminologia anglojęzyczna} \todo{nazwać to jakoś normalnie}
Literatura dotycząca informatyki jest dzisiaj zdominowana przez publikacje w języku angielskim. Dlatego przy każdym ważnym lub nietrywialnym w tłumaczeniu terminie została podana jego angielska wersja. 



% \chapter{Podstawowe metody przechowywania danych}
\chapter{Podstawowe reprezentacje i metody}

Metody wymienione w tym rozdziale są uniwersalne. 
Daje się je zaimplementować w każdej relacyjnej bazie danych. 
Można również (przy odrobinie wysiłku) korzystać z nich również w \emph{mapowaniach obiektowo-relacyjnych}\index{OR/M}.
Nic nie stoi na przeszkodzie aby wykorzystać te metody również w obiektowych bazach danych.

% Przedstawione zostaną tu różne reprezentacje lasów.
% Czym czym są reprezentacje?
% Otórz drzewo można zozumieć na wiele różnych sposobów.
% jako graf, jako strukturę rekurecyjną
    \clearpage

	% \section{Metoda klucza obcego do rodzica}
	\section{Metoda krawędziowa}
\index{metoda!krawędziowa|(textbf}


% Hierarchies like trees, organizational charts, ... are sometimes difficult to
% store in database tables. The most common database pattern to store 
% hierachical data is known as the adjacency model. It has been introduced 
% by the famous computer scientist Edgar Frank Codd. It's called the "adjacency"
% model because a reference to the parent data is stored in the same row as the
% child data, in an adjacent column. These kind of tables are also called self 
% referencing tables.
% http://www.scip.be/index.php?Page=ArticlesNET18


% It was Scott's pet cat called "tiger".
% And, who was Scott? His first name was Scott but Bruce. Bruce Scott was
% employee number #4 at the then Software Development Laboratories that
% eventually became Oracle. He co-authored and co-architected Oracle V1, V2 & V3.
% http://www.dba-oracle.com/t_scott_tiger.htm

% http://pl.wikipedia.org/wiki/Reprezentacja_grafu#Reprezentacja_przez_listy_s.C4.85siedztwa



Najprostszą, najbardziej intuicyjną i zapewne najpopularniejszą metodą jest metoda krawędziowa \eng{adjacency}\index{adjacency}.
Została ona pierwszy raz zaprezentowana przez Edgara Franka Codda\index{Codd Edgar Frank}.
%On też nadał im nazwę, która odnosi się do tego, że informacja o rodzicu elementu znajduje się w tej samej krotce\todo{krotka?} co dane.
%Nadał też jej nazwę. Wynika ona z tego, że każdy rekord zawiera własny identyfikator oraz identyfikator rodzica.
%Takie dane są wystarczające aby w drzewie opisać krawędz.


Zasługa spopularyzowania tej metody przypada Oracle.
Dołączył on do swojego produktu przykładową bazę danych, nazywaną ,,Scott/Tiger'' 
\footnote{
    Nazwa tej bazy danych pochodzi od metody autoryzacji w bazie Oracle (login/hasło).
    Login pochodził z nazwiska jednego z pierwszych pracowników 
    Software Development Laboratories (przekształconych ostatecznie w Oracle)
    % Bruce'a Scott'a.
    Bruca Scotta. 
    Natomiast hasło to imię jego kota.
}
korzystającą z tej metody.

Na popularność metody przekłada się również jej znaczące podobieństwo do 
używanego między innymi w~językach C i~C++ sposobu przechowywania list i drzew.
Mianowicie każdy węzeł zawiera wskaźnik na rodzica.

\begin{verbatim}[c]
typedef struct item {
    struct item* parent;
    char*        name;
} treeitem;
\end{verbatim}

W relacyjnych bazach danych odpowiednikiem wskaźnika jest klucz obcy%
%\index{klucz!obcy}%
. 


Ta metoda jest na tyle popularna, że w bazach danych pojawiły się specjalne konstrukcje do jej obsługi. 
Zostaną one przedstawione w rozdziale \todo{wstawić jakoś nazwę rozdziału} o konstrukcjach języka specyficznych dla systemu zarządzania bazą danych.


%\temat{Reprezentacja}

\begin{verbatim}[table] adjacency
>n1 _
id | parent | name
1  | NULL   | Bazy Danych

>n2 n1
id | parent | name
2  | 1      | Obiektowe

>n3 n2
id | parent | name
3  | 2      | db4o

>n4 n1
id | parent | name
4  | 1      | Relacyjne

>n5 n4
id | parent | name
5  | 4      | Open Source

>n7 n5
id | parent | name
6  | 5      | PostgreSQL

>n8 n5
id | parent | name
7  | 5      | MySQL

>n9 n5
id | parent | name
8  | 5      | SQLite

>n6 n4
id | parent | name
9  | 4      | Komercyjne

>n10 n1
id | parent | name
10 | 1      | XML

\end{verbatim}


\operacja{Reprezentacja w SQL}
%\vspace{-0.5cm}
%! method-sql simple.create

Należy zwrócić uwagę na \todo{term}warunek \texttt{ON DELETE CASCADE}. 
Sprawia on, że w razie usunięcia węzła \todo{ustalić terminologię} 
automatycznie zostaną usunięci wszyscy jego potomkowie.


\operacja{Wstawianie danych}

Ta metoda --- jako jedyna w tym rozdziale --- nie wymaga żadnego wstępnego przetwarzania danych.
W efekcie wstawianie węzłów jest bardzo proste.

%! method-sql simple.insert



\operacja{Pobranie korzeni}
Cechą charakterystyczną korzenia jest to, że jego rodzic jest ustawiony na \texttt{NULL}.
Więc pobranie rodziców sprowadza się do tego zapytania:
%! method-sql simple.roots


\operacja{Pobranie rodzica}
\todo{a może by zrobić to złączeniem?}

Identyfikator rodzica znajduje się w każdym węźle.
%Ponieważ przyjęty interface 
Więc zapytanie musi pobrać najpierw rekord o podanym jako parametr identyfikatorze a następnie
--- korzystając z tego identyfikatora rodzica --- wynikowy rekord.
Użyte został SQL z podzapytaniem, lecz równie dobrze można by było zastosować złączenie. 

%! method-sql simple.parent


\operacja{Pobranie dzieci}

Pobieranie dzieci jest bardzo prostą i szybką operacją
--- każdy rekord zawiera identyfikator rodzica.

%! method-sql simple.children


\operacja{Pobranie przodków}
Tu pojawia się po raz pierwszy największa wada tej reprezentacji --- 
dla kilku operacji wymaga wykonania wielu zapytań w bazie danych.

W tym przypadku widać, że pobranie przodków sprowadza się do pobrania aktualnego węzła,
a następnie w kolejnym zapytaniu jego rodzica,
potem rodzica jego rodzica\ldots
aż dotrze się do korzenia drzewa.

%! method-python simple.ancestors



\operacja{Pobieranie potomków}

Algorytm pobieranie potomków jest podobny do pobierania przodków --- też pobiera się po jednym poziomie.
%Różnica polega na tym, że w tym przypadku poziom do więcej niż jeden rekord, a 

%! method-python simple.descendants

Należy zwrócić uwagę na równoczesne pobieranie wszystkich węzłów na danym poziomie.
Wybrany algorytm wymaga wykonania tylko tylu zapytań ile wynosi wysokość poddrzewa,
którego korzeniem jest dany element.


Gdyby zastosować naiwny algorytm pobierania dzieci jednego węzła,
a następnie, rekurencyjnie, dzieci jego dzieci
to wymagał by on wykonania w sumie tylu zapytań ile węzłów ma poddrzewo,
którego korzeniem jest dany element.


\operacja{Uwagi}

Czasem nie ma potrzeby pobierać wszystkich potomków a tylko tych różniących się poziomem o nie więcej niż $N$.
Czyli dla $N = 1$ dzieci, dla $N = 2$, dzieci oraz wnuki, itd.
Mając taką dodatkową wiedzę program można wygenerować bardziej optymalne zapytanie.
Zasada działania się nie zmienia (dalej jest to przeszukiwanie wszerz)
ale zamiast łączyć wyniki w wnętrzu programu są one łączone wewnątrz bazy danych za pomocą operatora \texttt{UNION ALL}.

Poniżej przykład dla $N = 3$.

\begin{verbatim}[sql]
    SELECT * FROM simple WHERE parent = :id
UNION ALL 
    SELECT * FROM simple WHERE parent IN (
        SELECT id FROM simple WHERE parent = :id
    )
UNION ALL
    SELECT * FROM simple WHERE parent IN (
        SELECT id FROM simple WHERE parent IN (
            SELECT id FROM simple WHERE parent = :id
        )
    )
\end{verbatim}

Podejście to może być bardzo przydatne w drzewach o ustalonej, małej wysokości.
W takiej sytuacji może ono zastąpić ogólny algorytm.

W analogiczny sposób można pobierać przodków.


%SELECT * FROM [kosqx].[dbo].[simple] WHERE parent=1
%UNION ALL 
%SELECT * FROM [kosqx].[dbo].[simple] WHERE parent IN (SELECT id FROM [kosqx].[dbo].[simple] WHERE parent=1)
%UNION ALL
%SELECT * FROM [kosqx].[dbo].[simple] WHERE parent IN (SELECT id FROM [kosqx].[dbo].[simple] WHERE parent IN (SELECT id FROM [kosqx].[dbo].[simple] WHERE parent=1))





%Należy zwrócić uwagę na wykorzystanie operatora \texttt{IN}. Bez niego wydajność metody spada znacząco.

%Należy zwrócić uwagę na równoczesne pobieranie wszystkich węzłów na danym poziomie.
%Bez niego wydajność metody spada znacząco.
%
%Gdyby zastosować naiwny algorytm pobierania dzieci jednego węzła,
%a następnie, rekurencyjnie, dzieci jego dzieci
%%to wymagał by on wykonania w sumie tylu zapytań ile potomków ma dany element plus jeden.
%to wymagał by on wykonania w sumie tylu zapytań ile węzłów ma poddrzewo,
%którego korzeniem jest dany element.
%
%Wybrany algorytm wymaga wykonania tylko tylu zapytań ile wynosi wysokość poddrzewa,
%którego korzeniem jest dany element.

\temat{Wydajność}

\begin{qxtab}{simple}{Wydajność reprezentacji krawędziowej}
%! result-table simple deep3
\end{qxtab}

\begin{qxfig}{simple}{Wydajność reprezentacji krawędziowej}
%! result-chart simple deep3
\end{qxfig}

\index{metoda!krawędziowa|)}

	\clearpage

	% \section{Metoda zagnieżdzonych zbiorów}
	\section{Metoda zagnieżdzonych zbiorów}
\index{metoda!zagnieżdzonych zbiorów|(textbf}

Reprezentacja została spopularyzowana przez Joe Celko\cite{celko-sql}\index{Celko Joe}.
Dlatego czasem bywa nazywana ,,metodą Celko''.
Lecz to nie on jako pierwszy ją opisał.
Przed nim zrobił to Donald Knuth\cite{knuth}.



%Nazwa tej metody pochodzi od 
%Drzewo można 
%
%W tej reprezentacji

W tej metodzie każdy węzeł drzewa traktowany jest jako zbiór zbiorem.
Jeśli nie ma podzbiorów to jest liściem.
Natomiast jeśli ma podzbiory to staje się ich przodkiem.
Jeśli dodatkowo jest najmniejszym z nadzbiorów danego węzła to jest jego rodzicem.
Dodatkowym wymogiem jest to by każdy węzeł miał część wspólną tylko i~wyłącznie z swymi przodkami oraz potomkami (a~nie z~rodzeństwem).
W efekcie każdy zbiór jest \emph{,,zagnieżdzony''} w swoim rodzicu%
\footnote{
    Za przykład z realnego świata może posłużyć metoda pakowania przedmiotów w celu przewozu.
    W kontenerach zanajdują się palety, na których znajdują się kartony, wewnątrz których znajdują się opakowania indywidualne produktów.
    Innym przykładem są matrioszki
}.
Stąd pochodzi nazwa tej reprezentacji.


Ponieważ \texttt{left} oraz \texttt{right} są słowami kluczowymi SQL-92,
w~kodzie zostananą zastosowane ich skróty, odpowiednio \texttt{lft} oraz \texttt{rgt}.

Reprezentacja pozwala na przechowywanie drzew uporządkowanych.

\temat{Reprezentacja}

\begin{verbatim}[table] nested
>n1 _
id | lft | rgt | name
1  | 1   | 20  | Bazy Danych

>n2 n1
id | lft | rgt | name
2  | 2   | 5   | Obiektowe

>n3 n2
id | lft | rgt | name
3  | 3   | 4   | db4o

>n4 n1
id | lft | rgt | name
4  | 6   | 17  | Relacyjne

>n5 n4
id | lft | rgt | name
5  | 7   | 14  | Open Source

>n7 n5
id | lft | rgt | name
6  | 8   | 9   | PostgreSQL

>n8 n5
id | lft | rgt | name
7  | 10  | 11  | MySQL

>n9 n5
id | lft | rgt | name
8  | 12  | 13  | SQLite

>n6 n4
id | lft | rgt | name
9  | 15  | 16  | Komercyjne

>n10 n1
id | lft | rgt | name
10 | 18  | 19  | XML

\end{verbatim}

\temat{Operacje}

\operacja{Reprezentacja w SQL}
%! method-sql nested.create

\operacja{Wstawianie danych}
%! method-python nested.insert



% \begin{verbatim}[sql]
% INSERT INTO simple (parent, name) VALUES (:parent, :name)
% \end{verbatim}

Wstawianie danych w tej metodzie jest jej najsłabszą stroną.
\todo{terminologia}
Wymaga ono zmiany wartości \texttt{lft} i \texttt{rgt} wielu rekordów.

W powyższym przypadku
--- jako że wstawiamy dane w kolejności \emph{in depth} ---
wymagana jest modyfikacja tylko tylu rekordów na jakim poziomie\todo{?} aktualnie wstawiamy nowy węzeł.
Lecz należy sie liczyć z przypadkiem pesymistycznym --- dodaniem do korzenia potomka jako pierwszego elementu\todo{las}.
W takiej sytuacji wymagane jest zmodyfikowanie wszystkich istniejących rekordów.

\textbf{Uwaga:} Jeśli chce się jednorazowo załadować cały las do bazy danych można postąpić bardziej optymalnie\footnote{
    Powyższy kod tego nie robi gdyż ma być ogólny.
}.
Mianowicie przed załadowaniem danych do bazy danych wstępnie przetworzyć dane.
W takiej sytuacji należy sie posłużyć algorytmem \emph{wyszukiwania w głąb}
i~kolejno numerować wartości \texttt{lft} węzła po wejściu do niego oraz \texttt{rgt} przed jego opuszczeniem.
Przykładowa implementacja tego algorytmu:

\begin{verbatim}[python]
def preprocess(node):
  node.lft = get_next_int()
  for n in children(node):
    preprocess(n)
  node.rgt = get_next_int()
\end{verbatim}

\operacja{Pobranie korzeni}
%! method-python nested.roots

\operacja{Pobranie rodzica}
%! method-sql nested.parent

\operacja{Pobranie dzieci}
%! method-sql nested.children

\operacja{Pobranie przodków}
%! method-sql nested.ancestors

\operacja{Pobieranie potomków}
%! method-sql nested.descendants


\subsection{Uwagi}

Metoda umożliwia bardzo proste pobieranie liści drzewa. 
Jest ono możliwe dzięki temu, że $a.right - a.left = 2 * |a.descendants()| + 1$. 
Skoro liść nie posiada żadnych potomków to da różnica wynosi $1$, co prowadzi do zapytania: 
\begin{verbatim}[sql]
SELECT *
  FROM nested_sets
  WHERE lft + 1 = rgt
\end{verbatim}


\temat{Wydajność}

\begin{qxtab}{nested}{Wydajność reprezentacji zagnieżdzonych zbiorów}
%! result-table nested deep3
\end{qxtab}

\begin{qxfig}{nested}{Wydajność reprezentacji zagnieżdzonych zbiorów}
%! result-chart nested deep3
\end{qxfig}

\index{metoda!zagnieżdzonych zbiorów|)}

	\clearpage

	% \section{Metoda pełnych ścieżek}
	\section{Metoda pełnych ścieżek}
\index{metoda!pełnych ścieżek|(textbf}
% http://troels.arvin.dk/db/rdbms/links/#hierarchical
% 
% http://en.wikipedia.org/wiki/Transitive_closure
% http://pl.wikipedia.org/wiki/Domknięcie_przechodnie

% Domknięcie_przechodnie Cormen:644

% Reprezentacja ta opiera się na

% Ta metoda

Ta reprezentacja jest najmniej rozpowszechniona spośród zaprezentowanych w tym rozdziale.
Przy czym w Polsce jest bardziej popularna niż poza granicami kraju.
Jest to zasługa \emph{Huberta Lubaczewskiego}\index{Lubaczewski Hubert} lepiej znanego pod pseudonimem \emph{depesz}\index{depesz|see{Lubaczewski Hubert}}.
Stąd w polskim internecie często można spotkać się z tą metodą pod nazwą \emph{metoda depesza}.
On sam nazywa tę reprezentację \emph{metodą pełnych ścieżek}.
% na świecie można się spotkać z nazwami odwołującymi się do Domknięcie_przechodnie \emg{Transitive closure}

%Metoda został spopularyzowana w Polskim Internecie przez \emph{Huberta Lubaczewskiego}\index{Lubaczewski Hubert} lepiej znanego pod pseudonimem \emph{depesz}\index{depesz|see{Lubaczewski Hubert}}.

\todo{napisać czym moja implementacja różni się od implementacji Depesza}

\index{drzewo!domknięcie przechodnie}

Idea metody jest prosta. 
Głowna tabela z danymi nie zawiera żadnych informacji o hierarchii danych. 
Jedynym wymogiem jest istnienie w niej klucza głównego\index{klucz!główny}.

Cała informacja potrzebna do operowania na drzewie zawiera się w dodatkowej tabeli. 
Zawiera ona informacje o odległości pomiędzy każdym elementem a wszystkimi jego potomkami.
Ilość wierszy w tej tabeli wynosi
\begin{displaymath}
    \sum_{t \in T} level(t) + 2 \leq |T|(height(T) + 2)
\end{displaymath}
Jak widać jest to znacząca ilości rekordów, które trzeba wstawić i przechowywać by móc korzystać z tej metody.
Dlatego 

Poniższy graf pokazuje wyłącznie zawartość dodatkowej tabeli zawierającej strukturę drzewa.

%\temat{Reprezentacja}

\begin{verbatim}[table] full

>n1 _
top_id | bottom_id | distance
NULL   | 1         | 0
1      | 1         | 0

>n2 n1
top_id | bottom_id | distance
NULL   | 2         | 1
1      | 2         | 1
2      | 2         | 0

>n3 n2
top_id | bottom_id | distance
NULL   | 3         | 2
1      | 3         | 2
2      | 3         | 1
3      | 3         | 0

>n4 n1
top_id | bottom_id | distance
NULL   | 4         | 1
1      | 4         | 1
4      | 4         | 0

>n5 n4
top_id | bottom_id | distance
NULL   | 5         | 2
1      | 5         | 2
4      | 5         | 1
5      | 5         | 0


>n6 n5
top_id | bottom_id | distance
NULL   | 6         | 3
1      | 6         | 3
4      | 6         | 2
5      | 6         | 1
6      | 6         | 0

>n7 n5
top_id | bottom_id | distance
NULL   | 7         | 3
1      | 7         | 3
4      | 7         | 2
5      | 7         | 1
7      | 7         | 0

>n8 n5
top_id | bottom_id | distance
NULL   | 8         | 3
1      | 8         | 3
4      | 8         | 2
5      | 8         | 1
8      | 8         | 0

>n9 n4
top_id | bottom_id | distance
NULL   | 9         | 2
1      | 9         | 2
4      | 9         | 1
9      | 9         | 0

>n10 n1
top_id | bottom_id | distance
NULL   | 10        | 1
1      | 10        | 1
10     | 10        | 0

\end{verbatim}


\temat{Operacje}

\operacja{Reprezentacja w SQL}
%! method-sql full.create

\operacja{Wstawianie danych}
% %! method-sql full.insert

\operacja{Pobranie korzeni}
%! method-sql full.roots

\operacja{Pobranie rodzica}
%! method-sql full.parent

\operacja{Pobranie dzieci}
%! method-sql full.children

\operacja{Pobranie przodków}
%! method-sql full.ancestors

\operacja{Pobieranie potomków}
%! method-sql full.descendants


\subsection{Uwagi}

Ta metoda jest bardzo elastyczna, pozwala łatwo wykonywać bardzo bardzo różnorodne zapytania.


\begin{verbatim}[sql]
SELECT *
  FROM none
\end{verbatim}


\temat{Wydajność}

\begin{qxtab}{full}{Wydajność reprezentacji pełnych ścieżek}
%! result-table full deep3
\end{qxtab}

\begin{qxfig}{full}{Wydajność reprezentacji pełnych ścieżek}
%! result-chart full deep3
\end{qxfig}





\index{metoda!pełnych ścieżek|)}



	\clearpage
	
	% \section{Metoda drzew prefiksowych (trie)}
	\section{Metoda zmaterializowanych ścierzek}
\index{metoda!drzew prefiksowych|textbf}\index{metoda!lineage|see{drzew prefiksowych}}

Metoda ta polega na przechowywaniu ciągu identyfikatorów wszystkich węzłów pomiędzy korzeniem a danym węzłem.
Są one przechowywane w pojedyńczym polu danego rekordu. 
Identyfikatorem może być dowolne unikalne pole\todo{term:pole?} tabeli.
Dobrym rozwiązaniem jest numeryczny klucz główny, lecz często bywa też stosowana unikalna nazwa węzła.


Ten opis nie wymusza konkretnej implementacji. 
Listę elementów można przechowywać na wiele sposobów. 
Przykładowo ciąg identyfikatorów całkowitoliczbowych \texttt{4, 8, 15, 16, 23, 42} można przechować jako:
\begin{itemize}
 \item tablicę\todo{term:array} np. \verb|{4, 8, 15, 16, 23, 42}|
% wady: dostpne tylko w wybranych bazach danych, niedostpne dla ORM, potencjalne problemy z indeksami
 \item napis z elememtami rozdzielonymi separatorami. 
    Przykładowo dla separatora \verb|.| będzie to napis \verb|'4.8.15.16.23.42'|. 
    Ważne jest aby znak (lub znaki) będące separatorem nie mogły występować w identyfikatorze.
 \item napis z elememtami przekształconymi na napis stałej długości. 
    Przykładowo dla długości 3 będzie to \verb|'004008015016023042'|. 
    Wadą tej metody jest konieczność jednorazowego długości napisu dla identyfikatora.
    Dla długości $n$ można przechować tylko $10^n-1$ węzłów\footnote{zakładając, że wszystkie dopuszczalne wartości znajdą się w użyciu, a usuwanie elementow tworzy luki}.
    Natomiast zastosowanie dużej $n$ zwiększa zużycie zasobów SZDB, oraz spowalnia działanie.
\end{itemize}


Metoda jest też zwana Lineage.

\operacja{Reprezentacja w SQL}
%! method-sql pathenum.create

\operacja{Wstawianie danych}
%! method-sql pathenum.insert

\operacja{Pobranie korzeni}
%! method-sql pathenum.roots

\operacja{Pobranie rodzica}
% %! method-sql pathenum.parent

\operacja{Pobranie dzieci}
% %! method-sql pathenum.children

\operacja{Pobranie przodków}
%! method-sql pathenum.ancestors

\operacja{Pobieranie potomków}
%! method-sql pathenum.descendants
	\clearpage

    % \section{Metoda zmaterializowanych krawędzi}
\index{metoda!zmaterializowanych krawędzi|(textbf}

Metoda ta polega na przechowywaniu ciągu numerów porządkowych wszystkich węzłów pomiędzy korzeniem a danym węzłem.

% Podobieństwo do drzew katalogów gdzie 'ala' nic nie znaczy, za to znaczenie ma cała ścieżka
% tzn
% Podobne jest też do wyboru ŧrasy w mieście gdzie mówimy najpierw w lewo, potem w prawo, prosto i wprawo i jesteśmu
% Metoda ta ma szansę być o wiele bardziej zwięzła, 

\operacja{Reprezentacja}

\begin{verbatim}[table] edgeenum
>n1 _
id | path     | name
1  | /1       | Bazy Danych

>n2 n1
id | path     | name
2  | /1/1     | Obiektowe

>n3 n2
id | path     | name
3  | /1/1/1   | db4o

>n4 n1
id | path     | name
4  | /1/2     | Relacyjne

>n5 n4
id | path     | name
5  | /1/2/1   | Open Source

>n7 n5
id | path     | name
6  | /1/2/1/1 | PostgreSQL

>n8 n5
id | path     | name
7  | /1/2/1/2 | MySQL

>n9 n5
id | path     | name
8  | /1/2/1/3 | SQLite

>n6 n4
id | path     | name
9  | /1/2/2   | Komercyjne

>n10 n1
id | path     | name
10 | /1/3     | XML

\end{verbatim}


\operacja{Reprezentacja w SQL}
%! method-sql edgeenum.create

\operacja{Wstawianie danych}
%! method-sql edgeenum.insert

\operacja{Pobranie korzeni}
%! method-sql edgeenum.roots

\operacja{Pobranie rodzica}
%! method-sql edgeenum.parent

\operacja{Pobranie dzieci}
%! method-sql edgeenum.children

\operacja{Pobranie przodków}
%! method-sql edgeenum.ancestors

\operacja{Pobieranie potomków}
%! method-sql edgeenum.descendants


\temat{Wyniki}

% \begin{table}[h!]
%   \caption{Wyniki reprezentacji krawędziowej}
%   \begin{center}
% %! result-table pathenum deep3
%   \end{center}
% \end{table}
% 
% \begin{figure}[h!t]
%   \caption{Wyniki reprezentacji krawędziowej}
%   \label{fig:img_chart_simple}
%   \begin{center}
% %! result-chart pathenum deep3
%   \end{center}
% \end{figure}


\index{metoda!zmaterializowanych krawędzi|)}
    % \clearpage

	% drzewa binarne
	% http://commons.wikimedia.org/wiki/File:Binary_tree_in_array.svg
	% http://en.wikipedia.org/wiki/K-ary_tree


% \chapter{Modyfikacje metod przechowywania danych}
% 	% \section{Metoda łączona}


% \chapter{Metody specyficzne dla bazy danych}
\chapter{Metody specyficzne dla bazy danych - konstrukcje języka}
%     Opisane w tym rozdziale metody nie są uniwersalnymi reprezentacjami.

% \todo{Przepisać, bo stylistycznie to masakra}
% 
% Metody PL/SQL, CTEs, connect by korzystają z reprezentacji krawędziowej.
% Metoda krawędziowa posiada wiele zalet lecz ma też znaczącą wadę - 
% SQL (sciślej piszącL te jego dialetaki jaki są dostępny w wszystkich bazach danych)
% nie pozwala na pobranie potomków w jednym zapytaniu. 
% Podobnnie jest z przodkami. 
% Natomiast niektre bazy danych posiadają mechanizmy umożliwjące na wygodne wykonywanie zapytań rekurencyjnych.
% 
% 
% Pozostałe metody też korzystają z omówionych poprzednio reprezentacji.
% Oferują za to mechanizmy ułatwiające operowanie tymi reprezentacjami, 
% jak też optymalizacje zwiększające wydajność i zmniejszające zrozmiar pola.

Opisana w poprzednim rozdziale metoda krawędziowa ma szereg zalet. 
Niestety ma też powarzną wadę - aby pobrać przodków lub potomków należy wykonać wiele zapytań.
W praktyce oznacza to większą czasochłonność takiej operacji. 
Ponadto wymaga bardziej skomplikowanego kodu.

Obejściem tego problemu jest pełniejsze wykorzystanie moliwości dawanych przez bazy danych.
Nie są ona zwykłymi składami danych, lecz udostępniają coraz większe możliwości operowania na danych.


    \clearpage

	% \section{PL/SQL}
    \section{PL/SQL}
\index{metoda!PL/SQL|(textbf}

\temat{Opis PL/SQL}



\temat{Operacje w PostgreSQL}

\operacja{Utworzenie funkcji pobierającej przodków}

%! method-sql plsql.postgresql_ancestors_create

\operacja{Utworzenie funkcji pobierającej potomków}

%! method-sql plsql.postgresql_descendants_create

\operacja{Pobranie przodków}
%! method-sql plsql.postgresql_ancestors

\operacja{Pobieranie potomków}
%! method-sql plsql.postgresql_descendants




\temat{Operacje w Oracle}

\operacja{Utworzenie funkcji pobierającej przodków}

%! method-sql plsql.oracle_ancestors_create

\operacja{Utworzenie funkcji pobierającej potomków}

%! method-sql plsql.oracle_descendants_create

\operacja{Pobranie przodków}
%! method-sql plsql.oracle_ancestors

\operacja{Pobieranie potomków}
%! method-sql plsql.oracle_descendants




\temat{Operacje w SQLServer}

\operacja{Utworzenie funkcji pobierającej przodków}

%! method-sql plsql.sqlserver_ancestors_create

\operacja{Utworzenie funkcji pobierającej potomków}

%! method-sql plsql.sqlserver_descendants_create

\operacja{Pobranie przodków}
%! method-sql plsql.sqlserver_ancestors

\operacja{Pobieranie potomków}
%! method-sql plsql.sqlserver_descendants





%\operacja{Utworzenie funkcji}






%\operacja{Pobranie przodków}
%%! method-sql plsql.ancestors

%\begin{verbatim}[sql]
%SELECT * FROM tree_ancestors(111);
%SELECT t FROM tree_ancestors(111) AS t;
%SELECT count(*) FROM tree_ancestors(111);
%SELECT id, parent, value
%  FROM tree_ancestors(111) AS t
%    JOIN tree ON t = tree.id;
%\end{verbatim}


%\operacja{Pobieranie potomków}
%%! method-sql plsql.descendants

%\begin{verbatim}[sql]
%SELECT * FROM tree_descendants(1);
%SELECT t FROM tree_descendants(1) AS t;
%SELECT count(*) FROM tree_descendants(1);
%SELECT id, parent, value
%  FROM tree_descendants(1) AS t 
%    JOIN tree ON t = tree.id;
%\end{verbatim}


\temat{Wydajność}

\begin{qxtab}{plsql}{Wydajność metody PL/SQL}
%! result-table plsql deep3
\end{qxtab}

\begin{qxfig}{plsql}{Wydajność metody PL/SQL}
%! result-chart plsql deep3
\end{qxfig}

\operacja{Uwagi}






\index{metoda!PL/SQL|)}


	\clearpage

	% \section{IBM DB2 \texttt{with}}
	\section{Wspólne Wyrażenia Tabelowe}
\index{metoda!Wspolnych@Wspólnych Wyrażeń Tabelowych|(textbf}
\index{IBM DB2}\index{SQL Server}
% http://www.ibm.com/developerworks/data/library/techarticle/0307steinbach/0307steinbach.html


% http://publib.boulder.ibm.com/infocenter/dzichelp/v2r2/index.jsp?topic=/com.ibm.db29.doc.apsg/db2z_createcte.htm
% http://www.4guysfromrolla.com/webtech/071906-1.shtml - dobry opis, napisane do czego może się jeszcze przydać
% http://www.depesz.com/index.php/2008/10/07/waiting-for-84-common-table-expressions-with-queries/#more-1287

% Opis standardowej metody with która pojawiła się w standardzie SQL:1999. Porównanie z connect by
% UNION ALL (should allow UNION too), and we don't have SEARCH or CYCLE clauses
% Common Table Expressions - Wspólne wyrażenie tabelowe
% http://www.google.com/search?client=opera&rls=en&q=Wsp%C3%B3lne+wyra%C5%BCenie+tabelowe&sourceid=opera&ie=utf-8&oe=utf-8

%% TODO:
% - napisać http://en.wikipedia.org/wiki/Common_table_expressions
% - przerobić tytuły i opisy na CTE

% rekurencyjne wyrażenie tabelowe. Wspólne wyrażenie tabelowe odwołujące się do siebie w klauzuli FROM pełnej selekcji.
% Rekurencyjne wyrażenia tabelowe są używane do konstruowania zapytań rekurencyjnych.

% Common table expressions, or CTEs, are new to DB2 as of Version 8 and they greatly expand the useability of SQL. 
% A common table expression can be thought of as a named temporary table within a SQL statement that is retained for the duration of a SQL statement. 
% There can be many CTEs in a single SQL statement, but each must have a unique name and be defined only once.


% PGSQL doc:
% Tip: The recursive query evaluation algorithm produces its output in breadth-first search order.
% You can display the results in depth-first search order by making the outer query ORDER BY a
% “path” column constructed in this way.



W standardzie SQL:1999\index{SQL!SQL:1999} dodano \emph{Wspólne Wyrażenia Tabelowe} \eng{CTE --- Common Table Expressions}.
Głównym celem ich wprowadzenia było umożliwienie pisania bardziej zwięzłego, czytelnego a przede wszystkim prostszego kodu.

Ich działanie przypomina stworzenie tymczasowego widoku\index{widok}. 
%istniejącego tylko na czas wykonania danego zapytania.
Po zdefiniowaniu jest on dostępny dla zapytania.
Jednak w odróżnieniu od widoku jest on tworzony w samym zapytaniu i istnieje tylko na czas jego wykonywania.


Przykładowym zastosowaniem CTE jest uniknięcie wielokrotnego używania tego samego podzapytania\cite{apress-sqlserver}.
Nie tylko zwiększy to czytelność kodu, ale poprawi jego wydajność.
% http://www.postgresonline.com/journal/archives/127-PostgresQL-8.4-Common-Table-Expressions-CTE,-performance-improvement,-precalculated-functions-revisited.html



%Pozwala ono również na rekurencyjne wykonywanie zapytań w bazie danych.

Wspólne Wyrażenia Tabelowe mają też inne, ciekawsze z punktu widzenia tej pracy, zastosowanie.
Pozwalają one na rekurencyjne wykonywanie zapytań w bazie danych.

W chwili obecnej to rozszerzenie jest dostępne w:
\begin{itemize}
 \item \textbf{IBM DB2} począwszy od  wersji 8
 \item \textbf{Microsoft SQL Server}
	% http://www.mssqltips.com/tip.asp?tip=1520
	począwszy od SQL Server 2005
 \item \textbf{PostgreSQL}
	% http://developer.postgresql.org/pgdocs/postgres/queries-with.html
	począwszy od 8.4
% \item \textbf{Firebird}
%	wersja 2.1 
\end{itemize}

%\begin{description}
% \item[IBM DB2]
%    wprowadzone w wersji 8
% \item[Microsoft SQL Server]
%	% http://www.mssqltips.com/tip.asp?tip=1520
%	w wersjach począwszy od SQL Server 2005
% \item[PostgreSQL]
%	% http://developer.postgresql.org/pgdocs/postgres/queries-with.html
%	ma się pojawić w wersji 8.4
% \item[Firebird]
%	wersja 2.1 
%\end{description}


\temat{Opis Wspólnych Wyrażeń Tabelowych}
\index{Wspolne@Wspólne Wyrażenia Tabelowe|textbf}
\index{CTEs|see{Wspólne Wyrażenia Tabelowe}}
\index{Common Table Expressions|see{Wspólne Wyrażenia Tabelowe}}
\index{WITH@\texttt{WITH}|see{Wspólne Wyrażenia Tabelowe}}


Składnia Rekurencyjnych Wspólnych Wyrażeń Tabelowych prezentuje się następująco:


\index{WITH RECURSIVE@\texttt{WITH RECURSIVE}|see{Wspólne Wyrażenia Tabelowe}}


\begin{verbatim}[sql]
WITH <<nazwa widoku>>(<<nazwy kolumn>>) AS
(
  <<zapytanie startowe>>
UNION ALL
  <<zapytanie rekurencyjne>>
)
SELECT <<nazwy kolumn>> FROM <<nazwa widoku>>
\end{verbatim}

Działanie tej konstrukcji można streścić jako:
\begin{enumerate}
    \item
        Na początku wykonywane jest \texttt{zapytanie startowe}.
        Jego wynik jest widoczny pod nazwą \texttt{nazwa widoku}.
    \item
        Korzystając z wygenerowanych w poprzednim kroku rekordów
        (wyłącznie w ostatnim kroku, wyniki poprzednich wywołań nie są tu dostępne)
        \texttt{zapytanie rekurencyjne} zwraca nowe wyniki.
        %Jeśli nie został pobrany żaden nowy element
        Jeśli jest ich więcej niż zero to ten etap się powtarza.
    \item
        Po zakończeniu pracy części rekurencyjnej jej \emph{wszystkie} wyniki są dostępne dla zapytania znajdującego się po nim.
\end{enumerate}

Ten algorytm można przedstawić również w poniższy sposób:
\begin{verbatim}[python]
widok = []
tmp = zapytanie_startowe()
while tmp:
  widok.extend(tmp)
  tmp = zapytanie_rekurencyjne(tmp)
return widok
\end{verbatim}

Najprostszy przykład wykorzystania zapytań rekurencyjnych to wygenerowanie ciągu liczb. 
W tym przykładzie ciągu arytmetycznego zaczynającego się od $0$ (co wynika z zapytania startowego \texttt{SELECT \textbf{0} AS i}), 
różnica ciągu wynosi $1$ (\texttt{SELECT \textbf{i + 1} AS i})

\begin{verbatim}[sql]
WITH RECURSIVE a(i) AS (
    SELECT 0 AS i
  UNION ALL
    SELECT i + 1 AS i
      FROM a
      WHERE i < 10
)
SELECT i FROM a;
\end{verbatim}


%Rozszerzenie to można wykorzystać do pracy z reprezentacją krawędziową. 
%Umożliwi ono wysłanie do bazy danych tylko jednego zapytania podczas pobierania potomków i przodków. 
%Bez niego konieczne było wykonywanie odzielnego zapytania podczas przechodzenia przez każdy poziom drzewa.

\temat{Operacje}

%W bazie PostgreSQL należy urzyć \texttt{}

Niestety implementacje CTE różnią się od siebie.
PostgreSQL wymaga jawnego podania, że chodzi o zapytanie rekurencyjne czyli \texttt{WITH RECURSIVE}.
Pozostałe bazy wymagają samego \texttt{WITH} a słowo \texttt{RECURSIVE} powoduje zgłoszenie błędu.
Ponadto DB2 nie zezwala na jawne użycie złączenia (czyli z \texttt{JOIN}).
Czyli trzeba przenieść warunek złączenia do klauzuli \texttt{WHERE}.


\operacja{Pobranie przodków}
%! method-sql with.ancestors

\operacja{Pobieranie potomków}
%! method-sql with.descendants

By zapytanie zwracało potomków właściwych dodano dodatkową kolumnę \texttt{level}.
Jej wartość odpowiada poziomowi węzła w pobieranym poddrzewie.

Pomijając tą kolumnę, kod jest niemal identyczny jak w pobieraniu przodków.
Tym co je różni jest warunek złączenia.

%Warto zwrócić uwagę, że ta konstrukcja przeszukuje drzwewo wszerz. Można można to dostrzec w tym jak zapytanie jest sformuowane. 
%
%Jeśli by zaszła potrzeba pobrania całego drzewa wystarczy w zapytaniu startowym zmienić \texttt{WHERE root.id = :id} na \texttt{WHERE root.parent IS NULL}


\temat{Wydajność}

\begin{qxtab}{with}{Wydajność metody Wspólnych Wyrażeń Tabelowych}
%! result-table with
\end{qxtab}

\begin{qxfig}{with}{Wydajność metody Wspólnych Wyrażeń Tabelowych}
%! result-chart with
\end{qxfig}



\index{metoda!Wspolnych@Wspólnych Wyrażeń Tabelowych|)}





% \begin{verbatim}[sql]
% WITH temptab(deptid, empcount, superdept) AS
%    (    SELECT root.deptid, root.empcount, root.superdept
%             FROM departments root
%             WHERE deptname='Production'
%      UNION ALL
%         SELECT sub.deptid, sub.empcount, sub.superdept
%             FROM departments sub, temptab super
%             WHERE sub.superdept = super.deptid
%    )
% SELECT sum(empcount) FROM temptab
% \end{verbatim}



% P main.py sql db2 'WITH temptab(level, id, parent, name) AS (SELECT 0, r.id, r.parent, r.name FROM simple r WHERE r.id = 1 UNION ALL SELECT t.level + 1, s.id, s.parent, s.name FROM simple s, temptab t WHERE s.parent = t.id) SELECT level, id, parent, name FROM temptab'
% +-------+----+--------+-------------+
% | level | id | parent | name        |
% +-------+----+--------+-------------+
% | 0     | 1  | None   | Bazy Danych |
% | 1     | 2  | 1      | Obiektowe   |
% | 1     | 4  | 1      | Relacyjne   |
% | 1     | 10 | 1      | XML         |
% | 2     | 3  | 2      | db4o        |
% | 2     | 5  | 4      | Komercyjne  |
% | 2     | 6  | 4      | Open Source |
% | 3     | 7  | 6      | PostgreSQL  |
% | 3     | 8  | 6      | MySQL       |
% | 3     | 9  | 6      | SQLite      |
% +-------+----+--------+-------------+



% P main.py sql db2 'WITH temptab(level, id, parent, name) AS (SELECT 0, r.id, r.parent, r.name FROM simple r WHERE r.id = 7 UNION ALL SELECT t.level + 1, s.id, s.parent, s.name FROM simple s, temptab t WHERE s.id = t.parent) SELECT level, id, parent, name FROM temptab'
% +-------+----+--------+-------------+
% | level | id | parent | name        |
% +-------+----+--------+-------------+
% | 0     | 7  | 6      | PostgreSQL  |
% | 1     | 6  | 4      | Open Source |
% | 2     | 4  | 1      | Relacyjne   |
% | 3     | 1  | None   | Bazy Danych |
% +-------+----+--------+-------------+

	\clearpage

	% \section{Oracle \texttt{connect by}}
	\section{Oracle \texttt{connect by}}
\index{metoda!connect by@\texttt{connect by}|(textbf}\index{Oracle}
% napisać czy to przeszukiwanie w głąb czy wszerz

% http://download.oracle.com/docs/cd/B19306_01/server.102/b14200/queries003.htm
% http://www.dba-oracle.com/t_sql99_with_clause.htm

%% TODO:
% - napisać o sortowaniu wyników
% - operatory, zwłaszcza LEVEL, SYS_CONNECT_BY_PATH,
% - trochę o historii
% - literatura


% prior forces reporting to be from the root out toward the leaves (if the prior column is the
% parent) or from a leaf toward the root (if the prior column is the child).


System zarządzania bazą danych Oracle nie posiada możliwości przetwarzania danych hierarchicznych za pomocą klauzuli \texttt{WITH}.
\index{Wspolne@Wspólne Wyrażenia Tabelowe}
W prawdzie jest ona dostępna od wersji \emph{Oracle 9i release 2} ale służy wyłacznie do pracy z podzapytaniami.

% po ludzku: ma CTE ale tylko WITH, ale bez WITH RECURSIVE

W to miejsce \emph{Oracle} udostępnia własne rozszerzenie \texttt{CONNECT BY}. 
Jest ono dobrze opisane w \cite{oracle-ref11}, wiec tu zostaną przedstawione tylko podstawowe i najczęściej potrzebne jego możliwości.


\temat{Opis klauzuli \texttt{CONNECT BY}}

\begin{verbatim}[sql]
SELECT expression [,expression]...
    FROM [user.]table
    WHERE condition
    CONNECT BY [PRIOR] expression = [PRIOR] expression
    START WITH expression = expression
    ORDER BY expression
\end{verbatim}


Sposób działania da się w skrócie opisać regułami:
\begin{itemize}
    \item \texttt{START WITH}\index{START WITH@\texttt{START WITH}} wskazuje w którym miejscu drzewa rozpocząć działanie. 
        Klauzula może się znajdować zarówno przed jak i po \texttt{CONNECT BY}.
    \item kierunek przechodzenia po węzłach zależny jest od tego przed którym wyrażeniem stoi \texttt{PRIOR}\index{PRIOR@\texttt{PRIOR}}.
         Nie ma różnicy \verb|CONNECT BY PRIOR a = b| a \verb|CONNECT BY b = PRIOR a|.
    \item klauzula \texttt{WHERE} pozwala na wyeliminowanie z wyniku pojedyńczych rekordów, 
        ale nie usuwa rekordów, które są dostępne po przejściu przez ten pominięty rekord.
    \item zastosowanie zwykłej klauzuli \texttt{ORDER BY} niszczy hierarchiczny układ danych 
        (więc w praktyce rzadko bywa używana).
        Aby umożliwić kontrolowanie kolejności odwiedzania węzłów, 
        w \emph{Oracle 9i} wprowadzono klauzulę \texttt{ORDER SIBLINGS BY}\index{ORDER SIBLINGS BY@\texttt{ORDER SIBLINGS BY}}.
\end{itemize}


% START WITH tells where in the tree to begin. These are the rules:
%  - The position of PRIOR with respect to the CONNECT BY expressions determines which
%          expression identifies the root and which identifies the branches of the tree.
%  - A WHERE clause will eliminate individuals from the tree, but not their descendants (or
%          ancestors, depending on the location of PRIOR).
%  - A qualification in the CONNECT BY (particularly a not equal sign instead of the equal sign)
%          will eliminate both an individual and all of its descendants.
% - CONNECT BY cannot be used with a table join in the WHERE clause.
% 


Przydatne mechanizmy:
\begin{description}
    \item[\texttt{LEVEL}] \index{LEVEL@\texttt{LEVEL}}
        pseudokolumna,
        jest równa $1$ dla korzenia (lub węzła wskazanego przez \texttt{START WITH}), 
        dla dzieci tego węzła jest równa $2$, dla dzieci tych dzieci $3$, i tak dalej. 
    \item[\texttt{CONNECT\_BY\_ISLEAF}] \index{CONNECTBYISLEAF@\texttt{CONNECT\_BY\_ISLEAF}}
        pseudokolumna o wartości $1$ jeśli rekord jest liściem, $0$ w przeciwnym wypadku
    \item[\texttt{CONNECT\_BY\_ISCYCLE}]  \index{CONNECTBYISCYCLE@\texttt{CONNECT\_BY\_ISCYCLE}}
        pseudokolumna o wartości $1$ jeśli rekord ma potomka który też jest jego przodkiem, $0$ w przeciwnym wypadku
    \item[\texttt{SYS\_CONNECT\_BY\_PATH(kolumna\_wartości, znak\_separujacy)}] \index{SYSCONNECTBYPATH@\texttt{SYS\_CONNECT\_BY\_PATH()}}
        funkcja zwraca złączoną w listę wartości 
        z kolumny \texttt{kolumna\_wartości} wchodzące w skład ścieżki pomiędzy korzeniem a aktualnym węzłem.
        Każda wartość jest poprzedzona znakiem \texttt{znak\_separujacy}.
        Dla przykładu \texttt{SYS\_CONNECT\_BY\_PATH(name, '/')} może zwrócić \texttt{/Bazy danych/Obiektowe/db4o}.

        % It returns the path of a column value from root to node, with column values separated by char for each row returned by the CONNECT BY condition.
\end{description}


\temat{Operacje}

\operacja{Pobranie przodków}
%! method-sql connectby.ancestors

% \begin{verbatim}[sql]
% SELECT level, id, parent, name
%   FROM simple
%   START WITH id=7
%   CONNECT BY PRIOR parent =  id
% \end{verbatim}

\operacja{Pobieranie potomków}
%! method-sql connectby.descendants

% \begin{verbatim}[sql]
% SELECT level, id, parent, name
%   FROM simple
%   START WITH id=1
%   CONNECT BY parent = PRIOR id
% \end{verbatim}


\temat{Wydajność}

\begin{qxtab}{connectby}{Wydajność metody \texttt{CONNECT BY}}
%! result-table connectby deep3
\end{qxtab}

\begin{qxfig}{connectby}{Wydajność metody \texttt{CONNECT BY}}
%! result-chart connectby deep3
\end{qxfig}

\temat{Uwagi}

Oracle w wersjach wcześniejszych od 11g wykonując zapytanie korzystające z klauzuli \texttt{CONNECT BY}



% http://www.postgresql.org/docs/current/static/tablefunc.html
% wajig install postgresql-contrib-8.3

\podtemat{PostgreSQL \texttt{connectby()}}

PostgreSQL dostarcza moduł \texttt{tablefunc}\index{tablefunc} zawierający różne funkcje zwracające tabele.
% Ta funkcjonalność pojawiła się w wersji 7.4, natomiast dokumentacja dal niej dopiero w wersji 8.3.
Są one zaimplementowane w C\index{C} dla większej wydajności.

Jedną z dostępnych funkcji jest \texttt{connectby()}, będąca odpowiednikiem wyrażenia \texttt{CONNECT BY}. Jej składnia\todo{składnia?} wygląda następująco:

\begin{verbatim}[sql]
connectby(
  text relname,          -- nazwa tabeli
  text keyid_fld,        -- 
  text parent_keyid_fld, -- 
  [text orderby_fld,]    -- 
  text start_with,       -- 
  int max_depth,         -- 
  [text branch_delim])   -- 
\end{verbatim}




\index{metoda!connect by@\texttt{connect by}|)}




% \begin{verbatim}[sql]
% SELECT sum(empcount) FROM STRUCREL
%    CONNECT BY PRIOR superdept = deptid
%      START WITH deptname = 'Production';
% \end{verbatim}



% P main.py sql oracle 'SELECT level, id, parent, name FROM simple START WITH id=1 CONNECT BY parent = PRIOR id'
% +-------+----+--------+-------------+
% | level | id | parent | name        |
% +-------+----+--------+-------------+
% | 1     | 1  | None   | Bazy Danych |
% | 2     | 2  | 1      | Obiektowe   |
% | 3     | 3  | 2      | db4o        |
% | 2     | 4  | 1      | Relacyjne   |
% | 3     | 5  | 4      | Komercyjne  |
% | 3     | 6  | 4      | Open Source |
% | 4     | 7  | 6      | PostgreSQL  |
% | 4     | 8  | 6      | MySQL       |
% | 4     | 9  | 6      | SQLite      |
% | 2     | 10 | 1      | XML         |
% +-------+----+--------+-------------+
% 

% P main.py sql oracle 'SELECT level, id, parent, name FROM simple START WITH id=7 CONNECT BY PRIOR parent =  id'
% +-------+----+--------+-------------+
% | level | id | parent | name        |
% +-------+----+--------+-------------+
% | 1     | 7  | 6      | PostgreSQL  |
% | 2     | 6  | 4      | Open Source |
% | 3     | 4  | 1      | Relacyjne   |
% | 4     | 1  | None   | Bazy Danych |
% +-------+----+--------+-------------+




	\clearpage
\chapter{Typy danych}

%\chapter{Metody specyficzne dla bazy danych}

%    Opisane w tym rozdziale metody nie są uniwersalnymi reprezentacjami.
%
%\todo{Przepisać, bo stylistycznie to masakra}
%
%Metody PL/SQL, CTEs, connect by korzystają z reprezentacji krawędziowej.
%Metoda krawędziowa posiada wiele zalet lecz ma też znaczącą wadę - 
%SQL (sciślej piszącL te jego dialetaki jaki są dostępny w wszystkich bazach danych)
%nie pozwala na pobranie potomków w jednym zapytaniu. 
%Podobnnie jest z przodkami. 
%Natomiast niektre bazy danych posiadają mechanizmy umożliwjące na wygodne wykonywanie zapytań rekurencyjnych.
%
%
%Pozostałe metody też korzystają z omówionych poprzednio reprezentacji.
%Oferują za to mechanizmy ułatwiające operowanie tymi reprezentacjami, 
%jak też optymalizacje zwiększające wydajność i zmniejszające zrozmiar pola.


Standardowo bazy danych umożliwiają przechowywanie wielu typów danych.
Do powszechnie występujących można zaliczyć typy\footnote{Wymienione nazwy konkretnych typów pochodzą z bazy PostgreSQL}:
\begin{itemize}
    \item numeryczne
        (\verb|integer|, \verb|numeric|, \verb|real|, \ldots)
    \item znakowe
        (\verb|char|, \verb|varchar|, \verb|text|, \ldots)
    \item daty i czasu
        (\verb|date|, \verb|time|, \verb|timestamp|, \verb|interval|, \ldots)
    \item binarne
        (\verb|bytea|)
%    \item numeryczne
%        (\verb||, \verb||, \verb||, \verb||, \verb||, \verb||, )
%    \item numeryczne
%        (\verb||, \verb||, \verb||, \verb||, \verb||, \verb||, )
\end{itemize}

Ponadto bazy danych oferują własne, specyficzne dla implementacji typy.
Wśród nich warto wymienić typy tablicowe, sieciowe, logiczne, monetarne, wyliczeniowe oraz złożone.
Coraz częściej stosowane są też typy umożliwiające przechowywanie dokumentów XML.


Co najważniejsze --- z punktu widzenia tej pracy --- istnieją typy ułatwiające przechowywanie danych hierarchicznych.


 
    \clearpage
	% \section{PostgreSQL \texttt{ltree}}
	\section{PostgreSQL \texttt{ltree}}
\index{PostgreSQL}
\index{metoda!ltree@\texttt{ltree}|(textbf}

% Oleg Bartunov, Teodor Sigaev Oleg Bartunow, Teodor Sigaew
% Rok 2002, wersja 7.2
\temat{Opis modułu \texttt{ltree}}

\podtemat{Definicje}

\begin{description}
    \item[etykieta] \eng{label} węzła jest ciągiem jednego lub więcej słów rozdzielonych przez znak \verb|'_'|. 
        Słowa mogą zawierać litery i liczby.
        Długość etykiety nie może przekraczać 256 bajtów. 
        \todo{bajtów? znaków? czy mogą być pl-literki?}
    \item[ścieżka etykieta]
        \eng{label path} -- ciąg jednej lub więcej rozdzielonych kropkami etykiet $l_1.l_2...l_n$. 
        Reprezentuje ścieżkę korzenia do węzła. 
        Długość ścieżki etykiet jest ograniczony do $2^{16} - 1 = 65535 \approx 64 Kb$. 
        \todo{Kb? KB?} \todo{2Kb zalecane} 

%   \item[etykieta] \eng{label} 
\end{description}

\podtemat{Typy danych}

\begin{description}
    \item[\texttt{ltree}] -- typ danych
    \item[\texttt{ltree[]}]
    \item[\texttt{lquery}] wyrażenie ściekowe
    \begin{description}
        \item[\texttt{\{n\}}] asdf
        \item[\texttt{\{n,\}}] Dopasowuje dokładnie \emph{n} poziomów
        \item[\texttt{\{n,m\}}] Dopasowuje dokładnie \emph{n} poziomów
        \item[\texttt{\{,m\}}] Dopasowuje dokładnie \emph{n} poziomów
    \end{description}
\end{description}

\podtemat{Funkcje i operatory}

\begin{description}
    \item[\texttt{ltree subltree(ltree, start, end)}] \index{subltree@\texttt{subltree}}
    \item[\texttt{ltree subpath(ltree, offset[, len])}] \index{subpath@\texttt{subpath}}
    \item[\texttt{int4 nlevel(ltree)}] \index{nlevel@\texttt{nlevel}}
        zwraca poziom węzła, czyli ilość etykiet w ścieżce
    \item[\texttt{int4 index(ltree, ltree[, offset])}] \index{index@\texttt{index}}
    \item[\texttt{ltree text2ltree(text)}] \index{text2ltree@\texttt{text2ltree}}
    \item[\texttt{text ltree2text(ltree)}] \index{ltree2text@\texttt{ltree2text}}
    \item[\texttt{ltree lca(ltree[])}] \index{lca@\texttt{lca}}
\end{description}

\begin{description}
    \item[\texttt{<}, \texttt{>}, \texttt{<=}, \texttt{>=}, \texttt{=}, \texttt{<>}]
        have their usual meanings. Comparison is doing in the order of direct tree traversing, children of a node are sorted lexicographic.
    \item[\texttt{ltree @> ltree}]
        returns TRUE if left argument is an ancestor of right argument (or equal).
    \item[\texttt{ltree <@ ltree}]
        returns TRUE if left argument is a descendant of right argument (or equal).
    \item[\texttt{ltree \~{} lquery}, \texttt{lquery \~{} ltree}]
        returns TRUE if node represented by ltree satisfies lquery.
    \item[\texttt{ltree || ltree}, \texttt{ltree || text}, \texttt{text || ltree}]
        return concatenated ltree.
\end{description}


\podtemat{Indeksy}

\begin{description}
    \item[B-tree]\index{indeks!Btree@B-tree} na kolumnie \texttt{ltree} pozwala na skorzystanie z operatorów: \verb|<|, \verb|<=|, \verb|=|, \verb|>=|, \verb|>|
    \item[GiST]\index{indeks!GiST} na kolumnie \texttt{ltree} pozwala na skorzystanie z operatorów: 
        \verb|<|, \verb|<=|, \verb|=|, \verb|>=|, \verb|>|, \verb|<@|, \verb|@|, \verb|@>|, \verb|~|, \verb|?|.
        Indeksy GiST \eng{Generalized Search Tree} umożliwiają stosowanie różnych strategii indeksowania w zależności od potrzeb.
        Za ich pomocą może być zaimplementowane zarówno wyszukiwanie pełnotekstowe (\texttt{tsearch2}), 
        indeksowanie typu \texttt{hstore} (pozwalające na przechowywanie w jednym polu dowolnej ilości par klucz--wartość).
        Oczywiście wspiera również typ \texttt{ltree}.
\end{description}

Dla przykładu można stworzyć następujące indeksy:
\begin{verbatim}[sql]
CREATE INDEX test_path_idx_btree ON test USING btree (path);
CREATE INDEX test_path_idx_gist  ON test USING gost  (path);
\end{verbatim}



\temat{Operacje}

\operacja{Reprezentacja w SQL}
%! method-sql ltree.create

\operacja{Wstawianie danych}
%! method-sql ltree.insert

\operacja{Pobranie korzeni}
%! method-sql ltree.roots

\operacja{Pobranie rodzica}
%! method-sql ltree.parent

\operacja{Pobranie dzieci}
%! method-sql ltree.children

\operacja{Pobranie przodków}
%! method-sql ltree.ancestors

\operacja{Pobieranie potomków}
%! method-sql ltree.descendants


\temat{Wydajność}

\begin{qxtab}{ltree}{Wydajność metody \texttt{ltree}}
%! result-table ltree deep3
\end{qxtab}

\begin{qxfig}{ltree}{Wydajność metody \texttt{ltree}}
%! result-chart ltree deep3
\end{qxfig}







\index{metoda!ltree@\texttt{ltree}|)}
	\clearpage
	
	% \section{Microsoft SQL Server \texttt{hierarchyid}}
	\section{Microsoft SQL Server \texttt{hierarchyid}}
\index{metoda!hierarchyid@\texttt{hierarchyid}|(textbf}\index{SQL Server}

	% http://technet.microsoft.com/en-us/library/bb677173.aspx
	% http://blogs.msdn.com/manisblog/archive/2007/08/17/sql-server-2008-hierarchyid.aspx
	% http://www.microsoft.com/poland/technet/bazawiedzy/centrumrozwiazan/cr314_01.mspx


% Apparently the HierarchyID uses the ORDPATH algorithm (as far as I'm concerned). 
% Just found the following document that elaborates somewhat on the algorithm and other things related to hierarchical storage:
% http://sites.computer.org/debull/a07mar/kumaran.pdf


% SQL Server 2008 adds a new feature to help with modeling hierarchical relationships: the HIERARCHYID data type. It provides compact storage and convenient methods to manipulate hierarchies. In a way it is very much like optimized materialized path. In addition the SqlHierarchyId CLR data type is available for client applications. 
% 
% While HIERARCHYID has a lot to offer in terms of operations with hierarchical data, it is important to understand a few basic concepts:
% 
% - HIERARCHYID can have only a single root (although easy to work around by adding sub-roots)
% - It does not automatically represent a tree, the application has to define the relationships and enforce all rules 
% - The application needs to maintain the consistency
% http://pratchev.blogspot.com/2008/05/hierarchies-in-sql-server-2008.html

% http://technet.microsoft.com/en-us/library/cc721270.aspx  !! TODO
% The new HIERARCHYID data type in SQL Server 2008 is a system-supplied CLR UDT that can be useful for storing and manipulating hierarchies. This type is internally stored as a VARBINARY value that represents the position of the current node in the hierarchy (both in terms of parent-child position and position among siblings). You can perform manipulations on the type by using either Transact-SQL or client APIs to invoke methods exposed by the type. Let’s look at indexing strategies for the HIERARCHYID type, how to use the type to insert new nodes into a hierarchy, and how to query hierarchies.

%% TODO: 
% - opis algorytmu (zoptymalizowane ścieżki zmaterializowane)
% - opis wszystkich funkcji
% - http://www.cs.umb.edu/~poneil/ordpath.pdf google:ORDPATH
% - http://msdn.microsoft.com/en-us/library/bb677173.aspx


%  Traceback (most recent call last):
%   File "main.py", line 147, in <module>
%     main()
%   File "main.py", line 131, in main
%     print run_test(database, bases[database][i], testcases)
%   File "main.py", line 62, in run_test
%     tree.create_table()
%   File "Q:\src\methods.py", line 963, in create_table
%     """
%   File "Q:\src\pada.py", line 443, in execute
%     self._cur.execute(asql)
%   File "C:\Python26\lib\site-packages\pymssql.py", line 196, in execute
%     raise OperationalError, e[0]
% pymssql.OperationalError: SQL Server message 6510, severity 16, state 11, line 3:
% This functionality requires .NET Framework 3.5 SP1. Please install .NET Framework 3.5 SP1 to use this functionality.

Jedną z najciekawszych nowości jakie Microsoft wprowadził w SQL Server 2008 jest nowy typ danych \texttt{hierarchyid}.
Pozwala on na wygodne przechowywanie danych hierarchicznych.





Reprezentacja pozwala na przechowywanie drzew uporządkowanych.

\temat{Opis typu \texttt{hierarchyid}}

\podtemat{Budowa}

Typ \texttt{hierarchyid}\index{hierarchyid@\texttt{hierarchyid}} jest przechowywany wewnętrznie jako \texttt{VARBINARY}.

W budowie wewnętrznej przypomina reprezentację zmaterializowanych krawędzi, 
lecz aby osiągnąć mały rozmiar pola zastosowano algorytm OrdPath\cite{ordpath,kumaran}\index{OrdPath}.
Pierwotnie \todo{upewnić się} skonstruowano go aby ułarwić pracę z danymi XML\index{XML} pozbawionymi schematu\todo{schematu?}.
Nastepnie został on zastosowany w implementacji typu \texttt{hierarchyid}.

% Podobieństwo do UTF-8

% \todo{Czy to miżna napisać tu? To wygląda na mieszaninę z metodyką testów}
% W przypadku danych testowych najdłuższe hierarchyid zajmuje tylko ___ gdy w implementacjii 'ręcznej' korzystającej z varchar jest to ___ 


% Jest to 

\podtemat{Indeksowanie}

SQL Server udostępnia dwie strategie indeksowania hierarchicznych danych:
\begin{description}
 \item[w głąb] \index{drzewo!wyszukiwanie!w głąb} 
 \item[wszerz] \index{drzewo!wyszukiwanie!wszerz} \todo{napisać o kolumnie OrgLevel}
 \end{description}



\begin{verbatim}[sql]
CREATE TABLE tree (
    node hierarchyid PRIMARY KEY CLUSTERED,
    level AS node.GetLevel(),
    name varchar(100)
);

CREATE CLUSTERED INDEX Org_Breadth_First
    ON tree(level, node);

CREATE UNIQUE INDEX Org_Depth_First
    ON tree(node);
\end{verbatim}


\podtemat{Użyteczne funkcje}

\begin{description}
  \item[\texttt{child.GetAncestor(n)}] \index{GetAncestor@\texttt{GetAncestor()}}
    Zwraca węzeł będący \texttt{n-tym} przodkiem danego węzła.
    Dla $n = 0$ zwraca tem sam element, dla $n = 1$ rodzica, dla $n = 2$ dziadka, itd.
    % 0 -> ten element; 1->rodzic; 2->dziadek
	%This method is useful to find the (nth ancestor of the given child node.

  \item[\texttt{parent.GetDescendant(child1, child2)}] \index{GetDescendant@\texttt{GetDescendant()}}
	This method is very useful to get the descendant of a given node. 
	It has a great significance in terms of finding the new descendant position get the descendants etc. 

	This function returns one child node that is a descendant of the parent. 
	If parent is NULL, returns NULL. 
	If parent is not NULL, and both child1 and child2 are NULL, returns a child of parent. 
	If parent and child1 are not NULL, and child2 is NULL, returns a child of parent greater than child1. 
	If parent and child2 are not NULL and child1 is NULL, returns a child of parent less than child2. 
	If parent, child1, and child2 are all not NULL, returns a child of parent greater than child1 and less than child2. 
	If child1 or child2 is not NULL but is not a child of parent, an exception is raised. 
	If child1 >= child2, an exception is raised.

  \item[\texttt{node.GetLevel()}] \index{GetLevel@\texttt{GetLevel()}}
	Zwraca liczbę całowitą będącą poziomem danego węzła w drzewie \todo{słownik}. 
	Korzeń ma poziom równy $0$. 
	Jako, że typ \texttt{hierarchyid} nie obsługuje lasów to aby to kompensować tworzy się sztuczny korzeń, a korzenie obsługiwanych drzew znajdują się na poziomie $1$.

	% This function will return an integer that represents the depth of this node in the current tree. 

  \item[\texttt{hierarchyid::GetRoot()}] \index{GetRoot@\texttt{GetRoot()}}
	This method will return the root of the hierarchy tree and this is a static method if you are using it within CLR. 
	It will return the data type hierarchyID. 

  \item[\texttt{parent.IsDescendant(child)}] \index{IsDescendant@\texttt{IsDescendant()}}
	This method returns true/false (BIT) if the node is a descendant of the parent. 

  \item[\texttt{hierarchyid::Parse (input)}] \index{Parse@\texttt{Parse()}}
	Parse converts the canonical string representation of a hierarchyid to a hierarchyid value. 
	Parse is called implicitly when a conversion from a string type to hierarchyid occurs. 
	Acts as the opposite of ToString(). 
	Parse() is a static method. 

  \item[\texttt{void Read( BinaryReader r )}] \index{Read@\texttt{Read()}}
	Read reads binary representation of SqlHierarchyId from the passed-in BinaryReader and sets the SqlHierarchyId object to that value. 
	Read cannot be called by using Transact-SQL. Use CAST or CONVERT instead.

  \item[\texttt{node.Reparent(oldRoot, newRoot)}] \index{Reparent@\texttt{Reparent()}}
	This is a very useful method which helps you to reparent a node i.e. suppose if we want to align an existing node 
	to a new parent or any other existing parent then this method is very useful. 

  \item[\texttt{node.ToString()}] \index{ToString@\texttt{ToString()}}
	This method is useful to get the string representation of the HierarchyID. 
	The method returns a string that is a nvarchar(4000) data type.


  \item[\texttt{void Write( BinaryWriter w )}] \index{Write@\texttt{Write()}}
	Write writes out a binary representation of SqlHierarchyId to the passed-in BinaryWriter. 
	Write cannot be called by using Transact-SQL. Use CAST or CONVERT instead.

 \end{description}


\temat{Reprezentacja}

Typ \texttt{hierarchyid} jest przeznaczony do przechowywania pojedyńczego drzewa drzewa.
Aby umożliwić mu przechowywanie lasu należy połączyć korzenie wszystkich drzew z dodatkowym węzłem.
Stanie się on korzeniem przechowywanego drzewa.

\begin{verbatim}[table] hierarchyid

>n0 _
id | node      | name
1  | /         | ROOT

>n1 n0
id | node      | name
2  | /1/       | Bazy Danych

>n2 n1
id | node      | name
3  | /1/1/     | Obiektowe

>n3 n2
id | node      | name
4  | /1/1/1/   | db4o

>n4 n1
id | node      | name
5  | /1/2/     | Relacyjne

>n5 n4
id | node      | name
6  | /1/2/1/   | Open Source

>n7 n5
id | node      | name
7  | /1/2/1/1/ | PostgreSQL

>n8 n5
id | node      | name
8  | /1/2/1/2/ | MySQL

>n9 n5
id | node      | name
9  | /1/2/1/3/ | SQLite

>n6 n4
id | node      | name
10  | /1/2/2/  | Komercyjne

>n10 n1
id | node      | name
11 | /1/3/     | XML

\end{verbatim}

\temat{Operacje}

\operacja{Reprezentacja w SQL}
%! method-sql hierarchyid.create

\operacja{Wstawianie danych}
%! method-sql hierarchyid.insert

\operacja{Pobranie korzeni}
%! method-sql hierarchyid.roots

\operacja{Pobranie rodzica}
%! method-sql hierarchyid.parent

\operacja{Pobranie dzieci}
%! method-sql hierarchyid.children

\operacja{Pobranie przodków}
%! method-sql hierarchyid.ancestors

\operacja{Pobieranie potomków}
%! method-sql hierarchyid.descendants



\temat{Wyniki}

\begin{table}[h!]
  \caption{Wyniki \texttt{hierarchyid}}
  \begin{center}
%! result-table hierarchyid deep3
  \end{center}
\end{table}

\begin{figure}[h!t]
  \caption{Wyniki \texttt{hierarchyid}}
  \label{fig:img_chart_simple}
  \begin{center}
%! result-chart hierarchyid deep3
  \end{center}
\end{figure}


\temat{Uwagi}

SQL Server udostępnia \texttt{PERSISTED}\index{PERSISTED@\texttt{PERSISTED}} --- umożliwiające dynamiczne tworzenie dynamicznych kolumn. \todo{lepiej to sformuować}
Przykładowo może zostać to użyte do łatwego pobrania 

\begin{verbatim}[sql]
CREATE TABLE tree (
  node hierarchyid PRIMARY KEY CLUSTERED,
  level AS node.GetLevel() PERSISTED,
  name varchar(100)
);

SELECT level, name FROM tree;
\end{verbatim}





\index{metoda!hierarchyid@\texttt{hierarchyid}|)}





% P main.py sql sqlserver  'SELECT *, master.dbo.fn_varbintohexstr(cast(node as varbinary)) a, node.ToString() t FROM herid'
% +----+------+-------------+----------+-----------+
% | id | node | name        | a        | t         |
% +----+------+-------------+----------+-----------+
% | 1  |      | ROOT        | None     | /         |
% | 2  | X    | Bazy Danych | 0x58     | /1/       |
% | 3  | Z�   | Obiektowe   | 0x5ac0   | /1/1/     |
% | 4  | Z�   | db4o        | 0x5ad6   | /1/1/1/   |
% | 5  | [@   | Relacyjne   | 0x5b40   | /1/2/     |
% | 6  | [V   | Komercyjne  | 0x5b56   | /1/2/1/   |
% | 7  | [Z   | Open Source | 0x5b5a   | /1/2/2/   |
% | 8  | [Z�  | PostgreSQL  | 0x5b5ab0 | /1/2/2/1/ |
% | 9  | [Z�  | MySQL       | 0x5b5ad0 | /1/2/2/2/ |
% | 10 | [Z�  | SQLite      | 0x5b5af0 | /1/2/2/3/ |
% | 11 | [�   | XML         | 0x5bc0   | /1/3/     |
% +----+------+-------------+----------+-----------+


% 0x58     | /1/       | 01011000                   01011 000
% 0x5ac0   | /1/1/     | 0101101011000000           01011 01011 000000
% 0x5ad6   | /1/1/1/   | 0101101011010110           01011 01011 01011 0
% 0x5b40   | /1/2/     | 0101101101000000           01011 01101 000000
% 0x5b56   | /1/2/1/   | 0101101101010110           01011 01101 01011 0
% 0x5b5a   | /1/2/2/   | 0101101101011010           01011 01101 01101 0
% 0x5b5ab0 | /1/2/2/1/ | 010110110101101010110000   01011 01101 01101 01011 0000
% 0x5b5ad0 | /1/2/2/2/ | 010110110101101011010000   01011 01101 01101 01101 0000
% 0x5b5af0 | /1/2/2/3/ | 010110110101101011110000   01011 01101 01101 01111 0000
% 0x5bc0   | /1/3/     | 0101101111000000           01011 01111000000
% --
% 0x68     | /2/       | 01101000                   01101 000
% 0x6ac0   | /2/1/     | 0110101011000000           01101 01011 000000
% 0x6b40   | /2/2/     | 0110101101000000           01101 01101 000000




	\clearpage

\chapter{Podsumowanie}

%! summary-chart


\appendix

% \chapter{Program testujący}
\chapter{Program testujący}

Do tej pracy jest dołączone oprogramowanie umożliwiające łatwe sprawdzanie
opisanych metod. 

% Aby zautomatyzować przeprowadzanie testów został stworzony program testujący. 


% \chapter{Metodyka testów}
\chapter{Metodyka testów wydajności}

Ważnyme elementem tej pracy jest sprawdzenie wydajności przedstawionych metod w bazach danych.
Zostaną tu przedstawione warunki w jakich odbywały się testy.

\section*{Wybrane SZDB}

Do testów zostały wybrane popularne, dostępne bezpłatnie (również do zastosowań komercyjnych) bazy danych. 
%W przypadku baz Open Source wykorzystano najnowsze, stabilne wersje.
Dla baz komercyjnych zostały wybrane ich darmowe edycje. 
Posiadają one ograniczenia co do wielkości baz danych, wykorzystania zasobów oraz zmniejszoną funkcjonalność. 
Specyfika testów sprawia jednak, że te ograniczenia nie miały znaczenia podczas testów.




% \begin{itemize}
%  \item PostgreSQL\index{PostgreSQL}
%  \item MySQL\index{MySQL}
%  \item SQLite\index{SQLite}
%  \item Oracle Database 10g Express Edition\index{Oracle}
%  \item IBM DB2 Express-C\index{IBM DB2}
%  \item Microsoft SQL Server 2008 Express\index{SQL Server}
% \end{itemize}

\begin{qxtab}{edgeenum}{Użyte wersje baz danych}
\begin{tabular}{l|l|l}
Baza danych                            & Wersja & Edycja \\
\hline
PostgreSQL\index{PostgreSQL}           &  8.4      & \\
MySQL\index{MySQL}                     &  5.1      & \\
SQLite\index{SQLite}                   &  3        & \\
Oracle Database\index{Oracle}          &  10g      & Express Edition \\
IBM DB2\index{IBM DB2}                 &  9.7      & Express-C \\
Microsoft SQL Server\index{SQL Server} &  2008 R2  & Express\\
\end{tabular}
\end{qxtab}

\todo{Napisać jaki typ tabelek został użyty w MySQL (MyISAM, InnoDB)}

% Wybrane zostały SZDB spełniające następujące cechy:
% \begin{itemize}
%  \item popularne
%  \item darmowe do użytku domowego 
% \end{itemize}


% Dla porównania innych rozwiązań zostały dobrane następujące bazy danych:
% 
% \begin{itemize}
%  \item db4o
%  \item berkeley DB
%  \item ??
% \end{itemize}



\section*{Dane testowe}

By móc porównać metody przechowywania danych hierarchicznych należy zapewnić jednorodne warunki testów.
Oznacza to konieczność wykonywania dokładnie tych samych operacji z takimi samymi parametrami w tej samej kolejności.
Aby to osiągnąć testy są przechowywane jako pliki XML (w podkatalogu \verb|src/data|).

%W tym celu zastosowano pliki 

Dane użyte podczas testów zostały wygenerowane w sposób automatyczny.
Opisują one las o wysokości $10$, zawierający dwa drzewa, 
z każdym węzłem wewnętrznym o stopniu $2$.
W efekcie całe drzewo ma $4094$ węzłów.



%Program który zajmuje się tym zadaniem został dołączony do tej pracy.
%
%Dane użyte podczas testów pochodzą to drzewo kategorii \url{http://www.dmoz.org/}.
%
%
%%\begin{qxtab}{dmozdist}{Liczebność węzłów na poszczegółnych poziomach w danych testowych}\end{qxfig}
%\begin{tabular}{|l|l@{ }l@{ }l@{ }l@{ }l@{ }l@{ }l@{ }l@{ }l@{ }l@{ }l@{ }l@{ }|}
%\hline
%Poziom     & 0      & 1     & 2     & 3     & 4     & 5     & 6     & 7     & 8     & 9     & 10    & 11 \\
%\hline
%Liczebność & 21     & 539   & 6259  & 25834 & 50654 & 41752 & 31325 & 21573 & 7388  & 2090  & 303   & 16 \\
%\hline
%\end{tabular}
%%\end{qxtab}
%
%
%Większość testów wymaga identyfikatorów węzłów. 
%Wybranie losowego identyfikatora z istniejących w bazie danych było by najprostrzym 
%ale też niedoskonałym \todo{obarczonym błędem} rozwiązaniem.
%Wynika to z tego, że w drzewe (o stopniu większym niż 2\todo{bardzo nieprecyzyjnie napisane}) 
%liście stanowią najliczniejszą grupę węzłów.
%Natomiast węzły o niskim poziomie są małoliczne.
%Dlatego zapytanie o potomków tak wylosowanego węzła było by zadaniem prostym, natomiast przodków --- trudnym (gdyż z dużym prawdopodobieństwem wylosowany by został liść).
%
%Aby zapobiec takiej sytuacji identyfikatory dla zapytań losowane są dwuetapowo:
%\begin{enumerate}
% \item najpierw losowany jest jeden z poziomów węzłów w drzewie
% \item nastepnia losowany jest jeden z węzłów identyfikatorów
%\end{enumerate}
%Ta procedura powtarza się do uzyskania wymaganej ilości identyfikatorów węzłów. 
%W obu losowaniach wykorzystywany jest \emph{dyskretny rozkład jednostajny}. 




% Aby przetestować przedstawione rozwiązania wykożystane zostały następujące zestawy danych:
% \begin{itemize}
%  \item mały test 100 dwupoziomowo
%  \item test mocnego zagłębienia
%  \item test małego zagłebienia
%  \item test 6 pozimów po 3 zagłębienia 
% \end{itemize}


% Parametry generatora danych:
% \begin{itemize}
%  \item minimalny poziom zagłębienia
%  \item maksymalny poziom zagłębienia
%  \item rozklad prawdopodobieństwa wygenerowania dzieci (\cite{asdf}, \cite[Ala]{asdf})
% \end{itemize}


\section*{Środowisko testowe}

Bazy danych były testowane na systemie operacyjnym \emph{Microsoft Windows 7 32bit}.
Wybór ten był podyktowany faktem, że jedna z testowanych baz (SQL Server) posiadała wersję wyłącznie na systemy z rodziny Windows.

Sprzęt, na którym wykonywane były testy, składał się z:
\begin{itemize}
  \item procesora Intel Core2 Quad 2.4 Ghz
  \item 4 GB pamięci DDR2 (3,25 GB pamięci dostępnej dla systemu)
  \item dysku HDD 1TB klasy ekonomicznej
\end{itemize}

Server bazy danych działał na tej samej maszynie co program testujący.
Zminejszało to opóźnienia sieciowe ale zwiększało zużycie procesora%
\footnote{W praktyce na maszynie czterordzeniowej czas procesora nie był problemem.}.

%Aby zminimalizować potencjalne problemy wynikające z instalacji wielu baz na 
%jednej maszynie dla potrzeb testów zostały stworzone dla nich odzielne maszyny wirtualne. 
%Wykorzystano VirtualBox 2.1. Każdej maszynie został przyznany 10GB virtualny dysk 
%o~stałym rozmiarze oraz 512 MB RAM. Program testujący działał również na maszynie wirtualnej, 
%więc ominięto problem przepustowości interfacu sieciowego. 
%\todo{lepsze nazwy; wersja; usunac powtórzenia}
%
%Bazy posiadające wersję dla systemu GNU/Linux zostały zainstalowane na nim. 
%Baza Microsoft SQL Server została zainstalowana na Windows XP Service Pack 2.

\section*{Optymalizacje}

%Najczęstrzym zastrzerzeniem odnoszącym się do wyników benchmarków
%
%

Aby nie faworyzować żadnej z testowanych baz ustanowione zostały poniższe zasady:
\begin{itemize}
    \item Bazy nie były w żaden sposób dostrajane \eng{database tuning}, czyli pracują z domyślnymi ustawieniami.
    \item Nie zastosowano indeksów\footnote{Jawnych indeksów --- klucze główne były stosowane}
        oraz kluczy obcych.
  %\item Algorytm każdej operacji był pisane tak by był czytelne i ogólne, a nie
    Zapytania były pisane z myślą by być czytelne i dobrze ilustrować koncepcje danej reprezentacji.
    Została więc zastowowana maksyma Donalda Knutha: \emph{,,przedwczesna optymalizacja jest źródłem wszelkiego zła''}.
\end{itemize}

\section*{Prezentacja wyników}

Wyniki wydajności są prezentowane na końcu opisu każdej z opisywanych metod.
Są podawane zarówno w postaći tabeli (umożliwiającej dokładną analizę wyników)
jak i wykresu słupkowego (dającego możliwość szybkiego przejżenia wyników).

%%Uwagi:
%% - 
%%
%%Prezentacja wyników: jako przepustowość (czyli większa wartość - lepiej)

Zastosowaną miarą wydajności jest \emph{przepustowość} (czyli ilość wykonanych zapytań w danym czasie).
Takie podejście jest intuicyjne dla ludzi --- większa wartość oznacza lepszy wynik. 




%%\bibliographystyle{plain}
%\bibliography{main}

\clearpage
\phantomsection
\addcontentsline{toc}{chapter}{Literatura}
\begin{thebibliography}{99}
\thispagestyle{empty}

\bibitem[Cor04]{cormen} Thomas H. Cormen, Charles E. Leiserson, Ronald L. Rivest, Clifford Stein, 
    \emph{Wprowadzenie do algorytmów},
    Wydawnictwa Naukowo-Techniczne, Warszawa 2004

\bibitem[Dro04]{drozdek} Adam Drozdek
    \emph{C++. Algorytmy i struktury danych},
    Helion, Gliwice 2004

\bibitem[Knu02]{knuth} Donald E. Knuth
    \emph{Sztuka programowania. Tomy 1-3},
    Wydawnictwa Naukowo-Techniczne, Warszawa 2002

\bibitem[Cel00]{calko-sql} Joe Celko,
    \emph{SQL. Zaawansowane techniki programowania},
    MIKOM, 2000

\bibitem[Cel04]{celko-tree} Joe Celko,
    \emph{Joe Celko`s Trees and Hierarchies in SQL for Smarties}, 
    Morgan Kaufmann, 2004

\bibitem[Wal08]{apress-sqlserver} Robert E. Walters, Michael Coles, Robert Rae, Fabio Ferracchiati, Donald Farmer
    \emph{Accelerated SQL Server 2008},
    Apress, 2008

% \bibitem[]{} 
%     \emph{},
%     
% \bibitem[]{} 
%     \emph{},
%     

\end{thebibliography} 


\clearpage
\phantomsection
\addcontentsline{toc}{chapter}{\bibname}
\nocite{*}
\bibliography{tex/9_bibliography}

%\listoffimakegures
%\listoftables

%\clearpage
%\phantomsection
%\addcontentsline{toc}{chapter}{\indexname}
%\printindex


\end{document}


