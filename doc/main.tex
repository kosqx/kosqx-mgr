\documentclass[10pt,a4paper,draft]{book}
%\documentclass[10pt,a4paper,draft]{book}
%\documentclass[sfheadings,a4paper,11pt]{mwbk}

\usepackage{polski}
\usepackage[utf8]{inputenc}

%\usepackage[utf8x]{inputenc}
%\usepackage{ucs}



\newif\ifpdf
\ifx\pdfoutput\undefined
   \pdffalse
\else
   \pdfoutput=1
   \pdftrue
\fi

\ifpdf
   \usepackage[pdftex]{graphics}
   \usepackage[pdftex]{color}
   \pdfcompresslevel9
   \pdfinfo
   { /Title (Sztuka kochania)
     /Author (Michalina Wisłocka)
     /Subject ()
     /Keywords ()
   }
\else
   \usepackage{graphics}
   \usepackage{color}
\fi

\author{Krzysztof Kosyl}
\title{Przechowywanie danych hierarchicznych w relacyjnych bazach danych}

\usepackage[pdftex,unicode=true,backref]{hyperref}
\hypersetup{
  colorlinks=true,
  bookmarks=true
}

% \usepackage{indentfirst}
% \sloppy
% \clubpenalty = 10000
% \widowpenalty = 10000
% \frenchspacing

% \usepackage[pdftex]{hyperref}
% \hypersetup{
%   colorlinks=true,
%   bookmarks=true
% }

% \usepackage{amsmath}
% \usepackage{amsfonts}
% \usepackage{amssymb}
% \usepackage{makeidx}


% \usepackage[pdftex]{hyperref}
% \usepackage[pdftex]{graphics}
% \usepackage[pdftex]{color}




% \usepackage[pdftex]{hyperref}
% \hypersetup{
%   colorlinks=true,
%   bookmarks=true
% }


% \pdfcompresslevel9
% \pdfinfo
% { /Title (Sztuka kochania)
%      /Author (Michalina Wis�ocka)
%      /Subject ()
%      /Keywords ()
%    }

% 
% \hypersetup{
%   pdfauthor = {Krzysztof Kosyl},
%   pdftitle = {Przechowywanie danych hierarchicznych w relacyjnych bazach danych},
%   pdfsubject = {Subject},
%   pdfkeywords = {Keyword1, Keyword2},
%   pdfcreator = {LaTeX with hyperref package},
%   pdfproducer = {dvips + ps2pdf}
% }





\begin{document}
\maketitle{}
\tableofcontents{}

\chapter{Wstęp}

\include{tex/01_intro}


\chapter{Wprowadzenie do tematu}

\section{Co to jest drzewo}

Tu: matematyczny opis drzew, ich podstawowe własności, omówienie terminów:
\begin{itemize}
 \item drzewo
 \item las
 \item rodzic
 \item przodek
 \item korzeń
 \item dziecko
 \item potomek
 \item głębokość, mini
\end{itemize}



\section{Podstawowe algorytmy dla drzew}
\subsection{Wyszukiwanie w głąb}
\subsection{Wyszukiwanie w szerz}


\section{Różnice pomiędzy drzewami w algorytmice a w bazach danych}

Nie ma sensu coś takiego jak drzewo czerwono-czarne gdyż w drzewach w bazach danych chodzi o strukturę a nie o optywamizację czasu dostępu.

\section{Tematy porównania}
\subsection{Operacje}
\begin{description}
 \item[pobranie rodzica] polega na pobraniu rodzica bieżącego elementu
 \item[pobranie przodków] polega na pobraniu rodzica, rodzica rodzica aż do korzenia elemętów
 \end{description}

\section{Dostępność}
\subsection{Mapowanie relacyjno-obiektowe}
\subsection{W zależności od bazy danych}

\chapter{Metody przechowywania danych}

\section{Metoda stałej wysokości drzewa}

\section{Metoda klucza obcego do rodzica}

\section{Metoda zagnieżdzonych zbiorów}

\section{Metoda łączona}

\section{Metoda pełnych ścieżek}

\section{Metoda drzew prefiksowych (trie)}

\section{Metody specyficzne dla bazy danych}
\subsection{PostgreSQL \texttt{ltree}}
\subsection{Oracle \texttt{connect by}}
\subsection{DB2 \texttt{with}}

\chapter{Porównanie wydajności}

\section{Metodyka testów}

Dane użyte podczas testów zostały wygenerowane losowo w sposób automatyczny. Program który zajmuje się tym zadaniem został dołączony do tej pracy.

\chapter{Zastosowania i przykłady}

\section{LDAP}
\section{DNS?}
\section{XML}
\subsection{}


\end{document}


