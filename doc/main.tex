\documentclass[12pt,a4paper,intlimits,oneside]{report}
%\documentclass[10pt,a4paper,oneside]{book}
%\documentclass[10pt,a4paper,draft]{book}
%\documentclass[sfheadings,a4paper,11pt]{mwbk}

\usepackage{polski}
\usepackage[utf8]{inputenc}

\usepackage{ifpdf}

% \setcounter{tocdepth}{1}
\usepackage{geometry}
\geometry{verbose,a4paper,tmargin=2.5cm,bmargin=2.5cm,lmargin=3.5cm,rmargin=2.5cm}

%\usepackage[utf8x]{inputenc}
%\usepackage{ucs}

% \newif\ifpdf
% \ifx\pdfoutput\undefined
%    \pdffalse
% \else
%    \pdfoutput=1
%    \pdftrue
% \fi

\ifpdf
   \usepackage[pdftex]{graphics}
   \usepackage[pdftex]{color}
   \pdfcompresslevel9
\else
   \usepackage{graphics}
   \usepackage{color}
\fi

\usepackage{indentfirst}
\sloppy
\clubpenalty = 10000
\widowpenalty = 10000
\frenchspacing

% dla łatwiejszego sprawdzania
% \linespread{1.3}
% \linespread{1.0}
\setcounter{tocdepth}{1}

\usepackage{amsmath}
\usepackage{amsfonts}
\usepackage{amssymb}
\usepackage{makeidx}


\author{Krzysztof Kosyl}
\title{Przechowywanie danych hierarchicznych w relacyjnych bazach danych}

\makeindex

\newcommand{\HRule}{\rule{\linewidth}{0.5mm}}

\usepackage[unicode=true,pdftex,backref]{hyperref}
\hypersetup{
%kolory
  colorlinks=true,
  linkcolor=blue,
  %anchorcolor=black,
  %citecolor=green,
  %filecolor=magenta,
  %menucolor=red,
  %pagecolor=red,
  %urlcolor=cyan,
%zakladki
  bookmarks=true,
  bookmarksopen=true,
  bookmarksnumbered=true,
%metadane
  pdfauthor = {Krzysztof Kosyl},
  pdftitle = {Przechowywanie danych hierarchicznych w relacyjnych bazach danych},
  pdfsubject = {Trees in relational DB},
  pdfkeywords = {tree, db},
}

% numery stron u góry
\pagestyle{headings}




\begin{document}

%% desc: strona tytułowa

% To już nie jest używane
% \maketitle{}


\begin{titlepage}
	\begin{center}
		Uniwersytet Mikołaja Kopernika -- Wydział Matematyki i Informatyki
	\end{center}

	\vfill

	\begin{flushleft}
		Krzysztof Kosyl\\
		Numer albumu: 187411
	\end{flushleft}

	\vfill\vfill

	\begin{center}
		\huge{\textbf{Przechowywanie danych hierarchicznych w relacyjnych bazach danych}}
	\end{center}

	\vfill\vfill\vfill\vfill

	\begin{flushright}
		%Praca magisterska na kierunku:\\
		%Programowanie i Przetwarzanie Informacji
		Promotor:\\
		prof. dr hab Krzysztof Stencel
	\end{flushright}

	\vfill
	\begin{center}
		Toruń 2008
	\end{center}
\end{titlepage}

\input{tex/00_abstract}

\tableofcontents{}

% \chapter*{Wstęp}
To jest wstęp do pracy. Jak widać nie jest on obszerny.



% \chapter{Wprowadzenie do tematu}
\chapter{Wprowadzenie do tematu}

\section{Czym jest drzewo}
\index{drzewo|textbf}
Drzewo to bardzo powszechnie używane w informatyce pojęcie. W zależności od zastosowania może być różnie zdefiniowane.

\paragraph{Drzewo jako graf}
\index{graf}
\paragraph{Drzewo jako struktura rekurencyjna}
\index{rekurencja}


Tu: matematyczny opis drzew, ich podstawowe własności, omówienie terminów:
\begin{itemize}
    \item drzewo
    \item las
    \item rodzic
    \item przodek
    \item korzeń
    \item dziecko
    \item potomek
    \item głębokość, mini
\end{itemize}

\section{Podział drzew}
\subsection{Jednorodność}
Warto by dodać o drzewach jednorodnych oraz niejednorodnych.

\subsection{Kolejność}


\section{Podstawowe algorytmy dla drzew}
\subsection{Wyszukiwanie w głąb}
\index{drzewo!wyszukiwanie!w głąb}
\subsection{Wyszukiwanie wszerz}
\index{drzewo!wyszukiwanie!wszerz}


\section{Różnice pomiędzy drzewami w algorytmice a w bazach danych}

Nie ma sensu coś takiego jak drzewo czerwono-czarne gdyż w drzewach w bazach danych chodzi o strukturę a nie o optymalizację czasu dostępu.

\section{Tematy porównania - operacje}



\subsection{Operacje}
\subsection{Pobranie korzeni}
\index{drzewo!korzeń}
\subsection{Pobranie przodków}
\index{drzewo!przodekowie}
\subsection{Pobranie rodzica}
\index{drzewo!rodzic}
\subsection{Pobranie dzieci}
\index{drzewo!dzieci}
\subsection{Pobranie potomków}
\index{drzewo!potomkowie}
\subsection{Pobranie (najmłodszego) wspólnego przodka}



% \begin{description}
%   \item[pobranie rodzica] polega na pobraniu rodzica bieżącego elementu
%   \item[pobranie przodków] polega na pobraniu rodzica, rodzica rodzica aż do korzenia elementów
% \end{description}

\section{Dostępność}
\subsection{Mapowanie relacyjno-obiektowe}
\subsection{W zależności od bazy danych}


\section{Przyjęte założenia}
\todo{więcej wypisać, zastanowić się co z tym zrobić}

\paragraph{Tabela zawiera tylko jedną kolumnę z danymi użytkownika} Bez problemu można dodać więcej kolumn do tabeli. Obecność wyłącznie kolumny \texttt{name} zwiększa czytelność przykładów.
\paragraph{Zapytania wyłącznie związane z hierarchiczną strukturą danych}

\chapter{Podstawowe metody przechowywania danych}

	% \section{Metoda stałej wysokości drzewa}
	% Przenieść do sekcji o metodach, o których nie będziemy mówili. Takich jak również, Dział -- pracownik.
	
	% \section{Metoda klucza obcego do rodzica}
	\section{Metoda listy sąsiedztwa}


% Hierarchies like trees, organizational charts, ... are sometimes difficult to
% store in database tables. The most common database pattern to store 
% hierachical data is known as the adjacency model. It has been introduced 
% by the famous computer scientist Edgar Frank Codd. It's called the "adjacency"
% model because a reference to the parent data is stored in the same row as the
% child data, in an adjacent column. These kind of tables are also called self 
% referencing tables.
% http://www.scip.be/index.php?Page=ArticlesNET18


% It was Scott's pet cat called "tiger".
% And, who was Scott? His first name was Scott but Bruce. Bruce Scott was
% employee number #4 at the then Software Development Laboratories that
% eventually became Oracle. He co-authored and co-architected Oracle V1, V2 & V3.
% http://www.dba-oracle.com/t_scott_tiger.htm

% http://pl.wikipedia.org/wiki/Reprezentacja_grafu#Reprezentacja_przez_listy_s.C4.85siedztwa



Najprostszą, najbardziej intuicyjną i zapewne najpopularniejszą metodą jest metoda przyległej listy.
Została ona spopularyzowana przez Edgara Franka Codda. 
On też nadał im nazwę, która odnosi się do tego, że informacja o rodzicu elementu znajduje się w tej samej krotce co dane.

Zasługa spopularyzowania tej metody przypada Oracle. 
Dołączył on do swojego produktu przykładową bazę danych, nazywaną ,,Scott/Tiger'' 
korzystającą z tej metody.
\footnote{
    Nazwa tej bazy danych pochodzi od metody autoryzacji w bazie Oracle (login/hasło).
    Login pochodził z nazwiska jednego z pierwszych pracowników 
    Software Development Laboratories (przekształconych ostatecznie w Oracle) Bruce'a Scott'a. 
    Natomiast hasło to imię jego kota.
}

Na popularność metody przekłada się również jej znaczące podobieństwo do 
używanego między innymi w~językach C i~C++ sposobu przechowywania list i drzew.
Mianowicie każdy węzeł zawiera wskaźnik na rodzica. 
W relacyjnych bazach danych odpowiednikiem wskaźnika jest klucz obcy. 

\operacja{Definicja danych}

\begin{verbatim}[sql]
CREATE TABLE tree (
  id INTEGER PRIMARY KEY,
  parent INTEGER REFERENCES tree(id) ON DELETE CASCADE,
  value VARCHAR(100)
);
\end{verbatim}

\operacja{Pobieranie potomków}

Należy zwrócić uwagę na wykorzystanie operatora \emph{IN}. Bez niego wydajność metody spada znacząco. 

\operacja{Wstawianie danych}

\operacja{Wyniki}


\begin{table}[h!]
  \caption{Wyniki reprezentacji krawędziowej}
  \begin{center}
%! result-table simple deep3
  \end{center}
\end{table}

\begin{figure}[h!t]
  \caption{Wyniki reprezentacji krawędziowej}
  \label{fig:img_chart_simple}
  \begin{center}
%! result-chart simple deep3
  \end{center}
\end{figure}

\clearpage




	
	% \section{Metoda zagnieżdzonych zbiorów}
	\section{Metoda zagnieżdzonych zbiorów}
\index{metoda!zagnieżdzonych zbiorów|textbf}

Metoda została spopularyzowana przez Joe Celko\cite{celkosql}\index{Celko Joe}.



\operacja{Reprezentacja w SQL}
%! method-sql nested.create

\operacja{Wstawianie danych}
% %! method-sql nested.insert

% \begin{verbatim}[sql]
% INSERT INTO simple (parent, name) VALUES (:parent, :name)
% \end{verbatim}

Wstawianie danych w tej metodzie jest jej najsłabszą stroną.


\operacja{Pobranie korzeni}
%! method-sql nested.roots

\operacja{Pobranie rodzica}
% %! method-sql nested.parent

\operacja{Pobranie dzieci}
% %! method-sql nested.children

\operacja{Pobranie przodków}
%! method-sql nested.ancestors

\operacja{Pobieranie potomków}
%! method-sql nested.descendants


\begin{table}[h!]
  \caption{Wyniki reprezentacji zagnieżdzonych zbiorów}
   \begin{center}
%! result-table nested deep3
   \end{center}
\end{table}

\begin{figure}[h!t]
  \caption{Wyniki reprezentacji zagnieżdzonych zbiorów}
  \label{fig:img_chart_nested}
  \begin{center}
%! result-chart nested deep3
  \end{center}
\end{figure}


	
	% \section{Metoda pełnych ścieżek}
	\input{tex/23_full}
	
	% \section{Metoda drzew prefiksowych (trie)}
	\section{Metoda drzew prefiksowych (trie)}

	% zwane też lineage

	% drzewa binarne
	% http://commons.wikimedia.org/wiki/File:Binary_tree_in_array.svg
	% http://en.wikipedia.org/wiki/K-ary_tree


\chapter{Modifikacje metod przechowywania danych}
	% \section{Metoda łączona}


\chapter{Metody specyficzne dla bazy danych}
	% \section{PostgreSQL \texttt{ltree}}
	\input{tex/41_postgresql_ltree}
	
	% \section{Oracle \texttt{connect by}}
	\section{Oracle \texttt{connect by}}
\index{metoda!connect by@\texttt{connect by}|textbf}\index{Oracle}
% napisać czy to przeszukiwanie w głąb czy wszerz

% http://download.oracle.com/docs/cd/B19306_01/server.102/b14200/queries003.htm
% http://www.dba-oracle.com/t_sql99_with_clause.htm

System zarządzania bazą danych Oracle nie posiada możliwości przetwarzania danych hierarchicznych za pomocą klauzuli \texttt{with}.
W prawdzie jest ona dostępna od wersji \emph{Oracle 9i release 2} ale służy wyłacznie do pracy z podzapytaniami.

W to miejsce \emph{Oracle} udostępnia własne rozszerzenie. 

\begin{verbatim}[sql]
SELECT sum(empcount) FROM STRUCREL
   CONNECT BY PRIOR superdept = deptid
     START WITH deptname = 'Production';
\end{verbatim}


	% \section{DB2 \texttt{with}}
	\section{IBM DB2 \texttt{with}}
\index{metoda!with@\texttt{with}|textbf}\index{IBM DB2}\index{SQL Server}
% http://www.ibm.com/developerworks/data/library/techarticle/0307steinbach/0307steinbach.html

Opis standardowej metody with która pojawiła się w standardzie SQL:1999. Porównanie z connect by

W standardzie SQL:99\index{SQL!SQL:99} wprowadzono rozszerzenie \texttt{WITH} pozwalające na rekurencyjne wykonywanie zapytań w bazie danych.

\begin{verbatim}[sql]
WITH temptab(deptid, empcount, superdept) AS
   (    SELECT root.deptid, root.empcount, root.superdept
            FROM departments root
            WHERE deptname='Production'
     UNION ALL
        SELECT sub.deptid, sub.empcount, sub.superdept
            FROM departments sub, temptab super
            WHERE sub.superdept = super.deptid
   )
SELECT sum(empcount) FROM temptab
\end{verbatim}


	

\chapter{Porównanie wydajności}
\section{Po czym porównywać?}
\subsection{Metodzie}
\subsection{Bazie danych}



Dane użyte podczas testów zostały wygenerowane losowo w sposób automatyczny. Program który zajmuje się tym zadaniem został dołączony do tej pracy.

\chapter{Możliwości korzystania z danych hierarchicznych w O/RM}
% \section{Active Record}
% \section{Hibernate}
% \subsection{HQL}
% Ten akapit opisuje HQL\index{HQL}.

\chapter{Zastosowania i przykłady}

% \section{LDAP}
% \section{DNS?}
% \section{XML}
% \subsection{Różnice pomiędzy XML a drzewami}
% \begin{itemize}
%  \item kolejność jest istotna
%  \item język XPath i ścieżki w dokumencie
% \end{itemize}

\appendix

\chapter{Metodyka testów}

\section{Wybrane SZDB}

Do testów zostały wybrane popularne, dostępne bezpłatnie\footnote{również do zastosowań komercyjnych} bazy danych. 
W przypadku baz Open Source wykorzystano najnowsze, stabilne wersje.
Dla baz komercyjnych zostały wybrane ich darmowe edycje. Posiadają one
ograniczenia co do wielkości baz danych, wykorzystania zasobów oraz zmniejszoną funkcjonalność. 
Specyfika testów sprawia jednak, że te ograniczenia 


\begin{itemize}
 \item PostgreSQL\index{PostgreSQL}
 \item MySQL\index{MySQL}
 \item SQLite
 \item Oracle Database 10g Express Edition
 \item IBM DB2 Express-C
 \item Microsoft SQL Server 2008 Express
\end{itemize}

Aby zminimalizować potencjalne problemy wynikające z instalacji wielu baz na jednej maszynie dla potrzeb testów zostały stworzone dla nich odzielne maszyny wirtualne. 

Bazy posiadające wersję dla systemu GNU/Linux zostały zainstalowane na nim. Baza Microsoft SQL Server została zainstalowana na Windows XP Service Pack 2.



% Wybrane zostały SZDB spełniające następujące cechy:
% \begin{itemize}
%  \item popularne
%  \item darmowe do użytku domowego 
% \end{itemize}


% Dla porównania innych rozwiązań zostały dobrane następujące bazy danych:
% 
% \begin{itemize}
%  \item db4o
%  \item berkeley DB
%  \item ??
% \end{itemize}

\section{Dane testowe}



Aby przetestować przedstawione rozwiązania wykożystane zostały następujące zestawy danych:
\begin{itemize}
 \item mały test 100 dwupoziomowo
 \item test mocnego zagłębienia
 \item test małego zagłebienia
 \item test 6 pozimów po 3 zagłębienia 
\end{itemize}


Parametry generatora danych:
\begin{itemize}
 \item minimalny poziom zagłębienia
 \item maksymalny poziom zagłębienia
 \item rozklad prawdopodobieństwa wygenerowania dzieci (\cite{asdf}, \cite[Ala]{asdf})
\end{itemize}


\chapter{Program testujący}

Do tej pracy jest dołączone oprogramowanie umożliwiające łatwe sprawdzanie
opisanych metod. 

% Aby zautomatyzować przeprowadzanie testów został stworzony program testujący. 


%\bibliographystyle{plain}
%\bibliography{main}

\clearpage
\phantomsection
\addcontentsline{toc}{chapter}{Literatura}
\begin{thebibliography}{99}
\thispagestyle{empty}

\bibitem{asdf} Janek Kowal, 
    \emph{Coś na temat}, 
    WNT
\bibitem{celko} Joe Celko, 
    \emph{Joe Celko`s Trees and Hierarchies in SQL for Smarties}, 
    Morgan Kaufmann

\end{thebibliography} 

\end{document}




%\listoffigures
%\listoftables

\clearpage
\phantomsection
\addcontentsline{toc}{chapter}{\indexname}
\printindex


\end{document}
