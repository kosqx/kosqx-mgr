\documentclass[10pt,a4paper,oneside]{book}
%\documentclass[10pt,a4paper,draft]{book}
%\documentclass[sfheadings,a4paper,11pt]{mwbk}

\usepackage{polski}
\usepackage[utf8]{inputenc}

%\usepackage[utf8x]{inputenc}
%\usepackage{ucs}

\newif\ifpdf
\ifx\pdfoutput\undefined
   \pdffalse
\else
   \pdfoutput=1
   \pdftrue
\fi

\ifpdf
   \usepackage[pdftex]{graphics}
   \usepackage[pdftex]{color}
   \pdfcompresslevel9
%    \pdfinfo
%    { /Title (Przechowywanie danych hierarchicznych w relacyjnych bazach danych)
%      /Author (Krzysztof Kosyl)
%      /Subject ()
%      /Keywords ()
%    }
\else
   \usepackage{graphics}
   \usepackage{color}
\fi

\usepackage{indentfirst}
\sloppy
\clubpenalty = 10000
\widowpenalty = 10000
\frenchspacing


\usepackage{amsmath}
\usepackage{amsfonts}
\usepackage{amssymb}
\usepackage{makeidx}


\author{Krzysztof Kosyl}
\title{Przechowywanie danych hierarchicznych w relacyjnych bazach danych}



\usepackage[unicode=true,pdftex,backref]{hyperref}
\hypersetup{
%glowne
%kolory
  colorlinks=true,
  linkcolor=blue,
  %anchorcolor=black,
  %citecolor=green,
  %filecolor=magenta,
  %menucolor=red,
  %pagecolor=red,
  %urlcolor=cyan,
%zakladki
  bookmarks=true,
  bookmarksopen=true,
  bookmarksnumbered=true,
%metadane
  pdfauthor = {Krzysztof Kosyl},
  pdftitle = {Przechowywanie danych hierarchicznych w relacyjnych bazach danych},
  pdfsubject = {Trees in relational DB},
  pdfkeywords = {tree, db},
  %pdfcreator = {LaTeX with hyperref package},
  %pdfproducer = {dvips + ps2pdf}
}





\begin{document}

\begin{titlepage}
\begin{center}
 Uniwersytet Mikołaja Kopernika -- Wydział Matematyki i Informatyki
\end{center}
\vfill
\begin{center}
 \large{Tytuł}
\end{center}

\vfill
\begin{center}
 Toruń 2008
\end{center}
\end{titlepage}

% \begin{abstract}
%  O czym jest ten dokument.
% \end{abstract} 

\maketitle{}

\tableofcontents{}

\chapter{Wstęp}

\section{Wstęp}

To jest wstęp do pracy. Jak widać nie jest on obszerny.



\chapter{Wprowadzenie do tematu}

\section{Co to jest drzewo}

Tu: matematyczny opis drzew, ich podstawowe własności, omówienie terminów:
\begin{itemize}
 \item drzewo
 \item las
 \item rodzic
 \item przodek
 \item korzeń
 \item dziecko
 \item potomek
 \item głębokość, mini
\end{itemize}



\section{Podstawowe algorytmy dla drzew}
\subsection{Wyszukiwanie w głąb}
\subsection{Wyszukiwanie w szerz}


\section{Różnice pomiędzy drzewami w algorytmice a w bazach danych}

Nie ma sensu coś takiego jak drzewo czerwono-czarne gdyż w drzewach w bazach danych chodzi o strukturę a nie o optywamizację czasu dostępu.

\section{Tematy porównania}
\subsection{Operacje}
\begin{description}
 \item[pobranie rodzica] polega na pobraniu rodzica bieżącego elementu
 \item[pobranie przodków] polega na pobraniu rodzica, rodzica rodzica aż do korzenia elemętów
 \end{description}

\section{Dostępność}
\subsection{Mapowanie relacyjno-obiektowe}
\subsection{W zależności od bazy danych}

\chapter{Metody przechowywania danych}

\section{Metoda stałej wysokości drzewa}

\section{Metoda klucza obcego do rodzica}

\section{Metoda zagnieżdzonych zbiorów}

\section{Metoda łączona}

\section{Metoda pełnych ścieżek}

\section{Metoda drzew prefiksowych (trie)}

\section{Metody specyficzne dla bazy danych}
\subsection{PostgreSQL \texttt{ltree}}
\subsection{Oracle \texttt{connect by}}
\subsection{DB2 \texttt{with}}
Opis standardowej metody with która pojawiła się w standardzie SQL??. Porównanie z connect by

\chapter{Porównanie wydajności}
\section{Po czym porównywać?}
\subsection{Metodzie}
\subsection{Bazie danych}



Dane użyte podczas testów zostały wygenerowane losowo w sposób automatyczny. Program który zajmuje się tym zadaniem został dołączony do tej pracy.

\chapter{Możliwości korzystania z danych hierarchicznych w O/RM}
\section{Active Record}
\section{Hibernate}
\subsection{HQL}
\chapter{Zastosowania i przykłady}

\section{LDAP}
\section{DNS?}
\section{XML}
\subsection{Różnice pomiędzy XML a drzewami}
\begin{itemize}
 \item kolejność jest istotna
 \item język XPath i ścieżki w dokumencie
\end{itemize}

\appendix

\chapter{Opis Python DB API}

\chapter{Załączone programy}

\chapter{Metodyka testów}
\section{Wybrane SZDB}
\begin{itemize}
 \item PostgreSQL
 \item MySQL
 \item SQLite
 \item Firebirld?
 \item Oracle
 \item DB2
 \item MS SQL Server
\end{itemize}


Wybrane zostały SZDB spełniające następujące cechy:
\begin{itemize}
 \item popularne
 \item darmowe do użytku domowego 
\end{itemize}


Dla porównania innych rozwiązań zostały dobrane następujące bazy danych:

\begin{itemize}
 \item db4o
 \item berkeley DB
 \item ??
\end{itemize}

\section{Dane testowe}

Aby przetestować przedstawione rozwiązania wykożystane zostały następujące zestawy danych:
\begin{itemize}
 \item mały test 100 dwupoziomowo
 \item test mocnego zagłębienia
 \item test małego zagłebienia
 \item test 6 pozimów po 3 zagłębienia 
\end{itemize}


Parametry generatora danych:
\begin{itemize}
 \item minimalny poziom zagłębienia
 \item maksymalny poziom zagłębienia
 \item rozklad prawdopodobieństwa wygenerowania dzieci
\end{itemize}




\end{document}
